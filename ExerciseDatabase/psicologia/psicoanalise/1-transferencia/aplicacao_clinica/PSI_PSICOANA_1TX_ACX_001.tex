% Exercise ID: PSI_PSICOANA_1TX_ACX_001
% Created: 2025-11-30
% Difficulty: 2/5

\exercicio{Transferência em contextos clínicos — versão para jovens adultos

Leia atentamente as três vinhetas clínicas e, para cada uma, responda de forma concisa (máx. 6 linhas por resposta):

\subexercicio{1} Paciente A (jovem adulto em transição): Após terminar o secundário e começar a faculdade/trabalho, o paciente apresenta comportamentos repetidos em sessão que remetem para relações familiares anteriores:
(i) pede frequentemente a opinião do terapeuta sobre decisões pessoais importantes (curso, emprego, namoro) e interpreta a concordância como prova de apoio incondicional;  
(ii) quando o terapeuta sugere limites (p. ex. não comentar assuntos clínicos nas redes), o paciente reage com silêncio prolongado ou ironia, como fazia quando o progenitor reagia com retirada emocional;  
(iii) procura fora de sessão contacto e elogios por mensagens, ficando ansioso se não obtém resposta rápida — reproduzindo uma dependência afetiva prévia.  
Perguntas: a) Identifique que formas de transferência estão presentes (indique positiva/negativa quando apropriado). b) Explique brevemente como essas transferências podem influenciar a aliança terapêutica e proponha uma atitude clínica simples do terapeuta para as gerir.

\subexercicio{2} Paciente B (terapia de grupo): Um membro idealiza a terapeuta e espera que esta resolva todos os conflitos do grupo, tal como fazia com uma figura maternal na infância.  
Perguntas: a) Identifique a transferência e diferencie-a da admiração. b) Sugira uma intervenção breve que a terapeuta pode usar no grupo.

\subexercicio{3} Paciente C (adolescente): O adolescente recusa autoridade e desafia regras, comportamento que remete para rebeldia contra uma professora que o punia injustamente.  
Perguntas: a) Explique como a transferência reproduz padrões relacionais anteriores. b) Aponte uma estratégia concreta para o terapeuta lidar com esta transferência sem confrontos improdutivos.

\texto{Orientações:} Responda de forma clara e objetiva, usando conceitos básicos de psicanálise (transferência, transferência positiva/negativa, contratransferência) quando pertinente. Cada subexercício vale 1 ponto.}
