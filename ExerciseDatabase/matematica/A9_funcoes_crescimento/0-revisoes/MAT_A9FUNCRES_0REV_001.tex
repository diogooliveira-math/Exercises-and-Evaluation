% meta:
% id: MAT_A9FUNCRES_0REV_001
% module: A9_funcoes_crescimento
% concept: 0-revisoes
% concept_name: Revisões de Crescimento
% difficulty: 2
% tags: revisao,crescimento,funcoes
% author: sistema
\exercicioDesenvolvimento{
Considera a correspondência seguinte entre pessoas e o número de sapatos que calçam:
\[
\begin{array}{c|c}
\text{Pessoa} & \text{Número de sapatos que calça} \\
\hline
\text{Ana} & 37 \\
\text{Bruno} & 42 \\
\text{Carla} & 39 \\
\text{David} & 42 \\
\end{array}
\]
\textbf{Pergunta:} Esta correspondência é uma função? Justifica escolhendo a opção correta e explicando por que as outras estão erradas.
\begin{enumerate}
 \item[(A)] Não é uma função, porque o número 42 aparece duas vezes.
 \item[(B)] É uma função, porque cada pessoa está associada a um único número.
 \item[(C)] Não é uma função, porque Bruno e David calçam o mesmo número.
 \item[(D)] Não é uma função, porque há números repetidos na segunda coluna.
\end{enumerate}
}

\exercicioDesenvolvimento{
Considera a função $f(x)=2x-3$.
\begin{itemize}
  \item[a)] Calcula $f(0)$, $f(2)$ e $f(5)$.
  \item[b)] Qual é o valor de $x$ tal que $f(x)=1$?
\end{itemize}
}

\exercicioDesenvolvimento{
Uma receita de bolo de iogurte para 2 pessoas usa:
\begin{itemize}
  \item 1 iogurte natural
  \item 2 copos de açúcar
  \item 3 copos de farinha
  \item \tfrac{1}{2} copo de óleo
  \item 3 ovos
\end{itemize}
Pretende-se ajustar para 10 pessoas. Indica as quantidades proporcionais.
}

\exercicioDesenvolvimento{
O Clube A comprou um jogador por 12 milhões de euros, o que representa 20\% do seu orçamento anual. Determina o orçamento total e discute se a percentagem é sustentável face a uma segunda contratação igual.
}
