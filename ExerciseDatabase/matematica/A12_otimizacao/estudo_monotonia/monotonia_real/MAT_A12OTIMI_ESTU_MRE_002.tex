% Exercise ID: MAT_A12OTIMI_ESTU_MRE_002
% Module: Módulo A12 - Otimização | Concept: Estudo da Monotonia
% Type: monotonia_real | Difficulty: 2/5
% Tags: monotonia, derivadas, contexto_real, altura
% Author: Professor | Date: 2025-11-25

\exercicio{
Uma bola é lançada para cima. A sua altura $h$ (em metros) em função do tempo $t$ (em segundos) é dada por:
\[
h(t) = -5t^2 + 20t + 1
\]
}

\subexercicio{Calcula a derivada $h'(t)$ que representa a velocidade da bola.}

\subexercicio{Durante quanto tempo a bola está a subir? Indica o intervalo de tempo.}

\subexercicio{A partir de que instante a bola começa a descer?}

\subexercicio{Qual é a altura máxima atingida pela bola e em que instante isso acontece?}
