\documentclass[12pt]{article}
\usepackage[utf8]{inputenc}
\usepackage[portuguese]{babel}
\usepackage{geometry}
\usepackage{amsmath, amssymb}
\usepackage{graphicx}
\usepackage{enumitem}
\usepackage{tikz}
\usepackage{placeins}
\geometry{a4paper, margin=2cm}

% Custom commands (add project-specific macros here)
\newcommand{\exercicio}[1]{\textbf{#1}}
\newcommand{\subexercicio}[1]{\item #1}

\begin{document}

\section*{Teste: P4 funcoes / 4 - funcao inversa}

\begin{enumerate}[label=\arabic*)]

\item % MAT_P4FUNCOE_4FIN_TRH_003
% Exercise ID: MAT_P4FUNCOE_4FIN_TRH_003
% Module: MÓDULO P4 - Funções | Concept: Função Inversa | Type: Teste da Reta Horizontal
% Difficulty: 2/5 (Fácil) | Type: desenvolvimento
% Points: 10 | Time: 10 min
% Tags: inversa, injetividade, teste_reta_horizontal, grafico
% Author: Professor | Date: 2025-11-18
% Status: active
% Description: Determinar quais funções são invertíveis usando teste da reta horizontal

\exercicio
Considere as funções representadas nas figuras seguintes:

\begin{figure}[H]
\centering
\begin{minipage}{0.45\textwidth}
\centering
\begin{tikzpicture}[scale=0.8]
    \begin{axis}[
        axis lines = middle,
        xlabel = $x$,
        ylabel = $y$,
        xmin = -4, xmax = 4,
        ymin = -2, ymax = 2,
        grid = major,
        grid style = {dashed, gray!30},
        width = 8cm,
        height = 6cm,
        title = {Função E},
        title style = {font=\bfseries},
    ]
    \addplot[domain=-3.14:3.14, samples=200, thick, red] {sin(deg(x))};
    \end{axis}
\end{tikzpicture}
\end{minipage}
\hfill
\begin{minipage}{0.45\textwidth}
\centering
\begin{tikzpicture}[scale=0.8]
    \begin{axis}[
        axis lines = middle,
        xlabel = $x$,
        ylabel = $y$,
        xmin = -1, xmax = 4,
        ymin = -1, ymax = 4,
        grid = major,
        grid style = {dashed, gray!30},
        width = 8cm,
        height = 6cm,
        title = {Função F},
        title style = {font=\bfseries},
    ]
    \addplot[domain=0:3, thick, blue] {2*x};
    \addplot[mark=*, mark size=2pt, blue] coordinates {(0,0)};
    \addplot[mark=*, mark size=2pt, blue] coordinates {(3,6)};
    \end{axis}
\end{tikzpicture}
\end{minipage}
\end{figure}



Quais das duas funções são invertíveis (isto é, cuja inversa também é uma função)? Justifique usando o teste da reta horizontal.

\vspace{3cm}
\FloatBarrier


\item % MAT_P4FUNCOE_4FIN_GRA_004
% Module: MÓDULO P4 - Funções | Concept: Função Inversa | Type: Determinação Gráfica
% Difficulty: 2/5 (Fácil) | Type: desenvolvimento
% Points: 10 | Time: 10 min
% Tags: inversa, grafico, simetria, funcao_quadratica
% Author: Professor | Date: 2025-11-18
% Status: active
% Description: Dado o gráfico de ramos de funções, desenhar o gráfico da inversa

\exercicio
Na figura está representado o gráfico de uma função $f$ definida em $[1, 4]$. Represente, no referencial dado, o gráfico da função inversa $f^{-1}$.

\begin{figure}[ht]
\centering
\begin{tikzpicture}[scale=0.8]
    \begin{axis}[
        axis lines = middle,
        xlabel = $x$,
        ylabel = $y$,
        xmin = -1, xmax = 5,
        ymin = -1, ymax = 5,
        grid = major,
        grid style = {dashed, gray!30},
        width = 8cm,
        height = 6cm,
    ]
    \addplot[domain=-1:5, dashed, gray!70, thin] {x};
    
    \addplot[domain=1:4, samples=100, thick, blue] {(x-1)^2 + 1};
    \addplot[mark=*, mark size=2pt, blue] coordinates {(1,1)};
    \addplot[mark=*, mark size=2pt, blue] coordinates {(4,10/2)};
    \node[anchor=west] at (axis cs:4,4.2) {$(4,4)$};
    \node[anchor=north east] at (axis cs:1,0.8) {$(1,1)$};
    \end{axis}
\end{tikzpicture}
\end{figure}

\bigskip

\exercicio
Na figura está representado o gráfico de uma função $g$ definida em $]0, 4]$. Represente, no referencial dado, o gráfico da função inversa $g^{-1}$.

\begin{figure}[H]
\centering
\begin{tikzpicture}[scale=0.8]
    \begin{axis}[
        axis lines = middle,
        xlabel = $x$,
        ylabel = $y$,
        xmin = -1, xmax = 5,
        ymin = -1, ymax = 5,
        grid = major,
        grid style = {dashed, gray!30},
        width = 8cm,
        height = 6cm,
    ]
    \addplot[domain=-1:5, dashed, gray!70, thin] {x};
    
    \addplot[domain=0.25:4, samples=100, thick, red] {1/x + 0.75};
    \addplot[mark=o, mark size=2pt, red] coordinates {(0.25,4.75)};
    \addplot[mark=*, mark size=2pt, red] coordinates {(4,1)};
    \node[anchor=south] at (axis cs:4,1.2) {$(4,1)$};
    \end{axis}
\end{tikzpicture}
\end{figure}
\vspace{3cm}
\FloatBarrier


\item % MAT_P4FUNCOE_4FIN_GRA_003
% Module: MÓDULO P4 - Funções | Concept: Função Inversa | Type: Determinação Gráfica
% Difficulty: 2/5 (Fácil) | Type: desenvolvimento
% Points: 10 | Time: 10 min
% Tags: inversa, grafico, simetria, funcao_quadratica
% Author: Professor | Date: 2025-11-18
% Status: active
% Description: Dado o gráfico de ramos de funções, desenhar o gráfico da inversa

\exercicio
Na figura está representado o gráfico de uma função $f$ definida em $[0, +\infty[$. Represente, no referencial dado, o gráfico da função inversa $f^{-1}$.

\begin{figure}[ht]
\centering
\begin{tikzpicture}[scale=0.8]
    \begin{axis}[
        axis lines = middle,
        xlabel = $x$,
        ylabel = $y$,
        xmin = -1, xmax = 5,
        ymin = -1, ymax = 5,
        grid = major,
        grid style = {dashed, gray!30},
        width = 8cm,
        height = 6cm,
    ]
    \addplot[domain=-1:5, dashed, gray!70, thin] {x};
    
    \addplot[domain=0:4, samples=100, thick, blue] {sqrt(2*x)};
    \addplot[mark=*, mark size=2pt, blue] coordinates {(0,0)};
    \addplot[mark=*, mark size=2pt, blue] coordinates {(2,2)};
    \node[anchor=south west] at (axis cs:2,2.2) {$(2,2)$};
    \node[anchor=north east] at (axis cs:0.2,0.2) {$(0,0)$};
    \end{axis}
\end{tikzpicture}
\end{figure}

\bigskip

\exercicio
Na figura está representado o gráfico de uma função $g$ definida em $[-1, 3]$. Represente, no referencial dado, o gráfico da função inversa $g^{-1}$.

\begin{figure}[H]
\centering
\begin{tikzpicture}[scale=0.8]
    \begin{axis}[
        axis lines = middle,
        xlabel = $x$,
        ylabel = $y$,
        xmin = -2, xmax = 4,
        ymin = -2, ymax = 4,
        grid = major,
        grid style = {dashed, gray!30},
        width = 8cm,
        height = 6cm,
    ]
    \addplot[domain=-2:4, dashed, gray!70, thin] {x};
    
    \addplot[domain=-1:3, thick, red] {-0.5*x+1.5};
    \addplot[mark=*, mark size=2pt, red] coordinates {(-1,2)};
    \addplot[mark=*, mark size=2pt, red] coordinates {(3,0)};
    \node[anchor=south east] at (axis cs:-1,2) {$(-1,2)$};
    \node[anchor=north west] at (axis cs:3,0) {$(3,0)$};
    \end{axis}
\end{tikzpicture}
\end{figure}
\vspace{3cm}
\FloatBarrier


\item % MAT_P4FUNCOE_4FX_DAX_021
% Created: 2025-11-27
% Difficulty: 2/5

\exercicio{Dada a função $f(x) = 2x + 3$, determine a função inversa $f^{-1}(x)$ e justifique todos os passos do seu raciocínio.}
\FloatBarrier


\item % MAT_P4FUNCOE_4FX_DAX_022
% Created: 2025-11-27
% Difficulty: 2/5

\exercicio{Dada f(x)=2x+3 determine f^{-1} e justifique}
\FloatBarrier


\item % MAT_P4FUNCOE_4FX_DAX_002
% Module: MÓDULO P4 - Funções | Concept: Função Inversa | Type: Determinação Analítica da Função Inversa
% Difficulty: 2/5 (Fácil) | Format: standard
% Tags: algebra, simetria, resolucao_equacao, expressao_analitica, injetividade, inversa, sobrejetividade, calculo_analitico
% Author: Test Agent | Date: 2025-11-26
% Status: active

\exercicio{Determine a função inversa de f(x) = 2x + 3.}
\FloatBarrier

\end{enumerate}

\end{document}