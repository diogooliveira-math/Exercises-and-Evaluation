% Minimal sebenta template generated automatically
\documentclass[11pt,a4paper]{article}
\usepackage[utf8]{inputenc}
\usepackage[T1]{fontenc}
\usepackage{lmodern}
\usepackage{geometry}
\usepackage{fancyhdr}
\usepackage{hyperref}
\usepackage{graphicx}
\usepackage{float}
\usepackage{placeins}
\usepackage{bookmark}
\usepackage{booktabs}
\usepackage{amsmath,amssymb}
\usepackage{csquotes}
\usepackage{enumitem}
\usepackage{tikz}
\IfFileExists{pgfplots.sty}{\usepackage{pgfplots}\pgfplotsset{compat=1.17}}{}

\geometry{margin=2.5cm}

% Try to include project-specific style macros (containing \exercicio, \subexercicio, etc.)
% Try multiple relative locations to be robust across different generated output paths
\IfFileExists{../../../../Teste_modelo/config/style.tex}{% Sistema de exercícios com contadores automáticos
\newcounter{exerciciocount}          % Contador principal dos exercícios
\newcounter{subexerciciocount}       % Contador dos subexercícios
\newcounter{optioncount}             % Contador das opções

% Control whether the macro prints the automatic "Exercício N." heading.
% Default: show the heading. Call \showexerciciotitlefalse to suppress.
\newif\ifshowexerciciotitle
\showexerciciotitletrue

% Macro para exercício principal
\newcommand{\exercicio}[1]{%
        \par\vspace{1.5em}% Espaçamento antes
        \refstepcounter{exerciciocount}% Incrementa contador principal
        \setcounter{subexerciciocount}{0}% Reseta contador de subexercícios
        \setcounter{optioncount}{0}% Reseta contador de opções
        % Only print the automatic heading if the flag is true
        \ifshowexerciciotitle
            \noindent\textbf{Exercício~\theexerciciocount.}\space #1\par\vspace{0.5em}%
        \else
            % When suppressed, just print the content without the heading
            #1\par\vspace{0.5em}%
        \fi
}

% Macro para subexercício
\newcommand{\subexercicio}[1]{%
    \par\vspace{0.8em}% Espaçamento menor para subexercícios
    \refstepcounter{subexerciciocount}% Incrementa contador de subexercícios
    \noindent\textbf{\theexerciciocount.\thesubexerciciocount.} #1\par\vspace{0.3em}%
}

% Macro para opção
\newcommand{\option}[1]{%
    \par
    \refstepcounter{optioncount}%
    \noindent(\alph{optioncount}) #1%
}

% Título e informações do exame
\title{1ª Questão de aula do Módulo A10: Otimização}
\author{EPRALIMA - Escola Profissional Alto Lima}

\date{}

% Cabeçalho completo do teste dentro de uma caixa simples
\newcommand{\espacoAluno}{%
    \vspace{0.5cm}
    \fbox{%
        \parbox{\textwidth}{%
            \noindent\textbf{Nome do Aluno:} \underline{\hspace{7cm}} \textbf{Turma:} \underline{\hspace{1cm}}\\[0.5cm]
            \noindent\textbf{Assinatura do Professor:} \underline{\hspace{3cm}} \hfill \textbf{Nota:} \underline{\hspace{2cm}}\\[0.5cm]
            \noindent\textbf{Assinatura do Encarregado de Educação:} \underline{\hspace{3cm}}
        }%
    }
    \vspace{1cm}
}}{%
  \IfFileExists{../../../Teste_modelo/config/style.tex}{% Sistema de exercícios com contadores automáticos
\newcounter{exerciciocount}          % Contador principal dos exercícios
\newcounter{subexerciciocount}       % Contador dos subexercícios
\newcounter{optioncount}             % Contador das opções

% Control whether the macro prints the automatic "Exercício N." heading.
% Default: show the heading. Call \showexerciciotitlefalse to suppress.
\newif\ifshowexerciciotitle
\showexerciciotitletrue

% Macro para exercício principal
\newcommand{\exercicio}[1]{%
        \par\vspace{1.5em}% Espaçamento antes
        \refstepcounter{exerciciocount}% Incrementa contador principal
        \setcounter{subexerciciocount}{0}% Reseta contador de subexercícios
        \setcounter{optioncount}{0}% Reseta contador de opções
        % Only print the automatic heading if the flag is true
        \ifshowexerciciotitle
            \noindent\textbf{Exercício~\theexerciciocount.}\space #1\par\vspace{0.5em}%
        \else
            % When suppressed, just print the content without the heading
            #1\par\vspace{0.5em}%
        \fi
}

% Macro para subexercício
\newcommand{\subexercicio}[1]{%
    \par\vspace{0.8em}% Espaçamento menor para subexercícios
    \refstepcounter{subexerciciocount}% Incrementa contador de subexercícios
    \noindent\textbf{\theexerciciocount.\thesubexerciciocount.} #1\par\vspace{0.3em}%
}

% Macro para opção
\newcommand{\option}[1]{%
    \par
    \refstepcounter{optioncount}%
    \noindent(\alph{optioncount}) #1%
}

% Título e informações do exame
\title{1ª Questão de aula do Módulo A10: Otimização}
\author{EPRALIMA - Escola Profissional Alto Lima}

\date{}

% Cabeçalho completo do teste dentro de uma caixa simples
\newcommand{\espacoAluno}{%
    \vspace{0.5cm}
    \fbox{%
        \parbox{\textwidth}{%
            \noindent\textbf{Nome do Aluno:} \underline{\hspace{7cm}} \textbf{Turma:} \underline{\hspace{1cm}}\\[0.5cm]
            \noindent\textbf{Assinatura do Professor:} \underline{\hspace{3cm}} \hfill \textbf{Nota:} \underline{\hspace{2cm}}\\[0.5cm]
            \noindent\textbf{Assinatura do Encarregado de Educação:} \underline{\hspace{3cm}}
        }%
    }
    \vspace{1cm}
}}{%
    \IfFileExists{../../Teste_modelo/config/style.tex}{% Sistema de exercícios com contadores automáticos
\newcounter{exerciciocount}          % Contador principal dos exercícios
\newcounter{subexerciciocount}       % Contador dos subexercícios
\newcounter{optioncount}             % Contador das opções

% Control whether the macro prints the automatic "Exercício N." heading.
% Default: show the heading. Call \showexerciciotitlefalse to suppress.
\newif\ifshowexerciciotitle
\showexerciciotitletrue

% Macro para exercício principal
\newcommand{\exercicio}[1]{%
        \par\vspace{1.5em}% Espaçamento antes
        \refstepcounter{exerciciocount}% Incrementa contador principal
        \setcounter{subexerciciocount}{0}% Reseta contador de subexercícios
        \setcounter{optioncount}{0}% Reseta contador de opções
        % Only print the automatic heading if the flag is true
        \ifshowexerciciotitle
            \noindent\textbf{Exercício~\theexerciciocount.}\space #1\par\vspace{0.5em}%
        \else
            % When suppressed, just print the content without the heading
            #1\par\vspace{0.5em}%
        \fi
}

% Macro para subexercício
\newcommand{\subexercicio}[1]{%
    \par\vspace{0.8em}% Espaçamento menor para subexercícios
    \refstepcounter{subexerciciocount}% Incrementa contador de subexercícios
    \noindent\textbf{\theexerciciocount.\thesubexerciciocount.} #1\par\vspace{0.3em}%
}

% Macro para opção
\newcommand{\option}[1]{%
    \par
    \refstepcounter{optioncount}%
    \noindent(\alph{optioncount}) #1%
}

% Título e informações do exame
\title{1ª Questão de aula do Módulo A10: Otimização}
\author{EPRALIMA - Escola Profissional Alto Lima}

\date{}

% Cabeçalho completo do teste dentro de uma caixa simples
\newcommand{\espacoAluno}{%
    \vspace{0.5cm}
    \fbox{%
        \parbox{\textwidth}{%
            \noindent\textbf{Nome do Aluno:} \underline{\hspace{7cm}} \textbf{Turma:} \underline{\hspace{1cm}}\\[0.5cm]
            \noindent\textbf{Assinatura do Professor:} \underline{\hspace{3cm}} \hfill \textbf{Nota:} \underline{\hspace{2cm}}\\[0.5cm]
            \noindent\textbf{Assinatura do Encarregado de Educação:} \underline{\hspace{3cm}}
        }%
    }
    \vspace{1cm}
}}{%
      % style.tex not found - proceed without project macros
    }%
  }%
}

% Provide a robust fallback for macros that might be missing in style.tex
% This attempts to include the project style first (multiple relative paths),
% and only if none exist defines minimal counters and macros safely.
\IfFileExists{../../../../Teste_modelo/config/style.tex}{% Sistema de exercícios com contadores automáticos
\newcounter{exerciciocount}          % Contador principal dos exercícios
\newcounter{subexerciciocount}       % Contador dos subexercícios
\newcounter{optioncount}             % Contador das opções

% Control whether the macro prints the automatic "Exercício N." heading.
% Default: show the heading. Call \showexerciciotitlefalse to suppress.
\newif\ifshowexerciciotitle
\showexerciciotitletrue

% Macro para exercício principal
\newcommand{\exercicio}[1]{%
        \par\vspace{1.5em}% Espaçamento antes
        \refstepcounter{exerciciocount}% Incrementa contador principal
        \setcounter{subexerciciocount}{0}% Reseta contador de subexercícios
        \setcounter{optioncount}{0}% Reseta contador de opções
        % Only print the automatic heading if the flag is true
        \ifshowexerciciotitle
            \noindent\textbf{Exercício~\theexerciciocount.}\space #1\par\vspace{0.5em}%
        \else
            % When suppressed, just print the content without the heading
            #1\par\vspace{0.5em}%
        \fi
}

% Macro para subexercício
\newcommand{\subexercicio}[1]{%
    \par\vspace{0.8em}% Espaçamento menor para subexercícios
    \refstepcounter{subexerciciocount}% Incrementa contador de subexercícios
    \noindent\textbf{\theexerciciocount.\thesubexerciciocount.} #1\par\vspace{0.3em}%
}

% Macro para opção
\newcommand{\option}[1]{%
    \par
    \refstepcounter{optioncount}%
    \noindent(\alph{optioncount}) #1%
}

% Título e informações do exame
\title{1ª Questão de aula do Módulo A10: Otimização}
\author{EPRALIMA - Escola Profissional Alto Lima}

\date{}

% Cabeçalho completo do teste dentro de uma caixa simples
\newcommand{\espacoAluno}{%
    \vspace{0.5cm}
    \fbox{%
        \parbox{\textwidth}{%
            \noindent\textbf{Nome do Aluno:} \underline{\hspace{7cm}} \textbf{Turma:} \underline{\hspace{1cm}}\\[0.5cm]
            \noindent\textbf{Assinatura do Professor:} \underline{\hspace{3cm}} \hfill \textbf{Nota:} \underline{\hspace{2cm}}\\[0.5cm]
            \noindent\textbf{Assinatura do Encarregado de Educação:} \underline{\hspace{3cm}}
        }%
    }
    \vspace{1cm}
}}{%
  \IfFileExists{../../../Teste_modelo/config/style.tex}{% Sistema de exercícios com contadores automáticos
\newcounter{exerciciocount}          % Contador principal dos exercícios
\newcounter{subexerciciocount}       % Contador dos subexercícios
\newcounter{optioncount}             % Contador das opções

% Control whether the macro prints the automatic "Exercício N." heading.
% Default: show the heading. Call \showexerciciotitlefalse to suppress.
\newif\ifshowexerciciotitle
\showexerciciotitletrue

% Macro para exercício principal
\newcommand{\exercicio}[1]{%
        \par\vspace{1.5em}% Espaçamento antes
        \refstepcounter{exerciciocount}% Incrementa contador principal
        \setcounter{subexerciciocount}{0}% Reseta contador de subexercícios
        \setcounter{optioncount}{0}% Reseta contador de opções
        % Only print the automatic heading if the flag is true
        \ifshowexerciciotitle
            \noindent\textbf{Exercício~\theexerciciocount.}\space #1\par\vspace{0.5em}%
        \else
            % When suppressed, just print the content without the heading
            #1\par\vspace{0.5em}%
        \fi
}

% Macro para subexercício
\newcommand{\subexercicio}[1]{%
    \par\vspace{0.8em}% Espaçamento menor para subexercícios
    \refstepcounter{subexerciciocount}% Incrementa contador de subexercícios
    \noindent\textbf{\theexerciciocount.\thesubexerciciocount.} #1\par\vspace{0.3em}%
}

% Macro para opção
\newcommand{\option}[1]{%
    \par
    \refstepcounter{optioncount}%
    \noindent(\alph{optioncount}) #1%
}

% Título e informações do exame
\title{1ª Questão de aula do Módulo A10: Otimização}
\author{EPRALIMA - Escola Profissional Alto Lima}

\date{}

% Cabeçalho completo do teste dentro de uma caixa simples
\newcommand{\espacoAluno}{%
    \vspace{0.5cm}
    \fbox{%
        \parbox{\textwidth}{%
            \noindent\textbf{Nome do Aluno:} \underline{\hspace{7cm}} \textbf{Turma:} \underline{\hspace{1cm}}\\[0.5cm]
            \noindent\textbf{Assinatura do Professor:} \underline{\hspace{3cm}} \hfill \textbf{Nota:} \underline{\hspace{2cm}}\\[0.5cm]
            \noindent\textbf{Assinatura do Encarregado de Educação:} \underline{\hspace{3cm}}
        }%
    }
    \vspace{1cm}
}}{%
    \IfFileExists{../../Teste_modelo/config/style.tex}{% Sistema de exercícios com contadores automáticos
\newcounter{exerciciocount}          % Contador principal dos exercícios
\newcounter{subexerciciocount}       % Contador dos subexercícios
\newcounter{optioncount}             % Contador das opções

% Control whether the macro prints the automatic "Exercício N." heading.
% Default: show the heading. Call \showexerciciotitlefalse to suppress.
\newif\ifshowexerciciotitle
\showexerciciotitletrue

% Macro para exercício principal
\newcommand{\exercicio}[1]{%
        \par\vspace{1.5em}% Espaçamento antes
        \refstepcounter{exerciciocount}% Incrementa contador principal
        \setcounter{subexerciciocount}{0}% Reseta contador de subexercícios
        \setcounter{optioncount}{0}% Reseta contador de opções
        % Only print the automatic heading if the flag is true
        \ifshowexerciciotitle
            \noindent\textbf{Exercício~\theexerciciocount.}\space #1\par\vspace{0.5em}%
        \else
            % When suppressed, just print the content without the heading
            #1\par\vspace{0.5em}%
        \fi
}

% Macro para subexercício
\newcommand{\subexercicio}[1]{%
    \par\vspace{0.8em}% Espaçamento menor para subexercícios
    \refstepcounter{subexerciciocount}% Incrementa contador de subexercícios
    \noindent\textbf{\theexerciciocount.\thesubexerciciocount.} #1\par\vspace{0.3em}%
}

% Macro para opção
\newcommand{\option}[1]{%
    \par
    \refstepcounter{optioncount}%
    \noindent(\alph{optioncount}) #1%
}

% Título e informações do exame
\title{1ª Questão de aula do Módulo A10: Otimização}
\author{EPRALIMA - Escola Profissional Alto Lima}

\date{}

% Cabeçalho completo do teste dentro de uma caixa simples
\newcommand{\espacoAluno}{%
    \vspace{0.5cm}
    \fbox{%
        \parbox{\textwidth}{%
            \noindent\textbf{Nome do Aluno:} \underline{\hspace{7cm}} \textbf{Turma:} \underline{\hspace{1cm}}\\[0.5cm]
            \noindent\textbf{Assinatura do Professor:} \underline{\hspace{3cm}} \hfill \textbf{Nota:} \underline{\hspace{2cm}}\\[0.5cm]
            \noindent\textbf{Assinatura do Encarregado de Educação:} \underline{\hspace{3cm}}
        }%
    }
    \vspace{1cm}
}}{%
      % style.tex not found - define minimal counters/macros defensively
      \makeatletter
      \@ifundefined{exerciciocount}{\newcounter{exerciciocount}}{}
      \@ifundefined{subexerciciocount}{\newcounter{subexerciciocount}}{}
      \@ifundefined{optioncount}{\newcounter{optioncount}}{}

      \newcommand{\exercicio}[1]{%
        \par\vspace{1.5em}%
        \refstepcounter{exerciciocount}%
        \setcounter{subexerciciocount}{0}%
        \setcounter{optioncount}{0}%
        \noindent\textbf{Exercício~\theexerciciocount.} #1\par\vspace{0.5em}%
      }

      \newcommand{\subexercicio}[1]{%
        \par\vspace{0.8em}%
        \refstepcounter{subexerciciocount}%
        \noindent\textbf{\theexerciciocount.\thesubexerciciocount.} #1\par\vspace{0.3em}%
      }

      \newcommand{\exercicioDesenvolvimento}[1]{\par\noindent #1\par}
      \newcommand{\option}[1]{%
        \par\refstepcounter{optioncount}%
        \noindent(\alph{optioncount}) #1%
      }
      \makeatother
    }%
  }%
}

% ========== IP-BASED TEST SYSTEM MACROS (v3.5) ==========
% Support for modular exercise inclusion with numbered headings
% Provide a boolean flag to control whether the automatic heading is shown
\makeatletter
\@ifundefined{showexerciciotitletrue}{%
    \newif\ifshowexerciciotitle
    \showexerciciotitletrue
}{}
\makeatother

% Override \exercicio to respect the \ifshowexerciciotitle flag
% When false, it prints only the content without automatic heading
\renewcommand{\exercicio}[1]{%
    \ifshowexerciciotitle
        \par\vspace{1.5em}%
        \refstepcounter{exerciciocount}%
        \setcounter{subexerciciocount}{0}%
        \setcounter{optioncount}{0}%
        \noindent\textbf{Exercício~\theexerciciocount.} #1\par\vspace{0.5em}%
    \else
        #1\par
    \fi
}

\pagestyle{fancy}
\fancyhf{}
\lhead{MÓDULO P4 - Funções}
\rhead{Generalidades acerca de Funções}
\cfoot{\thepage}

\title{}
\author{}
\date{}

\begin{document}
\maketitle

\section*{Generalidades acerca de Funções}

\textit{
Conceitos fundamentais sobre funções reais de variável real, incluindo definição, domínio, contradomínio, imagem e representações.
}

\vspace{1em}

\subsection*{Tipos de Exercícios}
\begin{itemize}
  \item \textbf{Afirmações Causais (preço/despesa)}
  \item \textbf{Afirmações Causais sobre Funções} --- Exercícios que envolvem análise de afirmações causais relacionadas a funções, como relações entre variáveis dependentes e independentes.
  \item \textbf{Afirmacoes Causais Despesa Preco}
  \item \textbf{Afirmacoes Causais Monotonicidade}
  \item \textbf{Afirmacoes Causais Relacoes}
  \item \textbf{Afirmacoes Contexto}
  \item \textbf{Afirmacoes Contextuais}
  \item \textbf{Afirmacoes Relacionais}
  \item \textbf{Afirmações Verdadeiro/Falso} --- Exercícios onde os alunos avaliam afirmações (V/F) sobre relações entre preço, quantidade e despesa.
  \item \textbf{Análise de Afirmações} --- Exercícios nos quais o aluno analisa afirmações sobre relações entre preço e despesa total e decide se são verdadeiras ou falsas.
  \item \textbf{Interpretacao Contexto}
  \item \textbf{Juízo Causal sobre Monotonicidade}
  \item \textbf{Reconhecimento Variaveis}
  \item \textbf{Relações Causais (preço/despesa)}
  \item \textbf{Relacoes Causa Efeito}
  \item \textbf{Test Logging}
  \item \textbf{Test Payload File}
  \item \textbf{Test Wrapper Integration}
  \item \textbf{Tipo Novo}
\end{itemize}

\vspace{1em}

% Exercício 1: MAT_P4FUNCOE_1GX_ACA_001.tex
% Exercise ID: MAT_P4FUNCOE_1GX_ACA_001
% Created: 2025-11-28
% Difficulty: 3/5

\exercicio{Considere as seguintes afirmações sobre o preço p>0, a quantidade demandada q(p)\ge 0 e a despesa total D(p)=p\,q(p). Para cada afirmação: (i) classifique-a como "sempre verdadeira", "às vezes verdadeira" ou "falsa"; (ii) justifique brevemente a sua resposta (prova curta ou contraexemplo).

\begin{enumerate}[a)]
\item Quando o preço aumenta, a despesa total D(p) aumenta.
\item Se q(p) é decrescente em p, então D(p) é decrescente.
\item Se q(p)=k/p (k>0), então D(p) é constante e não depende de p.
\item Se p_1<p_2 e D(p_2)>D(p_1), então necessariamente q(p_2)>q(p_1).
\end{enumerate}}
\FloatBarrier

% Exercício 2: MAT_P4FUNCOE_1GX_ACX_001.tex
% Exercise ID: MAT_P4FUNCOE_1GX_ACX_001
% Created: 2025-11-28
% Difficulty: 3/5

\exercicio{'Considere as seguintes afirmações sobre preço (p)}
\FloatBarrier

% Exercício 3: MAT_P4FUNCOE_1GX_ACX_002.tex
% Exercise ID: MAT_P4FUNCOE_1GX_ACX_002
% Created: 2025-11-28
% Difficulty: 3/5

\exercicio{'Considere as seguintes afirmações sobre preço (p)}
\FloatBarrier

% Exercício 4: MAT_P4FUNCOE_1GFUN_AFC_001.tex
% Exercise ID: MAT_P4FUNCOE_1GFUN_AFC_001
% Module: MÓDULO P4 - Funções | Concept: Generalidades acerca de Funções
% Difficulty: 3/5 (Médio) | Format: desenvolvimento
% Tags: afirmacoes_causais, funcoes, relacoes_causais, analise_matematica
% Author: Professor | Date: 2025-11-28
% Status: active

\exercicio{Considere P (preço) e D (despesa) relacionadas por uma função D = f(P). Para cada uma das afirmações seguintes: (i) indique se é verdadeira ou falsa; (ii) justifique a sua resposta com argumentos matemáticos; (iii) quando possível, dê um exemplo explícito de função f que torne a afirmação verdadeira e outro que a torne falsa.
\par
\subexercicio{(A) "Quando o preço aumenta, a despesa aumenta."}
\subexercicio{(B) "Se o preço dobra, então a despesa dobra."}
\subexercicio{(C) "Se f for estritamente crescente, então qualquer aumento do preço implica um aumento da despesa."}
\subexercicio{(D) "Se f for constante, então a despesa não varia quando o preço varia."}}
\FloatBarrier

% Exercício 5: MAT_P4FUNCOE_1GX_ACD_001.tex
% Exercise ID: MAT_P4FUNCOE_1GX_ACD_001
% Created: 2025-11-27
% Difficulty: 3/5

\exercicio{Classifique V/F e justifique: (a) Quando o preço de um bem aumenta}
\FloatBarrier

% Exercício 6: MAT_P4FUNCOE_1GX_ACM_001.tex
% Exercise ID: MAT_P4FUNCOE_1GX_ACM_001
% Created: 2025-11-27
% Difficulty: 3/5

\exercicio{Classificação de afirmações: para cada afirmação indique VERDADEIRA/FALSA}
\FloatBarrier

% Exercício 7: MAT_P4FUNCOE_1GX_ACM_002.tex
% Exercise ID: MAT_P4FUNCOE_1GX_ACM_002
% Created: 2025-11-27
% Difficulty: 3/5

\exercicio{Teste de logging de inputs.}
\FloatBarrier

% Exercício 8: MAT_P4FUNCOE_1GX_ACR_001.tex
% Exercise ID: MAT_P4FUNCOE_1GX_ACR_001
% Created: 2025-11-27
% Difficulty: 3/5

\exercicio{Classifique as seguintes afirmações como VERDADEIRA ou FALSA; justifique rigorosamente e}
\FloatBarrier

% Exercício 9: MAT_P4FUNCOE_1GX_ACX_001.tex
% Exercise ID: MAT_P4FUNCOE_1GX_ACX_001
% Created: 2025-11-29
% Difficulty: 2/5

\exercicio{Teste de staging via wrapper}
\FloatBarrier

% Exercício 10: MAT_P4FUNCOE_1GX_ACX_001.tex
% Exercise ID: MAT_P4FUNCOE_1GX_ACX_001
% Created: 2025-11-27
% Difficulty: 2/5

\exercicio{Considere que}
\FloatBarrier

% Exercício 11: MAT_P4FUNCOE_1GEN_AR_001.tex
% Exercise ID: MAT_P4FUNCOE_1GEN_AR_001
% Created: 2025-11-27
% Difficulty: 2/5

\exercicio{Classifique cada afirmação como Verdadeira (V) ou Falsa (F) e justifique brevemente.}

\subexercicio{(a) Quando o preço de um bem aumenta, a despesa (gasto total) com esse bem aumenta.}

\subexercicio{(b) Se a quantidade consumida desse bem se mantém fixa, então um aumento do preço implica um aumento da despesa.}

\subexercicio{(c) A relação que associa a cada preço $p$ a despesa total $D(p)=p\cdot q(p)$ é, em toda situação, uma função (ou seja, a cada preço corresponde um único valor de despesa).}

\subexercicio{(d) Se $D(p)=p\cdot q(p)$ é uma função monótona crescente em $p$ (para $p>0$), então obrigatoriamente $q(p)$ é constante e positiva.}

\subexercicio{(e) Se $D(p)=p\cdot q(p)$ é injetora para $p>0$, então $q(p)$ é também uma função injetora.}
\FloatBarrier

% Exercício 12: MAT_P4FUNCOE_1GX_ARX_001.tex
% Exercise ID: MAT_P4FUNCOE_1GX_ARX_001
% Created: 2025-11-27
% Difficulty: 2/5

\exercicio{Teste de afirmação: 1+1=2}
\FloatBarrier

% Exercício 13: MAT_P4FUNCOE_1GX_ARX_002.tex
% Exercise ID: MAT_P4FUNCOE_1GX_ARX_002
% Created: 2025-11-27
% Difficulty: 2/5

\exercicio{Classifique cada uma das afirmações seguintes como Verdadeira (V) ou Falsa (F). Em cada caso, justifique brevemente a sua resposta.}

\subexercicio{(a) "Quando o preço de um produto aumenta, a despesa total com esse produto aumenta."}

\subexercicio{(b) "Se o preço de um produto aumenta e a quantidade vendida permanece exactamente a mesma, então a despesa total aumenta."}

\subexercicio{(c) "Se uma relação associa a cada preço uma despesa (ou seja, para cada preço existe exactamente uma despesa associada), então essa relação é uma função do conjunto dos preços no conjunto das despesas."}

\subexercicio{(d) "Se $f: \mathbb{R}\to\mathbb{R}$ é crescente, então para quaisquer $x_1<x_2$ temos $f(x_1)\le f(x_2)$."}

\subexercicio{(e) "Uma função que é injectiva garante que valores iguais da imagem correspondem a argumentos iguais."}
\FloatBarrier

% Exercício 14: MAT_P4FUNCOE_1GX_ARX_004.tex
% Exercise ID: MAT_P4FUNCOE_1GX_ARX_004
% Created: 2025-11-27
% Difficulty: 2/5

\exercicio{Quero que cries um exercício no módulo P4_funcoes na generalidades de funcoes (procurar nomenclatura) e adiciones um novo tipo de exercício (escolhe tu a nomenclatura do tipo de exercício!)}
\FloatBarrier

% Exercício 15: MAT_P4FUNCOE_1GX_ARX_005.tex
% Exercise ID: MAT_P4FUNCOE_1GX_ARX_005
% Created: 2025-11-27
% Difficulty: 2/5

\exercicio{Quero que cries um exercício no módulo P4_funcoes na generalidades de funcoes onde os alunos analisem afirmacoes tipo 'quando o preço aumenta a despesa vai aumentar'}
\FloatBarrier

% Exercício 16: MAT_P4FUNCOE_1GX_AVF_001.tex
% meta:
% id: MAT_P4FUNCOE_1GX_AVF_001
% title: "Afirmações sobre preço e despesa"
% difficulty: 2
% tags: ["afirmacoes","preco","despesa","verdadeiro_falso"]
% author: "OpenAI"
% has_parts: true
% parts_count: 3

\section{Afirmações sobre preço e despesa}

\exercicio{
Considere a seguinte situação:

Numa loja, o preço de um produto é representado por $p$ (em euros) e a despesa total dos clientes é dada por $D(p)$.

Para cada uma das afirmações seguintes, indique se é verdadeira ou falsa, justificando a sua resposta:
\begin{enumerate}
    \item Quando o preço aumenta, a despesa total $D(p)$ aumenta sempre.
    \item Se o preço for zero, a despesa total $D(0)$ é necessariamente zero.
    \item Se a despesa total $D(p)$ diminui quando o preço aumenta, então os clientes estão a comprar menos produtos.
\end{enumerate}
}
\FloatBarrier

% Exercício 17: MAT_P4FUNCOE_1GX_AAF_001.tex
% meta:
% id: MAT_P4FUNCOE_1GX_AAF_001
% title: "Análise de afirmações sobre preço e despesa"
% discipline: "matematica"
% module: "P4_funcoes"
% concept: "1-generalidades_funcoes"
% tipo: "analise_afirmacoes"
% difficulty: 2
% tags: "analise,afirmacoes,preco,despesa"
% author: "OpenAI"
% version: 1

\section{Análise de afirmações}

\exercicio{
Considere a seguinte situação:

"Numa loja, o preço de um produto é representado por $p$ (em euros) e a despesa total dos clientes é dada por $D(p)$.")

Para cada uma das afirmações seguintes, indique se é verdadeira ou falsa, justificando a sua resposta:
\begin{enumerate}
    \item Quando o preço aumenta, a despesa total $D(p)$ aumenta sempre.
    \item Se o preço for zero, a despesa total $D(0)$ é necessariamente zero.
    \item Se a despesa total $D(p)$ diminui quando o preço aumenta, então os clientes estão a comprar menos produtos.
\end{enumerate}
}
\FloatBarrier

% Exercício 18: MAT_P4FUNCOE_1GX_ICX_001.tex
% Exercise ID: MAT_P4FUNCOE_1GX_ICX_001
% Created: 2025-11-26
% Difficulty: 2/5

\exercicio{Considere a seguinte situação: Numa loja, o preço de um produto é representado por x (em euros) e a despesa total dos clientes é dada por D(x). Indique se as seguintes afirmações são verdadeiras ou falsas, justificando a sua resposta.}
\FloatBarrier

% Exercício 19: MAT_P4FUNCOE_1GX_ICX_002.tex
% Exercise ID: MAT_P4FUNCOE_1GX_ICX_002
% Created: 2025-11-26
% Difficulty: 2/5

\exercicio{AAAAAAAAAAAAAAAAAAAAAAAAAAAAAAAAAAAAAAAAAAAAAAAAAAAAAAAAAAAAAAAAAAAAAAAAAAAAAAAAAAAAAAAAAAAAAAAAAAAAAAAAAAAAAAAAAAAAAAAAAAAAAAAAAAAAAAAAAAAAAAAAAAAAAAAAAAAAAAAAAAAAAAAAAAAAAAAAAAAAAAAAAAAAAAAAAAAAAAAAAAAAAAAAAAAAAAAAAAAAAAAAAAAAAAAAAAAAAAAAAAAAAAAAAAAAAAAAAAAAAAAAAAAAAAAAAAAAAAAAAAAAAAAAAAAAAAAAAAAAAAAAAAAAAAAAAAAAAAAAAAAAAAAAAAAAAAAAAAAAAAAAAAAAAAAAAAAAAAAAAAAAAAAAAAAAAAAAAAAAAAAAAAAAAAAAAAAAAAAAAAAAAAAAAAAAAAAAAAAAAAAAAAAAAAAAAAAAAAAAAAAAAAAAAAAAAAAAAAAAAAAAAAAAAAAAAAAAAAAAAAAAAAAAAAAAAAAAAAAAAAAAAAAAAAAAAAAAAAAAAAAAAAAAAAAAAAAAAAAAAAAAAAAAAAAAAAAAAAAAAAAAAAAAAAAAAAAAAAAAAAAAAAAAAAAAAAAAAAAAAAAAAAAAAAAAAAAAAAAAAAAAAAAAAAAAAAAAAAAAAAAAAAAAAAAAAAAAAAAAAAAAAAAAAAAAAAAAAAAAAAAAAAAAAAAAAAAAAAAAAAAAAAAAAAAAAAAAAAAAAAAAAAAAAAAAAAAAAAAAAAAAAAAAAAAAAAAAAAAAAAAAAAAAAAAAAAAAAAAAAAAAAAAAAAAAAAAAAAAAAAAAAAAAAAAAAAAAAAAAAAAAAAAAAAAAAAAAAAAAAAAAAAAAAAAAAAAAAAAAAAAAAAAAAAAAAAAAAAAAAAAAAAAAAAAAAAAAAAAAAAAAAAAAAAAAAAAAAAAAAAAAAAAAAAAAAAAAAAAAAAAAAAAAAAAAAAAAAAAAAAAAAAAAAAAAAAAAAAAAAAAAAAAAAAAAAAAAAAAAAAAAAAAAAAAAAAAAAAAAAAAAAAAAAAAAAAAAAAAAAAAAAAAAAAAAAAAAAAAAAAAAAAAAAAAAAAAAAAAAAAAAAAAAAAAAAAAAAAAAAAAAAAAAAAAAAAAAAAAAAAAAAAAAAAAAAAAAAAAAAAAAAAAAAAAAAAAAAAAAAAAAAAAAAAAAAAAAAAAAAAAAAAAAAAAAAAAAAAAAAAAAAAAAAAAAAAAAAAAAAAAAAAAAAAAAAAAAAAAAAAAAAAAAAAAAAAAAAAAAAAAAAAAAAAAAAAAAAAAAAAAAAAAAAAAAAAAAAAAAAAAAAAAAAAAAAAAAAAAAAAAAAAAAAAAAAAAAAAAAAAAAAAAAAAAAAAAAAAAAAAAAAAAAAAAAAAAAAAAAAAAAAAAAAAAAAAAAAAAAAAAAAAAAAAAAAAAAAAAAAAAAAAAAAAAAAAAAAAAAAAAAAAAAAAAAAAAAAAAAAAAAAAAAAAAAAAAAAAAAAAAAAAAAAAAAAAAAAAAAAAAAAAAAAAAAAAAAAAAAAAAAAAAAAAAAAAAAAAAAAAAAAAAAAAAAAAAAAAAAAAAAAAAAAAAAAAAAAAAAAAAAAAAAAAAAAAAAAAAAAAAAAAAAAAAAAAAAAAAAAAAAAAAAAAAAAAAAAAAAAAAAAAAAAAAAAAAAAAAAAAAAAAAAAAAAAAAAAAAAAAAAAAAAAAAAAAAAAAAAAAAAAAAAAAAAAAAAAAAAAAAAAAAAAAAAAAAAAAAAAAAAAAAAAAAAAAAAAAAAAAAAAAAAAAAAAAAAAAAAAAAAAAAAAAAAAAAAAAAAAAAAAAAAAAAAAAAAAAAAAAAAAAAAAAAAAAAAAAAAAAAAAAAAAAAAAAAAAAAAAAAAAAAAAAAAAAAAAAAAAAAAAAAAAAAAAAAAAAAAAAAAAAAAAAAAAAAAAAAAAAAAAAAAAAAAAAAAAAAAAAAAAAAAAAAAAAAAAAAAAAAAAAAAAAAAAAAAAAAAAAAAAAAAAAAAAAAAAAAA}
\FloatBarrier

% Exercício 20: MAT_P4FUNCOE_1GX_ICX_003.tex
% Exercise ID: MAT_P4FUNCOE_1GX_ICX_003
% Created: 2025-11-26
% Difficulty: 2/5

\exercicio{Questão com caracteres especiais: % $ & { } [ ] ~ # _ ^ \}
\FloatBarrier

% Exercício 21: MAT_P4FUNCOE_1GX_ICX_004.tex
% Exercise ID: MAT_P4FUNCOE_1GX_ICX_004
% Created: 2025-11-26
% Difficulty: 2/5

\exercicio{Considere a seguinte situação: Numa loja, o preço de um produto é representado por x (em euros) e a despesa total dos clientes é dada por D(x). Indique se as seguintes afirmações são verdadeiras ou falsas, justificando a sua resposta.\par 1. Quando o preço aumenta, a despesa total D(x) aumenta sempre.\par 2. Se o preço for zero, a despesa total D(0) é necessariamente zero.\par 3. Se a despesa total D(x) diminui quando o preço aumenta, então os clientes estão a comprar menos produtos.}
\FloatBarrier

% Exercício 22: MAT_P4FUNCOE_1GX_ICX_005.tex
% Exercise ID: MAT_P4FUNCOE_1GX_ICX_005
% Created: 2025-11-26
% Difficulty: 2/5

\exercicio{Questão com caracteres especiais: % $ & { } [ ] ~ # _ ^ \}
\FloatBarrier

% Exercício 23: MAT_P4FUNCOE_1GX_ICX_006.tex
% Exercise ID: MAT_P4FUNCOE_1GX_ICX_006
% Created: 2025-11-26
% Difficulty: 2/5

\exercicio{A*3000}
\FloatBarrier

% Exercício 24: MAT_P4FUNCOE_1GX_ICX_007.tex
% Exercise ID: MAT_P4FUNCOE_1GX_ICX_007
% Created: 2025-11-26
% Difficulty: 2/5

\exercicio{Quando o preço aumenta, a despesa aumenta? Verdadeiro ou falso? Justifique.}
\FloatBarrier

% Exercício 25: MAT_P4FUNCOE_1GX_ICX_008.tex
% Exercise ID: MAT_P4FUNCOE_1GX_ICX_008
% Created: 2025-11-27
% Difficulty: 2/5

\exercicio{Considere a seguinte situação: Numa loja, o preço de um produto é representado por x (em euros) e a despesa total dos clientes é dada por D(x). Indique se as seguintes afirmações são verdadeiras ou falsas, justificando a sua resposta.}
\FloatBarrier

% Exercício 26: MAT_P4FUNCOE_1GX_ICX_009.tex
% Exercise ID: MAT_P4FUNCOE_1GX_ICX_009
% Created: 2025-11-27
% Difficulty: 2/5

\exercicio{AAAAAAAAAAAAAAAAAAAAAAAAAAAAAAAAAAAAAAAAAAAAAAAAAAAAAAAAAAAAAAAAAAAAAAAAAAAAAAAAAAAAAAAAAAAAAAAAAAAAAAAAAAAAAAAAAAAAAAAAAAAAAAAAAAAAAAAAAAAAAAAAAAAAAAAAAAAAAAAAAAAAAAAAAAAAAAAAAAAAAAAAAAAAAAAAAAAAAAAAAAAAAAAAAAAAAAAAAAAAAAAAAAAAAAAAAAAAAAAAAAAAAAAAAAAAAAAAAAAAAAAAAAAAAAAAAAAAAAAAAAAAAAAAAAAAAAAAAAAAAAAAAAAAAAAAAAAAAAAAAAAAAAAAAAAAAAAAAAAAAAAAAAAAAAAAAAAAAAAAAAAAAAAAAAAAAAAAAAAAAAAAAAAAAAAAAAAAAAAAAAAAAAAAAAAAAAAAAAAAAAAAAAAAAAAAAAAAAAAAAAAAAAAAAAAAAAAAAAAAAAAAAAAAAAAAAAAAAAAAAAAAAAAAAAAAAAAAAAAAAAAAAAAAAAAAAAAAAAAAAAAAAAAAAAAAAAAAAAAAAAAAAAAAAAAAAAAAAAAAAAAAAAAAAAAAAAAAAAAAAAAAAAAAAAAAAAAAAAAAAAAAAAAAAAAAAAAAAAAAAAAAAAAAAAAAAAAAAAAAAAAAAAAAAAAAAAAAAAAAAAAAAAAAAAAAAAAAAAAAAAAAAAAAAAAAAAAAAAAAAAAAAAAAAAAAAAAAAAAAAAAAAAAAAAAAAAAAAAAAAAAAAAAAAAAAAAAAAAAAAAAAAAAAAAAAAAAAAAAAAAAAAAAAAAAAAAAAAAAAAAAAAAAAAAAAAAAAAAAAAAAAAAAAAAAAAAAAAAAAAAAAAAAAAAAAAAAAAAAAAAAAAAAAAAAAAAAAAAAAAAAAAAAAAAAAAAAAAAAAAAAAAAAAAAAAAAAAAAAAAAAAAAAAAAAAAAAAAAAAAAAAAAAAAAAAAAAAAAAAAAAAAAAAAAAAAAAAAAAAAAAAAAAAAAAAAAAAAAAAAAAAAAAAAAAAAAAAAAAAAAAAAAAAAAAAAAAAAAAAAAAAAAAAAAAAAAAAAAAAAAAAAAAAAAAAAAAAAAAAAAAAAAAAAAAAAAAAAAAAAAAAAAAAAAAAAAAAAAAAAAAAAAAAAAAAAAAAAAAAAAAAAAAAAAAAAAAAAAAAAAAAAAAAAAAAAAAAAAAAAAAAAAAAAAAAAAAAAAAAAAAAAAAAAAAAAAAAAAAAAAAAAAAAAAAAAAAAAAAAAAAAAAAAAAAAAAAAAAAAAAAAAAAAAAAAAAAAAAAAAAAAAAAAAAAAAAAAAAAAAAAAAAAAAAAAAAAAAAAAAAAAAAAAAAAAAAAAAAAAAAAAAAAAAAAAAAAAAAAAAAAAAAAAAAAAAAAAAAAAAAAAAAAAAAAAAAAAAAAAAAAAAAAAAAAAAAAAAAAAAAAAAAAAAAAAAAAAAAAAAAAAAAAAAAAAAAAAAAAAAAAAAAAAAAAAAAAAAAAAAAAAAAAAAAAAAAAAAAAAAAAAAAAAAAAAAAAAAAAAAAAAAAAAAAAAAAAAAAAAAAAAAAAAAAAAAAAAAAAAAAAAAAAAAAAAAAAAAAAAAAAAAAAAAAAAAAAAAAAAAAAAAAAAAAAAAAAAAAAAAAAAAAAAAAAAAAAAAAAAAAAAAAAAAAAAAAAAAAAAAAAAAAAAAAAAAAAAAAAAAAAAAAAAAAAAAAAAAAAAAAAAAAAAAAAAAAAAAAAAAAAAAAAAAAAAAAAAAAAAAAAAAAAAAAAAAAAAAAAAAAAAAAAAAAAAAAAAAAAAAAAAAAAAAAAAAAAAAAAAAAAAAAAAAAAAAAAAAAAAAAAAAAAAAAAAAAAAAAAAAAAAAAAAAAAAAAAAAAAAAAAAAAAAAAAAAAAAAAAAAAAAAAAAAAAAAAAAAAAAAAAAAAAAAAAAAAAAAAAAAAAAAAAAAAAAAAAAAAAAAAAAAAAAAAAAAAAAAAAAAAAAAAAAAAAAAAAAAAAAAAAAAAAAAAAAAAAAAAAA}
\FloatBarrier

% Exercício 27: MAT_P4FUNCOE_1GX_ICX_010.tex
% Exercise ID: MAT_P4FUNCOE_1GX_ICX_010
% Created: 2025-11-27
% Difficulty: 2/5

\exercicio{Questão com caracteres especiais: % $ & { } [ ] ~ # _ ^ \}
\FloatBarrier

% Exercício 28: MAT_P4FUNCOE_1GX_ICX_011.tex
% Exercise ID: MAT_P4FUNCOE_1GX_ICX_011
% Created: 2025-11-27
% Difficulty: 2/5

\exercicio{Considere a seguinte situação: Numa loja, o preço de um produto é representado por x (em euros) e a despesa total dos clientes é dada por D(x). Indique se as seguintes afirmações são verdadeiras ou falsas, justificando a sua resposta.}
\FloatBarrier

% Exercício 29: MAT_P4FUNCOE_1GX_ICX_012.tex
% Exercise ID: MAT_P4FUNCOE_1GX_ICX_012
% Created: 2025-11-27
% Difficulty: 2/5

\exercicio{AAAAAAAAAAAAAAAAAAAAAAAAAAAAAAAAAAAAAAAAAAAAAAAAAAAAAAAAAAAAAAAAAAAAAAAAAAAAAAAAAAAAAAAAAAAAAAAAAAAAAAAAAAAAAAAAAAAAAAAAAAAAAAAAAAAAAAAAAAAAAAAAAAAAAAAAAAAAAAAAAAAAAAAAAAAAAAAAAAAAAAAAAAAAAAAAAAAAAAAAAAAAAAAAAAAAAAAAAAAAAAAAAAAAAAAAAAAAAAAAAAAAAAAAAAAAAAAAAAAAAAAAAAAAAAAAAAAAAAAAAAAAAAAAAAAAAAAAAAAAAAAAAAAAAAAAAAAAAAAAAAAAAAAAAAAAAAAAAAAAAAAAAAAAAAAAAAAAAAAAAAAAAAAAAAAAAAAAAAAAAAAAAAAAAAAAAAAAAAAAAAAAAAAAAAAAAAAAAAAAAAAAAAAAAAAAAAAAAAAAAAAAAAAAAAAAAAAAAAAAAAAAAAAAAAAAAAAAAAAAAAAAAAAAAAAAAAAAAAAAAAAAAAAAAAAAAAAAAAAAAAAAAAAAAAAAAAAAAAAAAAAAAAAAAAAAAAAAAAAAAAAAAAAAAAAAAAAAAAAAAAAAAAAAAAAAAAAAAAAAAAAAAAAAAAAAAAAAAAAAAAAAAAAAAAAAAAAAAAAAAAAAAAAAAAAAAAAAAAAAAAAAAAAAAAAAAAAAAAAAAAAAAAAAAAAAAAAAAAAAAAAAAAAAAAAAAAAAAAAAAAAAAAAAAAAAAAAAAAAAAAAAAAAAAAAAAAAAAAAAAAAAAAAAAAAAAAAAAAAAAAAAAAAAAAAAAAAAAAAAAAAAAAAAAAAAAAAAAAAAAAAAAAAAAAAAAAAAAAAAAAAAAAAAAAAAAAAAAAAAAAAAAAAAAAAAAAAAAAAAAAAAAAAAAAAAAAAAAAAAAAAAAAAAAAAAAAAAAAAAAAAAAAAAAAAAAAAAAAAAAAAAAAAAAAAAAAAAAAAAAAAAAAAAAAAAAAAAAAAAAAAAAAAAAAAAAAAAAAAAAAAAAAAAAAAAAAAAAAAAAAAAAAAAAAAAAAAAAAAAAAAAAAAAAAAAAAAAAAAAAAAAAAAAAAAAAAAAAAAAAAAAAAAAAAAAAAAAAAAAAAAAAAAAAAAAAAAAAAAAAAAAAAAAAAAAAAAAAAAAAAAAAAAAAAAAAAAAAAAAAAAAAAAAAAAAAAAAAAAAAAAAAAAAAAAAAAAAAAAAAAAAAAAAAAAAAAAAAAAAAAAAAAAAAAAAAAAAAAAAAAAAAAAAAAAAAAAAAAAAAAAAAAAAAAAAAAAAAAAAAAAAAAAAAAAAAAAAAAAAAAAAAAAAAAAAAAAAAAAAAAAAAAAAAAAAAAAAAAAAAAAAAAAAAAAAAAAAAAAAAAAAAAAAAAAAAAAAAAAAAAAAAAAAAAAAAAAAAAAAAAAAAAAAAAAAAAAAAAAAAAAAAAAAAAAAAAAAAAAAAAAAAAAAAAAAAAAAAAAAAAAAAAAAAAAAAAAAAAAAAAAAAAAAAAAAAAAAAAAAAAAAAAAAAAAAAAAAAAAAAAAAAAAAAAAAAAAAAAAAAAAAAAAAAAAAAAAAAAAAAAAAAAAAAAAAAAAAAAAAAAAAAAAAAAAAAAAAAAAAAAAAAAAAAAAAAAAAAAAAAAAAAAAAAAAAAAAAAAAAAAAAAAAAAAAAAAAAAAAAAAAAAAAAAAAAAAAAAAAAAAAAAAAAAAAAAAAAAAAAAAAAAAAAAAAAAAAAAAAAAAAAAAAAAAAAAAAAAAAAAAAAAAAAAAAAAAAAAAAAAAAAAAAAAAAAAAAAAAAAAAAAAAAAAAAAAAAAAAAAAAAAAAAAAAAAAAAAAAAAAAAAAAAAAAAAAAAAAAAAAAAAAAAAAAAAAAAAAAAAAAAAAAAAAAAAAAAAAAAAAAAAAAAAAAAAAAAAAAAAAAAAAAAAAAAAAAAAAAAAAAAAAAAAAAAAAAAAAAAAAAAAAAAAAAAAAAAAAAAAAAAAAAAAAAAAAAAAAAAAAAAAAAAAAAAAAAAAAAAA}
\FloatBarrier

% Exercício 30: MAT_P4FUNCOE_1GX_ICX_013.tex
% Exercise ID: MAT_P4FUNCOE_1GX_ICX_013
% Created: 2025-11-27
% Difficulty: 2/5

\exercicio{Questão com caracteres especiais: % $ & { } [ ] ~ # _ ^ \}
\FloatBarrier

% Exercício 31: MAT_P4FUNCOE_1GX_ICX_014.tex
% Exercise ID: MAT_P4FUNCOE_1GX_ICX_014
% Created: 2025-11-27
% Difficulty: 2/5

\exercicio{Considere a seguinte situação: Numa loja, o preço de um produto é representado por x (em euros) e a despesa total dos clientes é dada por D(x). Indique se as seguintes afirmações são verdadeiras ou falsas, justificando a sua resposta.}
\FloatBarrier

% Exercício 32: MAT_P4FUNCOE_1GX_ICX_015.tex
% Exercise ID: MAT_P4FUNCOE_1GX_ICX_015
% Created: 2025-11-27
% Difficulty: 2/5

\exercicio{AAAAAAAAAAAAAAAAAAAAAAAAAAAAAAAAAAAAAAAAAAAAAAAAAAAAAAAAAAAAAAAAAAAAAAAAAAAAAAAAAAAAAAAAAAAAAAAAAAAAAAAAAAAAAAAAAAAAAAAAAAAAAAAAAAAAAAAAAAAAAAAAAAAAAAAAAAAAAAAAAAAAAAAAAAAAAAAAAAAAAAAAAAAAAAAAAAAAAAAAAAAAAAAAAAAAAAAAAAAAAAAAAAAAAAAAAAAAAAAAAAAAAAAAAAAAAAAAAAAAAAAAAAAAAAAAAAAAAAAAAAAAAAAAAAAAAAAAAAAAAAAAAAAAAAAAAAAAAAAAAAAAAAAAAAAAAAAAAAAAAAAAAAAAAAAAAAAAAAAAAAAAAAAAAAAAAAAAAAAAAAAAAAAAAAAAAAAAAAAAAAAAAAAAAAAAAAAAAAAAAAAAAAAAAAAAAAAAAAAAAAAAAAAAAAAAAAAAAAAAAAAAAAAAAAAAAAAAAAAAAAAAAAAAAAAAAAAAAAAAAAAAAAAAAAAAAAAAAAAAAAAAAAAAAAAAAAAAAAAAAAAAAAAAAAAAAAAAAAAAAAAAAAAAAAAAAAAAAAAAAAAAAAAAAAAAAAAAAAAAAAAAAAAAAAAAAAAAAAAAAAAAAAAAAAAAAAAAAAAAAAAAAAAAAAAAAAAAAAAAAAAAAAAAAAAAAAAAAAAAAAAAAAAAAAAAAAAAAAAAAAAAAAAAAAAAAAAAAAAAAAAAAAAAAAAAAAAAAAAAAAAAAAAAAAAAAAAAAAAAAAAAAAAAAAAAAAAAAAAAAAAAAAAAAAAAAAAAAAAAAAAAAAAAAAAAAAAAAAAAAAAAAAAAAAAAAAAAAAAAAAAAAAAAAAAAAAAAAAAAAAAAAAAAAAAAAAAAAAAAAAAAAAAAAAAAAAAAAAAAAAAAAAAAAAAAAAAAAAAAAAAAAAAAAAAAAAAAAAAAAAAAAAAAAAAAAAAAAAAAAAAAAAAAAAAAAAAAAAAAAAAAAAAAAAAAAAAAAAAAAAAAAAAAAAAAAAAAAAAAAAAAAAAAAAAAAAAAAAAAAAAAAAAAAAAAAAAAAAAAAAAAAAAAAAAAAAAAAAAAAAAAAAAAAAAAAAAAAAAAAAAAAAAAAAAAAAAAAAAAAAAAAAAAAAAAAAAAAAAAAAAAAAAAAAAAAAAAAAAAAAAAAAAAAAAAAAAAAAAAAAAAAAAAAAAAAAAAAAAAAAAAAAAAAAAAAAAAAAAAAAAAAAAAAAAAAAAAAAAAAAAAAAAAAAAAAAAAAAAAAAAAAAAAAAAAAAAAAAAAAAAAAAAAAAAAAAAAAAAAAAAAAAAAAAAAAAAAAAAAAAAAAAAAAAAAAAAAAAAAAAAAAAAAAAAAAAAAAAAAAAAAAAAAAAAAAAAAAAAAAAAAAAAAAAAAAAAAAAAAAAAAAAAAAAAAAAAAAAAAAAAAAAAAAAAAAAAAAAAAAAAAAAAAAAAAAAAAAAAAAAAAAAAAAAAAAAAAAAAAAAAAAAAAAAAAAAAAAAAAAAAAAAAAAAAAAAAAAAAAAAAAAAAAAAAAAAAAAAAAAAAAAAAAAAAAAAAAAAAAAAAAAAAAAAAAAAAAAAAAAAAAAAAAAAAAAAAAAAAAAAAAAAAAAAAAAAAAAAAAAAAAAAAAAAAAAAAAAAAAAAAAAAAAAAAAAAAAAAAAAAAAAAAAAAAAAAAAAAAAAAAAAAAAAAAAAAAAAAAAAAAAAAAAAAAAAAAAAAAAAAAAAAAAAAAAAAAAAAAAAAAAAAAAAAAAAAAAAAAAAAAAAAAAAAAAAAAAAAAAAAAAAAAAAAAAAAAAAAAAAAAAAAAAAAAAAAAAAAAAAAAAAAAAAAAAAAAAAAAAAAAAAAAAAAAAAAAAAAAAAAAAAAAAAAAAAAAAAAAAAAAAAAAAAAAAAAAAAAAAAAAAAAAAAAAAAAAAAAAAAAAAAAAAAAAAAAAAAAAAAAAAAAAAAAAAAAAAAAAAAAAAAAAAAAAAAAAAAAAAAAAAAAAAAAAAAAAAAAAA}
\FloatBarrier

% Exercício 33: MAT_P4FUNCOE_1GX_ICX_016.tex
% Exercise ID: MAT_P4FUNCOE_1GX_ICX_016
% Created: 2025-11-27
% Difficulty: 2/5

\exercicio{Questão com caracteres especiais: % $ & { } [ ] ~ # _ ^ \}
\FloatBarrier

% Exercício 34: MAT_P4FUNCOE_1GX_ICX_017.tex
% Exercise ID: MAT_P4FUNCOE_1GX_ICX_017
% Created: 2025-11-27
% Difficulty: 2/5

\exercicio{Considere a seguinte situação: Numa loja, o preço de um produto é representado por x (em euros) e a despesa total dos clientes é dada por D(x). Indique se as seguintes afirmações são verdadeiras ou falsas, justificando a sua resposta.}
\FloatBarrier

% Exercício 35: MAT_P4FUNCOE_1GX_ICX_018.tex
% Exercise ID: MAT_P4FUNCOE_1GX_ICX_018
% Created: 2025-11-27
% Difficulty: 2/5

\exercicio{AAAAAAAAAAAAAAAAAAAAAAAAAAAAAAAAAAAAAAAAAAAAAAAAAAAAAAAAAAAAAAAAAAAAAAAAAAAAAAAAAAAAAAAAAAAAAAAAAAAAAAAAAAAAAAAAAAAAAAAAAAAAAAAAAAAAAAAAAAAAAAAAAAAAAAAAAAAAAAAAAAAAAAAAAAAAAAAAAAAAAAAAAAAAAAAAAAAAAAAAAAAAAAAAAAAAAAAAAAAAAAAAAAAAAAAAAAAAAAAAAAAAAAAAAAAAAAAAAAAAAAAAAAAAAAAAAAAAAAAAAAAAAAAAAAAAAAAAAAAAAAAAAAAAAAAAAAAAAAAAAAAAAAAAAAAAAAAAAAAAAAAAAAAAAAAAAAAAAAAAAAAAAAAAAAAAAAAAAAAAAAAAAAAAAAAAAAAAAAAAAAAAAAAAAAAAAAAAAAAAAAAAAAAAAAAAAAAAAAAAAAAAAAAAAAAAAAAAAAAAAAAAAAAAAAAAAAAAAAAAAAAAAAAAAAAAAAAAAAAAAAAAAAAAAAAAAAAAAAAAAAAAAAAAAAAAAAAAAAAAAAAAAAAAAAAAAAAAAAAAAAAAAAAAAAAAAAAAAAAAAAAAAAAAAAAAAAAAAAAAAAAAAAAAAAAAAAAAAAAAAAAAAAAAAAAAAAAAAAAAAAAAAAAAAAAAAAAAAAAAAAAAAAAAAAAAAAAAAAAAAAAAAAAAAAAAAAAAAAAAAAAAAAAAAAAAAAAAAAAAAAAAAAAAAAAAAAAAAAAAAAAAAAAAAAAAAAAAAAAAAAAAAAAAAAAAAAAAAAAAAAAAAAAAAAAAAAAAAAAAAAAAAAAAAAAAAAAAAAAAAAAAAAAAAAAAAAAAAAAAAAAAAAAAAAAAAAAAAAAAAAAAAAAAAAAAAAAAAAAAAAAAAAAAAAAAAAAAAAAAAAAAAAAAAAAAAAAAAAAAAAAAAAAAAAAAAAAAAAAAAAAAAAAAAAAAAAAAAAAAAAAAAAAAAAAAAAAAAAAAAAAAAAAAAAAAAAAAAAAAAAAAAAAAAAAAAAAAAAAAAAAAAAAAAAAAAAAAAAAAAAAAAAAAAAAAAAAAAAAAAAAAAAAAAAAAAAAAAAAAAAAAAAAAAAAAAAAAAAAAAAAAAAAAAAAAAAAAAAAAAAAAAAAAAAAAAAAAAAAAAAAAAAAAAAAAAAAAAAAAAAAAAAAAAAAAAAAAAAAAAAAAAAAAAAAAAAAAAAAAAAAAAAAAAAAAAAAAAAAAAAAAAAAAAAAAAAAAAAAAAAAAAAAAAAAAAAAAAAAAAAAAAAAAAAAAAAAAAAAAAAAAAAAAAAAAAAAAAAAAAAAAAAAAAAAAAAAAAAAAAAAAAAAAAAAAAAAAAAAAAAAAAAAAAAAAAAAAAAAAAAAAAAAAAAAAAAAAAAAAAAAAAAAAAAAAAAAAAAAAAAAAAAAAAAAAAAAAAAAAAAAAAAAAAAAAAAAAAAAAAAAAAAAAAAAAAAAAAAAAAAAAAAAAAAAAAAAAAAAAAAAAAAAAAAAAAAAAAAAAAAAAAAAAAAAAAAAAAAAAAAAAAAAAAAAAAAAAAAAAAAAAAAAAAAAAAAAAAAAAAAAAAAAAAAAAAAAAAAAAAAAAAAAAAAAAAAAAAAAAAAAAAAAAAAAAAAAAAAAAAAAAAAAAAAAAAAAAAAAAAAAAAAAAAAAAAAAAAAAAAAAAAAAAAAAAAAAAAAAAAAAAAAAAAAAAAAAAAAAAAAAAAAAAAAAAAAAAAAAAAAAAAAAAAAAAAAAAAAAAAAAAAAAAAAAAAAAAAAAAAAAAAAAAAAAAAAAAAAAAAAAAAAAAAAAAAAAAAAAAAAAAAAAAAAAAAAAAAAAAAAAAAAAAAAAAAAAAAAAAAAAAAAAAAAAAAAAAAAAAAAAAAAAAAAAAAAAAAAAAAAAAAAAAAAAAAAAAAAAAAAAAAAAAAAAAAAAAAAAAAAAAAAAAAAAAAAAAAAAAAAAAAAAAAAAAAAAAAAAAAAAAAAAAAAAAAAAAAAAAAAAAAAAAAAAAAAAAAAAA}
\FloatBarrier

% Exercício 36: MAT_P4FUNCOE_1GX_ICX_019.tex
% Exercise ID: MAT_P4FUNCOE_1GX_ICX_019
% Created: 2025-11-27
% Difficulty: 2/5

\exercicio{Questão com caracteres especiais: % $ & { } [ ] ~ # _ ^ \}
\FloatBarrier

% Exercício 37: MAT_P4FUNCOE_1GX_ICX_020.tex
% Exercise ID: MAT_P4FUNCOE_1GX_ICX_020
% Created: 2025-11-27
% Difficulty: 2/5

\exercicio{Considere a seguinte situação: Numa loja, o preço de um produto é representado por x (em euros) e a despesa total dos clientes é dada por D(x). Indique se as seguintes afirmações são verdadeiras ou falsas, justificando a sua resposta.}
\FloatBarrier

% Exercício 38: MAT_P4FUNCOE_1GX_ICX_021.tex
% Exercise ID: MAT_P4FUNCOE_1GX_ICX_021
% Created: 2025-11-27
% Difficulty: 2/5

\exercicio{AAAAAAAAAAAAAAAAAAAAAAAAAAAAAAAAAAAAAAAAAAAAAAAAAAAAAAAAAAAAAAAAAAAAAAAAAAAAAAAAAAAAAAAAAAAAAAAAAAAAAAAAAAAAAAAAAAAAAAAAAAAAAAAAAAAAAAAAAAAAAAAAAAAAAAAAAAAAAAAAAAAAAAAAAAAAAAAAAAAAAAAAAAAAAAAAAAAAAAAAAAAAAAAAAAAAAAAAAAAAAAAAAAAAAAAAAAAAAAAAAAAAAAAAAAAAAAAAAAAAAAAAAAAAAAAAAAAAAAAAAAAAAAAAAAAAAAAAAAAAAAAAAAAAAAAAAAAAAAAAAAAAAAAAAAAAAAAAAAAAAAAAAAAAAAAAAAAAAAAAAAAAAAAAAAAAAAAAAAAAAAAAAAAAAAAAAAAAAAAAAAAAAAAAAAAAAAAAAAAAAAAAAAAAAAAAAAAAAAAAAAAAAAAAAAAAAAAAAAAAAAAAAAAAAAAAAAAAAAAAAAAAAAAAAAAAAAAAAAAAAAAAAAAAAAAAAAAAAAAAAAAAAAAAAAAAAAAAAAAAAAAAAAAAAAAAAAAAAAAAAAAAAAAAAAAAAAAAAAAAAAAAAAAAAAAAAAAAAAAAAAAAAAAAAAAAAAAAAAAAAAAAAAAAAAAAAAAAAAAAAAAAAAAAAAAAAAAAAAAAAAAAAAAAAAAAAAAAAAAAAAAAAAAAAAAAAAAAAAAAAAAAAAAAAAAAAAAAAAAAAAAAAAAAAAAAAAAAAAAAAAAAAAAAAAAAAAAAAAAAAAAAAAAAAAAAAAAAAAAAAAAAAAAAAAAAAAAAAAAAAAAAAAAAAAAAAAAAAAAAAAAAAAAAAAAAAAAAAAAAAAAAAAAAAAAAAAAAAAAAAAAAAAAAAAAAAAAAAAAAAAAAAAAAAAAAAAAAAAAAAAAAAAAAAAAAAAAAAAAAAAAAAAAAAAAAAAAAAAAAAAAAAAAAAAAAAAAAAAAAAAAAAAAAAAAAAAAAAAAAAAAAAAAAAAAAAAAAAAAAAAAAAAAAAAAAAAAAAAAAAAAAAAAAAAAAAAAAAAAAAAAAAAAAAAAAAAAAAAAAAAAAAAAAAAAAAAAAAAAAAAAAAAAAAAAAAAAAAAAAAAAAAAAAAAAAAAAAAAAAAAAAAAAAAAAAAAAAAAAAAAAAAAAAAAAAAAAAAAAAAAAAAAAAAAAAAAAAAAAAAAAAAAAAAAAAAAAAAAAAAAAAAAAAAAAAAAAAAAAAAAAAAAAAAAAAAAAAAAAAAAAAAAAAAAAAAAAAAAAAAAAAAAAAAAAAAAAAAAAAAAAAAAAAAAAAAAAAAAAAAAAAAAAAAAAAAAAAAAAAAAAAAAAAAAAAAAAAAAAAAAAAAAAAAAAAAAAAAAAAAAAAAAAAAAAAAAAAAAAAAAAAAAAAAAAAAAAAAAAAAAAAAAAAAAAAAAAAAAAAAAAAAAAAAAAAAAAAAAAAAAAAAAAAAAAAAAAAAAAAAAAAAAAAAAAAAAAAAAAAAAAAAAAAAAAAAAAAAAAAAAAAAAAAAAAAAAAAAAAAAAAAAAAAAAAAAAAAAAAAAAAAAAAAAAAAAAAAAAAAAAAAAAAAAAAAAAAAAAAAAAAAAAAAAAAAAAAAAAAAAAAAAAAAAAAAAAAAAAAAAAAAAAAAAAAAAAAAAAAAAAAAAAAAAAAAAAAAAAAAAAAAAAAAAAAAAAAAAAAAAAAAAAAAAAAAAAAAAAAAAAAAAAAAAAAAAAAAAAAAAAAAAAAAAAAAAAAAAAAAAAAAAAAAAAAAAAAAAAAAAAAAAAAAAAAAAAAAAAAAAAAAAAAAAAAAAAAAAAAAAAAAAAAAAAAAAAAAAAAAAAAAAAAAAAAAAAAAAAAAAAAAAAAAAAAAAAAAAAAAAAAAAAAAAAAAAAAAAAAAAAAAAAAAAAAAAAAAAAAAAAAAAAAAAAAAAAAAAAAAAAAAAAAAAAAAAAAAAAAAAAAAAAAAAAAAAAAAAAAAAAAAAAAAAAAAAAAAAAAAAAAAAAAAAAAAAAAAAAAA}
\FloatBarrier

% Exercício 39: MAT_P4FUNCOE_1GX_ICX_022.tex
% Exercise ID: MAT_P4FUNCOE_1GX_ICX_022
% Created: 2025-11-27
% Difficulty: 2/5

\exercicio{Questão com caracteres especiais: % $ & { } [ ] ~ # _ ^ \}
\FloatBarrier

% Exercício 40: MAT_P4FUNCOE_1GX_ICX_023.tex
% Exercise ID: MAT_P4FUNCOE_1GX_ICX_023
% Created: 2025-11-27
% Difficulty: 2/5

\exercicio{Considere a seguinte situação: Numa loja, o preço de um produto é representado por x (em euros) e a despesa total dos clientes é dada por D(x). Indique se as seguintes afirmações são verdadeiras ou falsas, justificando a sua resposta.}
\FloatBarrier

% Exercício 41: MAT_P4FUNCOE_1GX_ICX_024.tex
% Exercise ID: MAT_P4FUNCOE_1GX_ICX_024
% Created: 2025-11-27
% Difficulty: 2/5

\exercicio{AAAAAAAAAAAAAAAAAAAAAAAAAAAAAAAAAAAAAAAAAAAAAAAAAAAAAAAAAAAAAAAAAAAAAAAAAAAAAAAAAAAAAAAAAAAAAAAAAAAAAAAAAAAAAAAAAAAAAAAAAAAAAAAAAAAAAAAAAAAAAAAAAAAAAAAAAAAAAAAAAAAAAAAAAAAAAAAAAAAAAAAAAAAAAAAAAAAAAAAAAAAAAAAAAAAAAAAAAAAAAAAAAAAAAAAAAAAAAAAAAAAAAAAAAAAAAAAAAAAAAAAAAAAAAAAAAAAAAAAAAAAAAAAAAAAAAAAAAAAAAAAAAAAAAAAAAAAAAAAAAAAAAAAAAAAAAAAAAAAAAAAAAAAAAAAAAAAAAAAAAAAAAAAAAAAAAAAAAAAAAAAAAAAAAAAAAAAAAAAAAAAAAAAAAAAAAAAAAAAAAAAAAAAAAAAAAAAAAAAAAAAAAAAAAAAAAAAAAAAAAAAAAAAAAAAAAAAAAAAAAAAAAAAAAAAAAAAAAAAAAAAAAAAAAAAAAAAAAAAAAAAAAAAAAAAAAAAAAAAAAAAAAAAAAAAAAAAAAAAAAAAAAAAAAAAAAAAAAAAAAAAAAAAAAAAAAAAAAAAAAAAAAAAAAAAAAAAAAAAAAAAAAAAAAAAAAAAAAAAAAAAAAAAAAAAAAAAAAAAAAAAAAAAAAAAAAAAAAAAAAAAAAAAAAAAAAAAAAAAAAAAAAAAAAAAAAAAAAAAAAAAAAAAAAAAAAAAAAAAAAAAAAAAAAAAAAAAAAAAAAAAAAAAAAAAAAAAAAAAAAAAAAAAAAAAAAAAAAAAAAAAAAAAAAAAAAAAAAAAAAAAAAAAAAAAAAAAAAAAAAAAAAAAAAAAAAAAAAAAAAAAAAAAAAAAAAAAAAAAAAAAAAAAAAAAAAAAAAAAAAAAAAAAAAAAAAAAAAAAAAAAAAAAAAAAAAAAAAAAAAAAAAAAAAAAAAAAAAAAAAAAAAAAAAAAAAAAAAAAAAAAAAAAAAAAAAAAAAAAAAAAAAAAAAAAAAAAAAAAAAAAAAAAAAAAAAAAAAAAAAAAAAAAAAAAAAAAAAAAAAAAAAAAAAAAAAAAAAAAAAAAAAAAAAAAAAAAAAAAAAAAAAAAAAAAAAAAAAAAAAAAAAAAAAAAAAAAAAAAAAAAAAAAAAAAAAAAAAAAAAAAAAAAAAAAAAAAAAAAAAAAAAAAAAAAAAAAAAAAAAAAAAAAAAAAAAAAAAAAAAAAAAAAAAAAAAAAAAAAAAAAAAAAAAAAAAAAAAAAAAAAAAAAAAAAAAAAAAAAAAAAAAAAAAAAAAAAAAAAAAAAAAAAAAAAAAAAAAAAAAAAAAAAAAAAAAAAAAAAAAAAAAAAAAAAAAAAAAAAAAAAAAAAAAAAAAAAAAAAAAAAAAAAAAAAAAAAAAAAAAAAAAAAAAAAAAAAAAAAAAAAAAAAAAAAAAAAAAAAAAAAAAAAAAAAAAAAAAAAAAAAAAAAAAAAAAAAAAAAAAAAAAAAAAAAAAAAAAAAAAAAAAAAAAAAAAAAAAAAAAAAAAAAAAAAAAAAAAAAAAAAAAAAAAAAAAAAAAAAAAAAAAAAAAAAAAAAAAAAAAAAAAAAAAAAAAAAAAAAAAAAAAAAAAAAAAAAAAAAAAAAAAAAAAAAAAAAAAAAAAAAAAAAAAAAAAAAAAAAAAAAAAAAAAAAAAAAAAAAAAAAAAAAAAAAAAAAAAAAAAAAAAAAAAAAAAAAAAAAAAAAAAAAAAAAAAAAAAAAAAAAAAAAAAAAAAAAAAAAAAAAAAAAAAAAAAAAAAAAAAAAAAAAAAAAAAAAAAAAAAAAAAAAAAAAAAAAAAAAAAAAAAAAAAAAAAAAAAAAAAAAAAAAAAAAAAAAAAAAAAAAAAAAAAAAAAAAAAAAAAAAAAAAAAAAAAAAAAAAAAAAAAAAAAAAAAAAAAAAAAAAAAAAAAAAAAAAAAAAAAAAAAAAAAAAAAAAAAAAAAAAAAAAAAAAAAAAAAAAAAAAAAAAAAAAAAAAAAAAA}
\FloatBarrier

% Exercício 42: MAT_P4FUNCOE_1GX_ICX_025.tex
% Exercise ID: MAT_P4FUNCOE_1GX_ICX_025
% Created: 2025-11-27
% Difficulty: 2/5

\exercicio{Questão com caracteres especiais: % $ & { } [ ] ~ # _ ^ \}
\FloatBarrier

% Exercício 43: MAT_P4FUNCOE_1GX_ICX_026.tex
% Exercise ID: MAT_P4FUNCOE_1GX_ICX_026
% Created: 2025-11-27
% Difficulty: 2/5

\exercicio{Considere a seguinte situação: Numa loja, o preço de um produto é representado por x (em euros) e a despesa total dos clientes é dada por D(x). Indique se as seguintes afirmações são verdadeiras ou falsas, justificando a sua resposta.}
\FloatBarrier

% Exercício 44: MAT_P4FUNCOE_1GX_ICX_027.tex
% Exercise ID: MAT_P4FUNCOE_1GX_ICX_027
% Created: 2025-11-27
% Difficulty: 2/5

\exercicio{AAAAAAAAAAAAAAAAAAAAAAAAAAAAAAAAAAAAAAAAAAAAAAAAAAAAAAAAAAAAAAAAAAAAAAAAAAAAAAAAAAAAAAAAAAAAAAAAAAAAAAAAAAAAAAAAAAAAAAAAAAAAAAAAAAAAAAAAAAAAAAAAAAAAAAAAAAAAAAAAAAAAAAAAAAAAAAAAAAAAAAAAAAAAAAAAAAAAAAAAAAAAAAAAAAAAAAAAAAAAAAAAAAAAAAAAAAAAAAAAAAAAAAAAAAAAAAAAAAAAAAAAAAAAAAAAAAAAAAAAAAAAAAAAAAAAAAAAAAAAAAAAAAAAAAAAAAAAAAAAAAAAAAAAAAAAAAAAAAAAAAAAAAAAAAAAAAAAAAAAAAAAAAAAAAAAAAAAAAAAAAAAAAAAAAAAAAAAAAAAAAAAAAAAAAAAAAAAAAAAAAAAAAAAAAAAAAAAAAAAAAAAAAAAAAAAAAAAAAAAAAAAAAAAAAAAAAAAAAAAAAAAAAAAAAAAAAAAAAAAAAAAAAAAAAAAAAAAAAAAAAAAAAAAAAAAAAAAAAAAAAAAAAAAAAAAAAAAAAAAAAAAAAAAAAAAAAAAAAAAAAAAAAAAAAAAAAAAAAAAAAAAAAAAAAAAAAAAAAAAAAAAAAAAAAAAAAAAAAAAAAAAAAAAAAAAAAAAAAAAAAAAAAAAAAAAAAAAAAAAAAAAAAAAAAAAAAAAAAAAAAAAAAAAAAAAAAAAAAAAAAAAAAAAAAAAAAAAAAAAAAAAAAAAAAAAAAAAAAAAAAAAAAAAAAAAAAAAAAAAAAAAAAAAAAAAAAAAAAAAAAAAAAAAAAAAAAAAAAAAAAAAAAAAAAAAAAAAAAAAAAAAAAAAAAAAAAAAAAAAAAAAAAAAAAAAAAAAAAAAAAAAAAAAAAAAAAAAAAAAAAAAAAAAAAAAAAAAAAAAAAAAAAAAAAAAAAAAAAAAAAAAAAAAAAAAAAAAAAAAAAAAAAAAAAAAAAAAAAAAAAAAAAAAAAAAAAAAAAAAAAAAAAAAAAAAAAAAAAAAAAAAAAAAAAAAAAAAAAAAAAAAAAAAAAAAAAAAAAAAAAAAAAAAAAAAAAAAAAAAAAAAAAAAAAAAAAAAAAAAAAAAAAAAAAAAAAAAAAAAAAAAAAAAAAAAAAAAAAAAAAAAAAAAAAAAAAAAAAAAAAAAAAAAAAAAAAAAAAAAAAAAAAAAAAAAAAAAAAAAAAAAAAAAAAAAAAAAAAAAAAAAAAAAAAAAAAAAAAAAAAAAAAAAAAAAAAAAAAAAAAAAAAAAAAAAAAAAAAAAAAAAAAAAAAAAAAAAAAAAAAAAAAAAAAAAAAAAAAAAAAAAAAAAAAAAAAAAAAAAAAAAAAAAAAAAAAAAAAAAAAAAAAAAAAAAAAAAAAAAAAAAAAAAAAAAAAAAAAAAAAAAAAAAAAAAAAAAAAAAAAAAAAAAAAAAAAAAAAAAAAAAAAAAAAAAAAAAAAAAAAAAAAAAAAAAAAAAAAAAAAAAAAAAAAAAAAAAAAAAAAAAAAAAAAAAAAAAAAAAAAAAAAAAAAAAAAAAAAAAAAAAAAAAAAAAAAAAAAAAAAAAAAAAAAAAAAAAAAAAAAAAAAAAAAAAAAAAAAAAAAAAAAAAAAAAAAAAAAAAAAAAAAAAAAAAAAAAAAAAAAAAAAAAAAAAAAAAAAAAAAAAAAAAAAAAAAAAAAAAAAAAAAAAAAAAAAAAAAAAAAAAAAAAAAAAAAAAAAAAAAAAAAAAAAAAAAAAAAAAAAAAAAAAAAAAAAAAAAAAAAAAAAAAAAAAAAAAAAAAAAAAAAAAAAAAAAAAAAAAAAAAAAAAAAAAAAAAAAAAAAAAAAAAAAAAAAAAAAAAAAAAAAAAAAAAAAAAAAAAAAAAAAAAAAAAAAAAAAAAAAAAAAAAAAAAAAAAAAAAAAAAAAAAAAAAAAAAAAAAAAAAAAAAAAAAAAAAAAAAAAAAAAAAAAAAAAAAAAAAAAAAAAAAAAAAAAAAAAAAAAAAAAAAAAAAAAAAAAAA}
\FloatBarrier

% Exercício 45: MAT_P4FUNCOE_1GX_ICX_028.tex
% Exercise ID: MAT_P4FUNCOE_1GX_ICX_028
% Created: 2025-11-27
% Difficulty: 2/5

\exercicio{Questão com caracteres especiais: % $ & { } [ ] ~ # _ ^ \}
\FloatBarrier

% Exercício 46: MAT_P4FUNCOE_1GX_ICX_029.tex
% Exercise ID: MAT_P4FUNCOE_1GX_ICX_029
% Created: 2025-11-27
% Difficulty: 2/5

\exercicio{Considere a seguinte situação: Numa loja, o preço de um produto é representado por x (em euros) e a despesa total dos clientes é dada por D(x). Indique se as seguintes afirmações são verdadeiras ou falsas, justificando a sua resposta.}
\FloatBarrier

% Exercício 47: MAT_P4FUNCOE_1GX_ICX_030.tex
% Exercise ID: MAT_P4FUNCOE_1GX_ICX_030
% Created: 2025-11-27
% Difficulty: 2/5

\exercicio{AAAAAAAAAAAAAAAAAAAAAAAAAAAAAAAAAAAAAAAAAAAAAAAAAAAAAAAAAAAAAAAAAAAAAAAAAAAAAAAAAAAAAAAAAAAAAAAAAAAAAAAAAAAAAAAAAAAAAAAAAAAAAAAAAAAAAAAAAAAAAAAAAAAAAAAAAAAAAAAAAAAAAAAAAAAAAAAAAAAAAAAAAAAAAAAAAAAAAAAAAAAAAAAAAAAAAAAAAAAAAAAAAAAAAAAAAAAAAAAAAAAAAAAAAAAAAAAAAAAAAAAAAAAAAAAAAAAAAAAAAAAAAAAAAAAAAAAAAAAAAAAAAAAAAAAAAAAAAAAAAAAAAAAAAAAAAAAAAAAAAAAAAAAAAAAAAAAAAAAAAAAAAAAAAAAAAAAAAAAAAAAAAAAAAAAAAAAAAAAAAAAAAAAAAAAAAAAAAAAAAAAAAAAAAAAAAAAAAAAAAAAAAAAAAAAAAAAAAAAAAAAAAAAAAAAAAAAAAAAAAAAAAAAAAAAAAAAAAAAAAAAAAAAAAAAAAAAAAAAAAAAAAAAAAAAAAAAAAAAAAAAAAAAAAAAAAAAAAAAAAAAAAAAAAAAAAAAAAAAAAAAAAAAAAAAAAAAAAAAAAAAAAAAAAAAAAAAAAAAAAAAAAAAAAAAAAAAAAAAAAAAAAAAAAAAAAAAAAAAAAAAAAAAAAAAAAAAAAAAAAAAAAAAAAAAAAAAAAAAAAAAAAAAAAAAAAAAAAAAAAAAAAAAAAAAAAAAAAAAAAAAAAAAAAAAAAAAAAAAAAAAAAAAAAAAAAAAAAAAAAAAAAAAAAAAAAAAAAAAAAAAAAAAAAAAAAAAAAAAAAAAAAAAAAAAAAAAAAAAAAAAAAAAAAAAAAAAAAAAAAAAAAAAAAAAAAAAAAAAAAAAAAAAAAAAAAAAAAAAAAAAAAAAAAAAAAAAAAAAAAAAAAAAAAAAAAAAAAAAAAAAAAAAAAAAAAAAAAAAAAAAAAAAAAAAAAAAAAAAAAAAAAAAAAAAAAAAAAAAAAAAAAAAAAAAAAAAAAAAAAAAAAAAAAAAAAAAAAAAAAAAAAAAAAAAAAAAAAAAAAAAAAAAAAAAAAAAAAAAAAAAAAAAAAAAAAAAAAAAAAAAAAAAAAAAAAAAAAAAAAAAAAAAAAAAAAAAAAAAAAAAAAAAAAAAAAAAAAAAAAAAAAAAAAAAAAAAAAAAAAAAAAAAAAAAAAAAAAAAAAAAAAAAAAAAAAAAAAAAAAAAAAAAAAAAAAAAAAAAAAAAAAAAAAAAAAAAAAAAAAAAAAAAAAAAAAAAAAAAAAAAAAAAAAAAAAAAAAAAAAAAAAAAAAAAAAAAAAAAAAAAAAAAAAAAAAAAAAAAAAAAAAAAAAAAAAAAAAAAAAAAAAAAAAAAAAAAAAAAAAAAAAAAAAAAAAAAAAAAAAAAAAAAAAAAAAAAAAAAAAAAAAAAAAAAAAAAAAAAAAAAAAAAAAAAAAAAAAAAAAAAAAAAAAAAAAAAAAAAAAAAAAAAAAAAAAAAAAAAAAAAAAAAAAAAAAAAAAAAAAAAAAAAAAAAAAAAAAAAAAAAAAAAAAAAAAAAAAAAAAAAAAAAAAAAAAAAAAAAAAAAAAAAAAAAAAAAAAAAAAAAAAAAAAAAAAAAAAAAAAAAAAAAAAAAAAAAAAAAAAAAAAAAAAAAAAAAAAAAAAAAAAAAAAAAAAAAAAAAAAAAAAAAAAAAAAAAAAAAAAAAAAAAAAAAAAAAAAAAAAAAAAAAAAAAAAAAAAAAAAAAAAAAAAAAAAAAAAAAAAAAAAAAAAAAAAAAAAAAAAAAAAAAAAAAAAAAAAAAAAAAAAAAAAAAAAAAAAAAAAAAAAAAAAAAAAAAAAAAAAAAAAAAAAAAAAAAAAAAAAAAAAAAAAAAAAAAAAAAAAAAAAAAAAAAAAAAAAAAAAAAAAAAAAAAAAAAAAAAAAAAAAAAAAAAAAAAAAAAAAAAAAAAAAAAAAAAAAAAAAAAAAAAAAAAAAAAAAAAAAAAAAAAAAAAAAAAAAAAA}
\FloatBarrier

% Exercício 48: MAT_P4FUNCOE_1GX_ICX_031.tex
% Exercise ID: MAT_P4FUNCOE_1GX_ICX_031
% Created: 2025-11-27
% Difficulty: 2/5

\exercicio{Questão com caracteres especiais: % $ & { } [ ] ~ # _ ^ \}
\FloatBarrier

% Exercício 49: MAT_P4FUNCOE_1GX_ICX_032.tex
% Exercise ID: MAT_P4FUNCOE_1GX_ICX_032
% Created: 2025-11-27
% Difficulty: 2/5

\exercicio{Considere a seguinte situação: Numa loja, o preço de um produto é representado por x (em euros) e a despesa total dos clientes é dada por D(x). Indique se as seguintes afirmações são verdadeiras ou falsas, justificando a sua resposta.}
\FloatBarrier

% Exercício 50: MAT_P4FUNCOE_1GX_ICX_033.tex
% Exercise ID: MAT_P4FUNCOE_1GX_ICX_033
% Created: 2025-11-27
% Difficulty: 2/5

\exercicio{AAAAAAAAAAAAAAAAAAAAAAAAAAAAAAAAAAAAAAAAAAAAAAAAAAAAAAAAAAAAAAAAAAAAAAAAAAAAAAAAAAAAAAAAAAAAAAAAAAAAAAAAAAAAAAAAAAAAAAAAAAAAAAAAAAAAAAAAAAAAAAAAAAAAAAAAAAAAAAAAAAAAAAAAAAAAAAAAAAAAAAAAAAAAAAAAAAAAAAAAAAAAAAAAAAAAAAAAAAAAAAAAAAAAAAAAAAAAAAAAAAAAAAAAAAAAAAAAAAAAAAAAAAAAAAAAAAAAAAAAAAAAAAAAAAAAAAAAAAAAAAAAAAAAAAAAAAAAAAAAAAAAAAAAAAAAAAAAAAAAAAAAAAAAAAAAAAAAAAAAAAAAAAAAAAAAAAAAAAAAAAAAAAAAAAAAAAAAAAAAAAAAAAAAAAAAAAAAAAAAAAAAAAAAAAAAAAAAAAAAAAAAAAAAAAAAAAAAAAAAAAAAAAAAAAAAAAAAAAAAAAAAAAAAAAAAAAAAAAAAAAAAAAAAAAAAAAAAAAAAAAAAAAAAAAAAAAAAAAAAAAAAAAAAAAAAAAAAAAAAAAAAAAAAAAAAAAAAAAAAAAAAAAAAAAAAAAAAAAAAAAAAAAAAAAAAAAAAAAAAAAAAAAAAAAAAAAAAAAAAAAAAAAAAAAAAAAAAAAAAAAAAAAAAAAAAAAAAAAAAAAAAAAAAAAAAAAAAAAAAAAAAAAAAAAAAAAAAAAAAAAAAAAAAAAAAAAAAAAAAAAAAAAAAAAAAAAAAAAAAAAAAAAAAAAAAAAAAAAAAAAAAAAAAAAAAAAAAAAAAAAAAAAAAAAAAAAAAAAAAAAAAAAAAAAAAAAAAAAAAAAAAAAAAAAAAAAAAAAAAAAAAAAAAAAAAAAAAAAAAAAAAAAAAAAAAAAAAAAAAAAAAAAAAAAAAAAAAAAAAAAAAAAAAAAAAAAAAAAAAAAAAAAAAAAAAAAAAAAAAAAAAAAAAAAAAAAAAAAAAAAAAAAAAAAAAAAAAAAAAAAAAAAAAAAAAAAAAAAAAAAAAAAAAAAAAAAAAAAAAAAAAAAAAAAAAAAAAAAAAAAAAAAAAAAAAAAAAAAAAAAAAAAAAAAAAAAAAAAAAAAAAAAAAAAAAAAAAAAAAAAAAAAAAAAAAAAAAAAAAAAAAAAAAAAAAAAAAAAAAAAAAAAAAAAAAAAAAAAAAAAAAAAAAAAAAAAAAAAAAAAAAAAAAAAAAAAAAAAAAAAAAAAAAAAAAAAAAAAAAAAAAAAAAAAAAAAAAAAAAAAAAAAAAAAAAAAAAAAAAAAAAAAAAAAAAAAAAAAAAAAAAAAAAAAAAAAAAAAAAAAAAAAAAAAAAAAAAAAAAAAAAAAAAAAAAAAAAAAAAAAAAAAAAAAAAAAAAAAAAAAAAAAAAAAAAAAAAAAAAAAAAAAAAAAAAAAAAAAAAAAAAAAAAAAAAAAAAAAAAAAAAAAAAAAAAAAAAAAAAAAAAAAAAAAAAAAAAAAAAAAAAAAAAAAAAAAAAAAAAAAAAAAAAAAAAAAAAAAAAAAAAAAAAAAAAAAAAAAAAAAAAAAAAAAAAAAAAAAAAAAAAAAAAAAAAAAAAAAAAAAAAAAAAAAAAAAAAAAAAAAAAAAAAAAAAAAAAAAAAAAAAAAAAAAAAAAAAAAAAAAAAAAAAAAAAAAAAAAAAAAAAAAAAAAAAAAAAAAAAAAAAAAAAAAAAAAAAAAAAAAAAAAAAAAAAAAAAAAAAAAAAAAAAAAAAAAAAAAAAAAAAAAAAAAAAAAAAAAAAAAAAAAAAAAAAAAAAAAAAAAAAAAAAAAAAAAAAAAAAAAAAAAAAAAAAAAAAAAAAAAAAAAAAAAAAAAAAAAAAAAAAAAAAAAAAAAAAAAAAAAAAAAAAAAAAAAAAAAAAAAAAAAAAAAAAAAAAAAAAAAAAAAAAAAAAAAAAAAAAAAAAAAAAAAAAAAAAAAAAAAAAAAAAAAAAAAAAAAAAAAAAAAAAAAAAAAAAAAAAAAAAAAAAAAAAAAAAAAAA}
\FloatBarrier

% Exercício 51: MAT_P4FUNCOE_1GX_ICX_034.tex
% Exercise ID: MAT_P4FUNCOE_1GX_ICX_034
% Created: 2025-11-27
% Difficulty: 2/5

\exercicio{Questão com caracteres especiais: % $ & { } [ ] ~ # _ ^ \}
\FloatBarrier

% Exercício 52: MAT_P4FUNCOE_1GX_ICX_035.tex
% Exercise ID: MAT_P4FUNCOE_1GX_ICX_035
% Created: 2025-11-27
% Difficulty: 2/5

\exercicio{Considere a seguinte situação: Numa loja, o preço de um produto é representado por x (em euros) e a despesa total dos clientes é dada por D(x). Indique se as seguintes afirmações são verdadeiras ou falsas, justificando a sua resposta.}
\FloatBarrier

% Exercício 53: MAT_P4FUNCOE_1GX_ICX_036.tex
% Exercise ID: MAT_P4FUNCOE_1GX_ICX_036
% Created: 2025-11-27
% Difficulty: 2/5

\exercicio{AAAAAAAAAAAAAAAAAAAAAAAAAAAAAAAAAAAAAAAAAAAAAAAAAAAAAAAAAAAAAAAAAAAAAAAAAAAAAAAAAAAAAAAAAAAAAAAAAAAAAAAAAAAAAAAAAAAAAAAAAAAAAAAAAAAAAAAAAAAAAAAAAAAAAAAAAAAAAAAAAAAAAAAAAAAAAAAAAAAAAAAAAAAAAAAAAAAAAAAAAAAAAAAAAAAAAAAAAAAAAAAAAAAAAAAAAAAAAAAAAAAAAAAAAAAAAAAAAAAAAAAAAAAAAAAAAAAAAAAAAAAAAAAAAAAAAAAAAAAAAAAAAAAAAAAAAAAAAAAAAAAAAAAAAAAAAAAAAAAAAAAAAAAAAAAAAAAAAAAAAAAAAAAAAAAAAAAAAAAAAAAAAAAAAAAAAAAAAAAAAAAAAAAAAAAAAAAAAAAAAAAAAAAAAAAAAAAAAAAAAAAAAAAAAAAAAAAAAAAAAAAAAAAAAAAAAAAAAAAAAAAAAAAAAAAAAAAAAAAAAAAAAAAAAAAAAAAAAAAAAAAAAAAAAAAAAAAAAAAAAAAAAAAAAAAAAAAAAAAAAAAAAAAAAAAAAAAAAAAAAAAAAAAAAAAAAAAAAAAAAAAAAAAAAAAAAAAAAAAAAAAAAAAAAAAAAAAAAAAAAAAAAAAAAAAAAAAAAAAAAAAAAAAAAAAAAAAAAAAAAAAAAAAAAAAAAAAAAAAAAAAAAAAAAAAAAAAAAAAAAAAAAAAAAAAAAAAAAAAAAAAAAAAAAAAAAAAAAAAAAAAAAAAAAAAAAAAAAAAAAAAAAAAAAAAAAAAAAAAAAAAAAAAAAAAAAAAAAAAAAAAAAAAAAAAAAAAAAAAAAAAAAAAAAAAAAAAAAAAAAAAAAAAAAAAAAAAAAAAAAAAAAAAAAAAAAAAAAAAAAAAAAAAAAAAAAAAAAAAAAAAAAAAAAAAAAAAAAAAAAAAAAAAAAAAAAAAAAAAAAAAAAAAAAAAAAAAAAAAAAAAAAAAAAAAAAAAAAAAAAAAAAAAAAAAAAAAAAAAAAAAAAAAAAAAAAAAAAAAAAAAAAAAAAAAAAAAAAAAAAAAAAAAAAAAAAAAAAAAAAAAAAAAAAAAAAAAAAAAAAAAAAAAAAAAAAAAAAAAAAAAAAAAAAAAAAAAAAAAAAAAAAAAAAAAAAAAAAAAAAAAAAAAAAAAAAAAAAAAAAAAAAAAAAAAAAAAAAAAAAAAAAAAAAAAAAAAAAAAAAAAAAAAAAAAAAAAAAAAAAAAAAAAAAAAAAAAAAAAAAAAAAAAAAAAAAAAAAAAAAAAAAAAAAAAAAAAAAAAAAAAAAAAAAAAAAAAAAAAAAAAAAAAAAAAAAAAAAAAAAAAAAAAAAAAAAAAAAAAAAAAAAAAAAAAAAAAAAAAAAAAAAAAAAAAAAAAAAAAAAAAAAAAAAAAAAAAAAAAAAAAAAAAAAAAAAAAAAAAAAAAAAAAAAAAAAAAAAAAAAAAAAAAAAAAAAAAAAAAAAAAAAAAAAAAAAAAAAAAAAAAAAAAAAAAAAAAAAAAAAAAAAAAAAAAAAAAAAAAAAAAAAAAAAAAAAAAAAAAAAAAAAAAAAAAAAAAAAAAAAAAAAAAAAAAAAAAAAAAAAAAAAAAAAAAAAAAAAAAAAAAAAAAAAAAAAAAAAAAAAAAAAAAAAAAAAAAAAAAAAAAAAAAAAAAAAAAAAAAAAAAAAAAAAAAAAAAAAAAAAAAAAAAAAAAAAAAAAAAAAAAAAAAAAAAAAAAAAAAAAAAAAAAAAAAAAAAAAAAAAAAAAAAAAAAAAAAAAAAAAAAAAAAAAAAAAAAAAAAAAAAAAAAAAAAAAAAAAAAAAAAAAAAAAAAAAAAAAAAAAAAAAAAAAAAAAAAAAAAAAAAAAAAAAAAAAAAAAAAAAAAAAAAAAAAAAAAAAAAAAAAAAAAAAAAAAAAAAAAAAAAAAAAAAAAAAAAAAAAAAAAAAAAAAAAAAAAAAAAAAAAAAAAAAAAAAAAAAAAAAAAAAAAAAAAAAAAAAAAA}
\FloatBarrier

% Exercício 54: MAT_P4FUNCOE_1GX_ICX_037.tex
% Exercise ID: MAT_P4FUNCOE_1GX_ICX_037
% Created: 2025-11-27
% Difficulty: 2/5

\exercicio{Questão com caracteres especiais: % $ & { } [ ] ~ # _ ^ \}
\FloatBarrier

% Exercício 55: MAT_P4FUNCOE_1GX_ICX_038.tex
% Exercise ID: MAT_P4FUNCOE_1GX_ICX_038
% Created: 2025-11-27
% Difficulty: 2/5

\exercicio{Considere a seguinte situação: Numa loja, o preço de um produto é representado por x (em euros) e a despesa total dos clientes é dada por D(x). Indique se as seguintes afirmações são verdadeiras ou falsas, justificando a sua resposta.}
\FloatBarrier

% Exercício 56: MAT_P4FUNCOE_1GX_ICX_039.tex
% Exercise ID: MAT_P4FUNCOE_1GX_ICX_039
% Created: 2025-11-27
% Difficulty: 2/5

\exercicio{AAAAAAAAAAAAAAAAAAAAAAAAAAAAAAAAAAAAAAAAAAAAAAAAAAAAAAAAAAAAAAAAAAAAAAAAAAAAAAAAAAAAAAAAAAAAAAAAAAAAAAAAAAAAAAAAAAAAAAAAAAAAAAAAAAAAAAAAAAAAAAAAAAAAAAAAAAAAAAAAAAAAAAAAAAAAAAAAAAAAAAAAAAAAAAAAAAAAAAAAAAAAAAAAAAAAAAAAAAAAAAAAAAAAAAAAAAAAAAAAAAAAAAAAAAAAAAAAAAAAAAAAAAAAAAAAAAAAAAAAAAAAAAAAAAAAAAAAAAAAAAAAAAAAAAAAAAAAAAAAAAAAAAAAAAAAAAAAAAAAAAAAAAAAAAAAAAAAAAAAAAAAAAAAAAAAAAAAAAAAAAAAAAAAAAAAAAAAAAAAAAAAAAAAAAAAAAAAAAAAAAAAAAAAAAAAAAAAAAAAAAAAAAAAAAAAAAAAAAAAAAAAAAAAAAAAAAAAAAAAAAAAAAAAAAAAAAAAAAAAAAAAAAAAAAAAAAAAAAAAAAAAAAAAAAAAAAAAAAAAAAAAAAAAAAAAAAAAAAAAAAAAAAAAAAAAAAAAAAAAAAAAAAAAAAAAAAAAAAAAAAAAAAAAAAAAAAAAAAAAAAAAAAAAAAAAAAAAAAAAAAAAAAAAAAAAAAAAAAAAAAAAAAAAAAAAAAAAAAAAAAAAAAAAAAAAAAAAAAAAAAAAAAAAAAAAAAAAAAAAAAAAAAAAAAAAAAAAAAAAAAAAAAAAAAAAAAAAAAAAAAAAAAAAAAAAAAAAAAAAAAAAAAAAAAAAAAAAAAAAAAAAAAAAAAAAAAAAAAAAAAAAAAAAAAAAAAAAAAAAAAAAAAAAAAAAAAAAAAAAAAAAAAAAAAAAAAAAAAAAAAAAAAAAAAAAAAAAAAAAAAAAAAAAAAAAAAAAAAAAAAAAAAAAAAAAAAAAAAAAAAAAAAAAAAAAAAAAAAAAAAAAAAAAAAAAAAAAAAAAAAAAAAAAAAAAAAAAAAAAAAAAAAAAAAAAAAAAAAAAAAAAAAAAAAAAAAAAAAAAAAAAAAAAAAAAAAAAAAAAAAAAAAAAAAAAAAAAAAAAAAAAAAAAAAAAAAAAAAAAAAAAAAAAAAAAAAAAAAAAAAAAAAAAAAAAAAAAAAAAAAAAAAAAAAAAAAAAAAAAAAAAAAAAAAAAAAAAAAAAAAAAAAAAAAAAAAAAAAAAAAAAAAAAAAAAAAAAAAAAAAAAAAAAAAAAAAAAAAAAAAAAAAAAAAAAAAAAAAAAAAAAAAAAAAAAAAAAAAAAAAAAAAAAAAAAAAAAAAAAAAAAAAAAAAAAAAAAAAAAAAAAAAAAAAAAAAAAAAAAAAAAAAAAAAAAAAAAAAAAAAAAAAAAAAAAAAAAAAAAAAAAAAAAAAAAAAAAAAAAAAAAAAAAAAAAAAAAAAAAAAAAAAAAAAAAAAAAAAAAAAAAAAAAAAAAAAAAAAAAAAAAAAAAAAAAAAAAAAAAAAAAAAAAAAAAAAAAAAAAAAAAAAAAAAAAAAAAAAAAAAAAAAAAAAAAAAAAAAAAAAAAAAAAAAAAAAAAAAAAAAAAAAAAAAAAAAAAAAAAAAAAAAAAAAAAAAAAAAAAAAAAAAAAAAAAAAAAAAAAAAAAAAAAAAAAAAAAAAAAAAAAAAAAAAAAAAAAAAAAAAAAAAAAAAAAAAAAAAAAAAAAAAAAAAAAAAAAAAAAAAAAAAAAAAAAAAAAAAAAAAAAAAAAAAAAAAAAAAAAAAAAAAAAAAAAAAAAAAAAAAAAAAAAAAAAAAAAAAAAAAAAAAAAAAAAAAAAAAAAAAAAAAAAAAAAAAAAAAAAAAAAAAAAAAAAAAAAAAAAAAAAAAAAAAAAAAAAAAAAAAAAAAAAAAAAAAAAAAAAAAAAAAAAAAAAAAAAAAAAAAAAAAAAAAAAAAAAAAAAAAAAAAAAAAAAAAAAAAAAAAAAAAAAAAAAAAAAAAAAAAAAAAAAAAAAAAAAAAAAAAAAAAAAAAAAAAAAAAAA}
\FloatBarrier

% Exercício 57: MAT_P4FUNCOE_1GX_ICX_040.tex
% Exercise ID: MAT_P4FUNCOE_1GX_ICX_040
% Created: 2025-11-27
% Difficulty: 2/5

\exercicio{Questão com caracteres especiais: % $ & { } [ ] ~ # _ ^ \}
\FloatBarrier

% Exercício 58: MAT_P4FUNCOE_1GX_ICX_041.tex
% Exercise ID: MAT_P4FUNCOE_1GX_ICX_041
% Created: 2025-11-28
% Difficulty: 2/5

\exercicio{Considere a seguinte situação: Numa loja, o preço de um produto é representado por x (em euros) e a despesa total dos clientes é dada por D(x). Indique se as seguintes afirmações são verdadeiras ou falsas, justificando a sua resposta.}
\FloatBarrier

% Exercício 59: MAT_P4FUNCOE_1GX_ICX_042.tex
% Exercise ID: MAT_P4FUNCOE_1GX_ICX_042
% Created: 2025-12-02
% Difficulty: 2/5

\exercicio{Considere a seguinte situação: Numa loja, o preço de um produto é representado por x (em euros) e a despesa total dos clientes é dada por D(x). Indique se as seguintes afirmações são verdadeiras ou falsas, justificando a sua resposta.}
\FloatBarrier

% Exercício 60: MAT_P4FUNCOE_1GX_ICX_043.tex
% Exercise ID: MAT_P4FUNCOE_1GX_ICX_043
% Created: 2025-12-02
% Difficulty: 2/5

\exercicio{AAAAAAAAAAAAAAAAAAAAAAAAAAAAAAAAAAAAAAAAAAAAAAAAAAAAAAAAAAAAAAAAAAAAAAAAAAAAAAAAAAAAAAAAAAAAAAAAAAAAAAAAAAAAAAAAAAAAAAAAAAAAAAAAAAAAAAAAAAAAAAAAAAAAAAAAAAAAAAAAAAAAAAAAAAAAAAAAAAAAAAAAAAAAAAAAAAAAAAAAAAAAAAAAAAAAAAAAAAAAAAAAAAAAAAAAAAAAAAAAAAAAAAAAAAAAAAAAAAAAAAAAAAAAAAAAAAAAAAAAAAAAAAAAAAAAAAAAAAAAAAAAAAAAAAAAAAAAAAAAAAAAAAAAAAAAAAAAAAAAAAAAAAAAAAAAAAAAAAAAAAAAAAAAAAAAAAAAAAAAAAAAAAAAAAAAAAAAAAAAAAAAAAAAAAAAAAAAAAAAAAAAAAAAAAAAAAAAAAAAAAAAAAAAAAAAAAAAAAAAAAAAAAAAAAAAAAAAAAAAAAAAAAAAAAAAAAAAAAAAAAAAAAAAAAAAAAAAAAAAAAAAAAAAAAAAAAAAAAAAAAAAAAAAAAAAAAAAAAAAAAAAAAAAAAAAAAAAAAAAAAAAAAAAAAAAAAAAAAAAAAAAAAAAAAAAAAAAAAAAAAAAAAAAAAAAAAAAAAAAAAAAAAAAAAAAAAAAAAAAAAAAAAAAAAAAAAAAAAAAAAAAAAAAAAAAAAAAAAAAAAAAAAAAAAAAAAAAAAAAAAAAAAAAAAAAAAAAAAAAAAAAAAAAAAAAAAAAAAAAAAAAAAAAAAAAAAAAAAAAAAAAAAAAAAAAAAAAAAAAAAAAAAAAAAAAAAAAAAAAAAAAAAAAAAAAAAAAAAAAAAAAAAAAAAAAAAAAAAAAAAAAAAAAAAAAAAAAAAAAAAAAAAAAAAAAAAAAAAAAAAAAAAAAAAAAAAAAAAAAAAAAAAAAAAAAAAAAAAAAAAAAAAAAAAAAAAAAAAAAAAAAAAAAAAAAAAAAAAAAAAAAAAAAAAAAAAAAAAAAAAAAAAAAAAAAAAAAAAAAAAAAAAAAAAAAAAAAAAAAAAAAAAAAAAAAAAAAAAAAAAAAAAAAAAAAAAAAAAAAAAAAAAAAAAAAAAAAAAAAAAAAAAAAAAAAAAAAAAAAAAAAAAAAAAAAAAAAAAAAAAAAAAAAAAAAAAAAAAAAAAAAAAAAAAAAAAAAAAAAAAAAAAAAAAAAAAAAAAAAAAAAAAAAAAAAAAAAAAAAAAAAAAAAAAAAAAAAAAAAAAAAAAAAAAAAAAAAAAAAAAAAAAAAAAAAAAAAAAAAAAAAAAAAAAAAAAAAAAAAAAAAAAAAAAAAAAAAAAAAAAAAAAAAAAAAAAAAAAAAAAAAAAAAAAAAAAAAAAAAAAAAAAAAAAAAAAAAAAAAAAAAAAAAAAAAAAAAAAAAAAAAAAAAAAAAAAAAAAAAAAAAAAAAAAAAAAAAAAAAAAAAAAAAAAAAAAAAAAAAAAAAAAAAAAAAAAAAAAAAAAAAAAAAAAAAAAAAAAAAAAAAAAAAAAAAAAAAAAAAAAAAAAAAAAAAAAAAAAAAAAAAAAAAAAAAAAAAAAAAAAAAAAAAAAAAAAAAAAAAAAAAAAAAAAAAAAAAAAAAAAAAAAAAAAAAAAAAAAAAAAAAAAAAAAAAAAAAAAAAAAAAAAAAAAAAAAAAAAAAAAAAAAAAAAAAAAAAAAAAAAAAAAAAAAAAAAAAAAAAAAAAAAAAAAAAAAAAAAAAAAAAAAAAAAAAAAAAAAAAAAAAAAAAAAAAAAAAAAAAAAAAAAAAAAAAAAAAAAAAAAAAAAAAAAAAAAAAAAAAAAAAAAAAAAAAAAAAAAAAAAAAAAAAAAAAAAAAAAAAAAAAAAAAAAAAAAAAAAAAAAAAAAAAAAAAAAAAAAAAAAAAAAAAAAAAAAAAAAAAAAAAAAAAAAAAAAAAAAAAAAAAAAAAAAAAAAAAAAAAAAAAAAAAAAAAAAAAAAAAAAAAAAAAAAAAAAAAAAAAAAAAAAAAAAAAAAAAAAAA}
\FloatBarrier

% Exercício 61: MAT_P4FUNCOE_1GX_ICX_044.tex
% Exercise ID: MAT_P4FUNCOE_1GX_ICX_044
% Created: 2025-12-02
% Difficulty: 2/5

\exercicio{Questão com caracteres especiais: % $ & { } [ ] ~ # _ ^ \}
\FloatBarrier

% Exercício 62: MAT_P4FUNCOE_1GX_JCM_001.tex
% Exercise ID: MAT_P4FUNCOE_1GX_JCM_001
% Created: 2025-11-27
% Difficulty: 3/5

\exercicio{Considere as seguintes funções que relacionam o preço $p$ (em euros) com a despesa $E(p)$. Para cada uma, analise a afirmação: "Quando o preço aumenta, a despesa aumenta." Decida se a afirmação é verdadeira para todos os valores de $p$; justifique rigorosamente ou apresente um contraexemplo.\\
\begin{enumerate}[a)]
\item $E(p)=10+2p$.
\item $E(p)=100-3p$.
\item $E(p)=\dfrac{50p}{p+10},\quad p
eq -10$.
\item $E(p)=\begin{cases} p^2,& p\ge 0,\\ -p,& p<0.\end{cases}$
\end{enumerate}

% Metadata file: ExerciseDatabase/matematica/P4_funcoes/1-generalidades_funcoes/juizo_causal/MAT_P4FUNCOE_1GX_JCM_001.json
\FloatBarrier

% Exercício 63: MAT_P4FUNCOE_1GEN_001.tex
% Exercise ID: MAT_P4FUNCOE_1GEN_001
% Module: MÓDULO P4 - Funções | Concept: Generalidades acerca de Funções
% Difficulty: 2/5 (Fácil) | Type: desenvolvimento
% Author: Exemplo | Date: 2025-11-15
% Status: active

\exercicio{Defina o conceito de função entre conjuntos e dê dois exemplos simples.
\vspace{3cm}
}
\FloatBarrier

% Exercício 64: MAT_P4FUNCOE_1GX_RVX_001.tex
% Exercise ID: MAT_P4FUNCOE_1GX_RVX_001
% Created: 2025-11-26
% Difficulty: 1/5

\exercicio{Numa fábrica, o número de peças produzidas depende do número de horas de funcionamento das máquinas. Identifique a variável independente e a variável dependente neste contexto.}
\FloatBarrier

% Exercício 65: MAT_P4FUNCOE_1GX_RVX_002.tex
% Exercise ID: MAT_P4FUNCOE_1GX_RVX_002
% Created: 2025-11-26
% Difficulty: 1/5

\exercicio{Um agricultor rega as suas plantas e observa o crescimento das mesmas ao longo do tempo. Indique qual é a variável independente e qual é a variável dependente.}
\FloatBarrier

% Exercício 66: MAT_P4FUNCOE_1GX_RVX_003.tex
% Exercise ID: MAT_P4FUNCOE_1GX_RVX_003
% Created: 2025-11-26
% Difficulty: 1/5

\exercicio{Numa experiência laboratorial, um cientista altera a temperatura de um líquido e regista a quantidade de gás libertado. Identifique as variáveis independente e dependente.}
\FloatBarrier

% Exercício 67: MAT_P4FUNCOE_1GX_RVX_004.tex
% Exercise ID: MAT_P4FUNCOE_1GX_RVX_004
% Created: 2025-11-26
% Difficulty: 1/5

\exercicio{Uma empresa aumenta o investimento em publicidade e analisa o número de novos clientes obtidos. Indique qual é a variável independente e qual é a variável dependente.}
\FloatBarrier

% Exercício 68: MAT_P4FUNCOE_1GX_RVX_005.tex
% Exercise ID: MAT_P4FUNCOE_1GX_RVX_005
% Created: 2025-11-26
% Difficulty: 1/5

\exercicio{Um estudante estuda mais horas por semana e observa a evolução das suas notas nos testes. Identifique a variável independente e a variável dependente neste caso.}
\FloatBarrier

% Exercício 69: MAT_P4FUNCOE_1GX_RVX_006.tex
% Exercise ID: MAT_P4FUNCOE_1GX_RVX_006
% Module: MÓDULO P4 - Funções | Concept: Generalidades acerca de Funções | Type: Reconhecimento Variaveis
% Difficulty: 5/5 (Muito Difícil) | Format: desenvolvimento
% Tags: contradominio, definicao, dominio, teste, automacao
% Author: Teste Automático | Date: 2025-11-26
% Status: active

\exercicio{Exercício de teste para reconhecimento_variaveis}
\FloatBarrier

% Exercício 70: MAT_P4FUNCOE_1GX_RVX_007.tex
% Exercise ID: MAT_P4FUNCOE_1GX_RVX_007
% Module: MÓDULO P4 - Funções | Concept: Generalidades acerca de Funções | Type: Reconhecimento Variaveis
% Difficulty: 1/5 (Muito Fácil) | Format: desenvolvimento
% Tags: dominio, contradominio, definicao, opencode_test, terminal
% Author: opencode-terminal-tester | Date: 2025-11-26
% Status: active

\exercicio{Exercício de teste para reconhecimento_variaveis}

% Solution:
% \begin{solucao}
% Solução exemplo
% \end{solucao}
\FloatBarrier

% Exercício 71: MAT_P4FUNCOE_1GX_RCA_001.tex
% Exercise ID: MAT_P4FUNCOE_1GX_RCA_001
% Created: 2025-11-28
% Difficulty: 3/5

\exercicio{Considere p>0, q(p)\ge 0 e a despesa total D(p)=p\,q(p). Para cada uma das afirmações seguintes: (i) classifique-a como "sempre verdadeira", "às vezes verdadeira" ou "falsa"; (ii) justifique brevemente a sua resposta (prova curta ou contraexemplo).\begin{enumerate}[a)]\item Sempre que o preço p aumenta, a despesa total D(p) aumenta.\item Se q(p) é decrescente em p, então D(p) é decrescente.\item Se q(p)=k/p com k>0, então D(p) é constante e independe de p.\item Se p_1<p_2 e D(p_2)>D(p_1), então necessariamente q(p_2)>q(p_1).\end{enumerate}}
\FloatBarrier

% Exercício 72: MAT_P4FUNCOE_1GX_RCX_001.tex
% Exercise ID: MAT_P4FUNCOE_1GX_RCX_001
% Created: 2025-11-28
% Difficulty: 3/5

\exercicio{'\exercicio{Considere p>0}
\FloatBarrier

% Exercício 73: MAT_P4FUNCOE_1GX_RCE_001.tex
% Exercise ID: MAT_P4FUNCOE_1GX_RCE_001
% Created: 2025-11-28
% Difficulty: 2/5

\exercicio{'Classifique as afirmações seguintes como Verdadeira (V) ou Falsa (F) e justifique brevemente cada resposta.

(a) Quando o preço de um bem aumenta}
\FloatBarrier

% Exercício 74: MAT_P4FUNCOE_1GX_RCE_002.tex
% Exercise ID: MAT_P4FUNCOE_1GX_RCE_002
% Created: 2025-11-28
% Difficulty: 2/5

\exercicio{'Classifique as afirmações seguintes como Verdadeira (V) ou Falsa (F) e justifique brevemente cada resposta.

(a) Quando o preço de um bem aumenta}
\FloatBarrier

% Exercício 75: MAT_P4FUNCOE_1GX_RCE_003.tex
% Exercise ID: MAT_P4FUNCOE_1GX_RCE_003
% Created: 2025-11-28
% Difficulty: 2/5

\exercicio{'Classifique as afirmações seguintes como Verdadeira (V) ou Falsa (F) e justifique brevemente cada resposta.

(a) Quando o preço de um bem aumenta}
\FloatBarrier

% Exercício 76: MAT_P4FUNCOE_1GX_TLX_001.tex
% Exercise ID: MAT_P4FUNCOE_1GX_TLX_001
% Created: 2025-11-27
% Difficulty: 2/5

\exercicio{'Teste logging'}
\FloatBarrier

% Exercício 77: MAT_P4FUNCOE_1GX_TLX_002.tex
% Exercise ID: MAT_P4FUNCOE_1GX_TLX_002
% Created: 2025-11-27
% Difficulty: 2/5

\exercicio{'Teste logging'}
\FloatBarrier

% Exercício 78: MAT_P4FUNCOE_1GX_TLX_003.tex
% Exercise ID: MAT_P4FUNCOE_1GX_TLX_003
% Created: 2025-11-27
% Difficulty: 2/5

\exercicio{'Teste logging'}
\FloatBarrier

% Exercício 79: MAT_P4FUNCOE_1GX_TLX_004.tex
% Exercise ID: MAT_P4FUNCOE_1GX_TLX_004
% Created: 2025-11-27
% Difficulty: 2/5

\exercicio{'Teste logging'}
\FloatBarrier

% Exercício 80: MAT_P4FUNCOE_1GX_TLX_005.tex
% Exercise ID: MAT_P4FUNCOE_1GX_TLX_005
% Created: 2025-11-27
% Difficulty: 2/5

\exercicio{'Teste logging'}
\FloatBarrier

% Exercício 81: MAT_P4FUNCOE_1GX_TLX_006.tex
% Exercise ID: MAT_P4FUNCOE_1GX_TLX_006
% Created: 2025-11-27
% Difficulty: 2/5

\exercicio{'Teste logging'}
\FloatBarrier

% Exercício 82: MAT_P4FUNCOE_1GX_TLX_007.tex
% Exercise ID: MAT_P4FUNCOE_1GX_TLX_007
% Created: 2025-11-27
% Difficulty: 2/5

\exercicio{'Teste logging'}
\FloatBarrier

% Exercício 83: MAT_P4FUNCOE_1GX_TLX_008.tex
% Exercise ID: MAT_P4FUNCOE_1GX_TLX_008
% Created: 2025-11-27
% Difficulty: 2/5

\exercicio{'Teste logging'}
\FloatBarrier

% Exercício 84: MAT_P4FUNCOE_1GX_TLX_009.tex
% Exercise ID: MAT_P4FUNCOE_1GX_TLX_009
% Created: 2025-11-27
% Difficulty: 2/5

\exercicio{'Teste logging'}
\FloatBarrier

% Exercício 85: MAT_P4FUNCOE_1GX_TLX_010.tex
% Exercise ID: MAT_P4FUNCOE_1GX_TLX_010
% Created: 2025-11-27
% Difficulty: 2/5

\exercicio{'Teste logging'}
\FloatBarrier

% Exercício 86: MAT_P4FUNCOE_1GX_TWI_001.tex
% Exercise ID: MAT_P4FUNCOE_1GX_TWI_001
% Created: 2025-11-27
% Difficulty: 2/5

\exercicio{'Teste integração wrapper: classifique afirmações.'}
\FloatBarrier

% Exercício 87: MAT_P4FUNCOE_1GX_TNX_001.tex
% Exercise ID: MAT_P4FUNCOE_1GX_TNX_001
% Created: 2025-11-26
% Difficulty: 2/5

\exercicio{Exemplo de exercício para tipo novo. O script deve criar o diretório e o tipo automaticamente.}
\FloatBarrier

\end{document}
