% Template para geração automática de sebentas
% Gerado automaticamente - NÃO EDITAR MANUALMENTE
\documentclass[12pt,a4paper]{article}

% Encoding e idioma
\usepackage[utf8]{inputenc}
\usepackage[T1]{fontenc}
\usepackage[portuguese]{babel}

% Matemática
\usepackage{amsmath}
\usepackage{amssymb}
\usepackage{amsthm}
\usepackage{mathtools}

% Gráficos e figuras
\usepackage{graphicx}
\usepackage{tikz}
\usetikzlibrary{calc,patterns,angles,quotes}
\usepackage{pgfplots}
\pgfplotsset{compat=1.18}
% Force float barriers when needed to keep figures right after exercises
\usepackage{placeins}
\usepackage{float}  % Para figure[H] - força figuras exatamente onde definidas

% Layout e formatação
\usepackage{geometry}
\geometry{a4paper,margin=2.5cm,top=3cm,bottom=3cm}
\usepackage{fancyhdr}
\usepackage{enumitem}
\usepackage{multicol}

% Hyperlinks e referências
\usepackage{hyperref}
\hypersetup{
    colorlinks=true,
    linkcolor=blue,
    urlcolor=blue,
    citecolor=blue
}

% Sistema de exercícios - macros personalizadas
\newcounter{exerciciocount}
\newcounter{subexerciciocount}
\newcounter{optioncount}

\newcommand{\exercicio}[1][]{%
    \par\vspace{1.5em}%
    \refstepcounter{exerciciocount}%
    \setcounter{subexerciciocount}{0}%
    \setcounter{optioncount}{0}%
    \noindent\textbf{Exercício~\theexerciciocount.}%
    \ifx&#1&%
        % Argumento vazio - o conteúdo vem depois
        \par\vspace{0.3em}%
    \else%
        % Argumento fornecido - incluir inline
        \ #1\par\vspace{0.5em}%
    \fi%
}

\newcommand{\subexercicio}[1]{%
    \par\vspace{0.8em}%
    \refstepcounter{subexerciciocount}%
    \noindent\textbf{\theexerciciocount.\thesubexerciciocount.} #1\par\vspace{0.3em}%
}

\newcommand{\option}[1]{%
    \par
    \refstepcounter{optioncount}%
    \noindent(\alph{optioncount}) #1%
}

% Cabeçalho e rodapé
\pagestyle{fancy}
\fancyhf{}
\fancyhead[L]{MÓDULO P4 - Funções}
\fancyhead[R]{Função Inversa}
\fancyfoot[C]{\thepage}

% Metadados do documento
\title{}
\author{}
\date{}

\begin{document}

% Remover título, autor e data - apenas conteúdo
\thispagestyle{fancy}

\section*{Função Inversa}

% MAT_P4FUNCOE_4FIN_ANA_001.tex
% Exercise ID: MAT_P4FUNCOE_4FIN_ANA_001
% Module: MÓDULO P4 - Funções | Concept: Função Inversa | Type: Determinação Analítica
% Difficulty: 2/5 (Fácil) | Type: desenvolvimento
% Points: 10 | Time: 10 min
% Tags: inversa, funcao_linear, grafico, expressao_analitica
% Author: Professor | Date: 2025-11-18
% Status: active
% Description: Determinar expressão analítica e representar graficamente função e inversa

\exercicio
Considere a função $f(x) = 2x - 3$.

\begin{enumerate}[label=\alph*]
    \item Determine a expressão analítica da função inversa $f^{-1}(x)$.
    
    \vspace{3cm}
    \item Represente graficamente a função $f$ e a sua inversa $f^{-1}$ no mesmo referencial.
    
    \vspace{3cm}

    \begin{center}
    \begin{tikzpicture}[scale=0.8]
        % grid
        \draw[step=1cm,gray!30,very thin] (-6.5,-6.5) grid (6.5,6.5);
        % axes
        \draw[->,thick] (-6.5,0) -- (6.8,0) node[right] {$x$};
        \draw[->,thick] (0,-6.5) -- (0,6.8) node[above] {$y$};
        % origin
        \fill (0,0) circle (0.05) node[below left=2pt,fill=white,inner sep=1pt] {\small $0$};
    \end{tikzpicture}
    \end{center}
\end{enumerate}
\FloatBarrier

% MAT_P4FUNCOE_4FIN_ANA_002.tex
% Exercise ID: MAT_P4FUNCOE_4FIN_ANA_002
% Module: MÓDULO P4 - Funções | Concept: Função Inversa | Type: Determinação Analítica
% Difficulty: 2/5 (Fácil) | Type: desenvolvimento
% Points: 10 | Time: 10 min
% Tags: inversa, expressao_analitica, calculo
% Author: Professor | Date: 2025-11-18
% Status: active
% Description: Calcular a expressão da inversa de funções simples

\exercicio
Determine analiticamente a função inversa de:

\begin{enumerate}[label=\alph*]
    \item $f(x) = x + 1$.
    
    \vspace{3cm}
    
    \item $g(x) = 2x$
\end{enumerate}
\FloatBarrier

% MAT_P4FUNCOE_4FIN_ANA_003.tex
% Exercise ID: MAT_P4FUNCOE_4FIN_005
% Exercise ID: MAT_P4FUNCOE_4FIN_ANA_003
% Module: MÓDULO P4 - Funções | Concept: Função Inversa | Type: Determinação Analítica
% Difficulty: 2/5 (Fácil) | Type: desenvolvimento
% Points: 10 | Time: 10 min
% Tags: inversa, funcao_linear, grafico, expressao_analitica
% Author: Professor | Date: 2025-11-18
% Status: active
% Description: Determinar expressão analítica e representar graficamente função e inversa

\exercicio
Considere a função $f(x) = 2x - 4$.

\begin{enumerate}[label=\alph*]
    \item Determine a expressão analítica da função inversa $f^{-1}(x)$.
    
    \vspace{3cm}
    \item Represente graficamente a função $f$ e a sua inversa $f^{-1}$ no mesmo referencial.
    
    \vspace{3cm}
\end{enumerate}
\FloatBarrier

% MAT_P4FUNCOE_4FIN_ANA_004.tex
% Exercise ID: MAT_P4FUNCOE_4FIN_ANA_004
% Exercise ID: MAT_P4FUNCOE_4FIN_ANA_004
% Module: MÓDULO P4 - Funções | Concept: Função Inversa | Type: Determinação Analítica
% Difficulty: 2/5 (Fácil) | Type: desenvolvimento
% Points: 10 | Time: 10 min
% Tags: inversa, funcao_linear, grafico, expressao_analitica
% Author: Professor | Date: 2025-11-18
% Status: active
% Description: Determinar expressão analítica e representar graficamente função e inversa

\exercicio
Considere a função $f(x) = 3x - 1$.

\begin{enumerate}[label=\alph*]
    \item Determine a expressão analítica da função inversa $f^{-1}(x)$.
    
    \vspace{3cm}
    \item Represente graficamente a função $f$ e a sua inversa $f^{-1}$ no mesmo referencial.
    
    \vspace{3cm}

    \begin{center}
    \begin{tikzpicture}[scale=0.8]
        % grid
        \draw[step=1cm,gray!30,very thin] (-6.5,-6.5) grid (6.5,6.5);
        % axes
        \draw[->,thick] (-6.5,0) -- (6.8,0) node[right] {$x$};
        \draw[->,thick] (0,-6.5) -- (0,6.8) node[above] {$y$};
        % origin
        \fill (0,0) circle (0.05) node[below left=2pt,fill=white,inner sep=1pt] {\small $1$};
    \end{tikzpicture}
    \end{center}
\end{enumerate}
\FloatBarrier

% MAT_P4FUNCOE_4FIN_ANA_005.tex
% Exercise ID: MAT_P4FUNCOE_4FIN_ANA_005
% Exercise ID: MAT_P4FUNCOE_4FIN_ANA_005
% Module: MÓDULO P4 - Funções | Concept: Função Inversa | Type: Determinação Analítica
% Difficulty: 2/5 (Fácil) | Type: desenvolvimento
% Points: 10 | Time: 10 min
% Tags: inversa, funcao_linear, grafico, expressao_analitica
% Author: Professor | Date: 2025-11-18
% Status: active
% Description: Determinar expressão analítica e representar graficamente função e inversa

\exercicio
Considere a função $f(x) = 4x - 1$.

\begin{enumerate}[label=\alph*]
    \item Determine a expressão analítica da função inversa $f^{-1}(x)$.
    
    \vspace{3cm}
    \item Represente graficamente a função $f$ e a sua inversa $f^{-1}$ no mesmo referencial.
    
    \vspace{3cm}

    \begin{center}
    \begin{tikzpicture}[scale=0.8]
        % grid
        \draw[step=1cm,gray!30,very thin] (-6.5,-6.5) grid (6.5,6.5);
        % axes
        \draw[->,thick] (-6.5,0) -- (6.8,0) node[right] {$x$};
        \draw[->,thick] (0,-6.5) -- (0,6.8) node[above] {$y$};
        % origin
        \fill (0,0) circle (0.05) node[below left=2pt,fill=white,inner sep=1pt] {\small $1$};
    \end{tikzpicture}
    \end{center}
\end{enumerate}
\FloatBarrier

% MAT_P4FUNCOE_4FIN_GRA_001.tex
% Exercise ID: MAT_P4FUNCOE_4FIN_GRA_001
% Module: MÓDULO P4 - Funções | Concept: Função Inversa | Type: Determinação Gráfica
% Difficulty: 2/5 (Fácil) | Type: desenvolvimento
% Points: 10 | Time: 10 min
% Tags: inversa, grafico, simetria, funcao_quadratica
% Author: Professor | Date: 2025-11-18
% Status: active
% Description: Dado o gráfico de ramos de funções, desenhar o gráfico da inversa

\exercicio
Na figura está representado o gráfico de uma função $f$ definida em $[0, +\infty[$. Represente, no referencial dado, o gráfico da função inversa $f^{-1}$.

\begin{figure}[ht]
\centering
\begin{tikzpicture}[scale=0.8]
    \begin{axis}[
        axis lines = middle,
        xlabel = $x$,
        ylabel = $y$,
        xmin = -1, xmax = 5,
        ymin = -1, ymax = 5,
        grid = major,
        grid style = {dashed, gray!30},
        width = 8cm,
        height = 6cm,
    ]
    \addplot[domain=-1:5, dashed, gray!70, thin] {x};
    
    \addplot[domain=0:4, samples=100, thick, blue] {x^2/2};
    \addplot[mark=*, mark size=2pt, blue] coordinates {(0,0)};
    \addplot[mark=*, mark size=2pt, blue] coordinates {(2,2)};
    \node[anchor=south west] at (axis cs:2,2.2) {$(2,2)$};
    \node[anchor=north east] at (axis cs:0.2,0.2) {$(0,0)$};
    \end{axis}
\end{tikzpicture}
\end{figure}

\bigskip

\exercicio
Na figura está representado o gráfico de uma função $g$. Represente, no referencial dado, o gráfico da função inversa $g^{-1}$.

\begin{figure}[H]
\centering
\begin{tikzpicture}[scale=0.8]
    \begin{axis}[
        axis lines = middle,
        xlabel = $x$,
        ylabel = $y$,
        xmin = -1, xmax = 6,
        ymin = -1, ymax = 6,
        grid = major,
        grid style = {dashed, gray!30},
        width = 8cm,
        height = 6cm,
    ]
    \addplot[domain=-1:6, dashed, gray!70, thin] {x};
    
    \addplot[domain=-1:5, thick, red] {2*x-1};
    \addplot[mark=*, mark size=2pt, red] coordinates {(1,1)};
    \addplot[mark=*, mark size=2pt, red] coordinates {(3,3)};
    \node[anchor=south west] at (axis cs:3,3.2) {$(3,3)$};
    \node[anchor=north east] at (axis cs:1.15,1.1) {$(1,1)$};
    \end{axis}
\end{tikzpicture}
\end{figure}
\FloatBarrier

% MAT_P4FUNCOE_4FIN_GRA_002.tex
% Exercise ID: MAT_P4FUNCOE_4FIN_GRA_002
% Exercise ID: MAT_P4FUNCOE_4FIN_GRA_002
% Module: MÓDULO P4 - Funções | Concept: Função Inversa | Type: Determinação Gráfica
% Difficulty: 2/5 (Fácil) | Type: desenvolvimento
% Points: 10 | Time: 10 min
% Tags: inversa, grafico, simetria, funcao_quadratica
% Author: Professor | Date: 2025-11-18
% Status: active
% Description: Dado o gráfico de ramos de funções, desenhar o gráfico da inversa

\exercicio
Na figura está representado o gráfico de uma função $f$ definida em $[1, +\infty[$. Represente, no referencial dado, o gráfico da função inversa $f^{-1}$.

\begin{figure}[H]
\centering
\begin{tikzpicture}[scale=0.8]
    \begin{axis}[
        axis lines = middle,
        xlabel = $x$,
        ylabel = $y$,
        xmin = -1, xmax = 5,
        ymin = -1, ymax = 5,
        grid = major,
        grid style = {dashed, gray!30},
        width = 8cm,
        height = 6cm,
    ]
    \addplot[domain=-1:5, dashed, gray!70, thin] {x};
    \addplot[domain=0:4, samples=100, thick, blue] {x^2/2};
    \addplot[mark=*, mark size=2pt, blue] coordinates {(0,0)};
    \addplot[mark=*, mark size=2pt, blue] coordinates {(2,2)};
    \node[anchor=south west] at (axis cs:2,2.2) {$(2,2)$};
    \node[anchor=north east] at (axis cs:0.2,0.2) {$(0,0)$};
    \end{axis}
\end{tikzpicture}
\end{figure}

\bigskip

\exercicio
Na figura está representado o gráfico de uma função $g$ definida em $[2, +\infty[$. Represente, no referencial dado, o gráfico da função inversa $g^{-1}$.

\begin{figure}[H]
\centering
\begin{tikzpicture}[scale=0.8]
    \begin{axis}[
        axis lines = middle,
        xlabel = $x$,
        ylabel = $y$,
        xmin = -1, xmax = 6,
        ymin = -1, ymax = 6,
        grid = major,
        grid style = {dashed, gray!30},
        width = 8cm,
        height = 6cm,
    ]
    \addplot[domain=-1:6, dashed, gray!70, thin] {x};
    \addplot[domain=0:5, thick, red] {2*x-1};
    \addplot[mark=*, mark size=2pt, red] coordinates {(2,3)};
    \addplot[mark=*, mark size=2pt, red] coordinates {(3,5)};
    \node[anchor=south west] at (axis cs:3,5) {$(3,5)$};
    \node[anchor=north east] at (axis cs:2,3) {$(2,3)$};
    \end{axis}
\end{tikzpicture}
\end{figure}
\FloatBarrier

% MAT_P4FUNCOE_4FIN_GRA_003.tex
% Exercise ID: MAT_P4FUNCOE_4FIN_GRA_003
% Exercise ID: MAT_P4FUNCOE_4FIN_GRA_003
% Module: MÓDULO P4 - Funções | Concept: Função Inversa | Type: Determinação Gráfica
% Difficulty: 2/5 (Fácil) | Type: desenvolvimento
% Points: 10 | Time: 10 min
% Tags: inversa, grafico, simetria, funcao_quadratica
% Author: Professor | Date: 2025-11-18
% Status: active
% Description: Dado o gráfico de ramos de funções, desenhar o gráfico da inversa

\exercicio
Na figura está representado o gráfico de uma função $f$ definida em $[0, +\infty[$. Represente, no referencial dado, o gráfico da função inversa $f^{-1}$.

\begin{figure}[ht]
\centering
\begin{tikzpicture}[scale=0.8]
    \begin{axis}[
        axis lines = middle,
        xlabel = $x$,
        ylabel = $y$,
        xmin = -1, xmax = 5,
        ymin = -1, ymax = 5,
        grid = major,
        grid style = {dashed, gray!30},
        width = 8cm,
        height = 6cm,
    ]
    \addplot[domain=-1:5, dashed, gray!70, thin] {x};
    
    \addplot[domain=0:4, samples=100, thick, blue] {sqrt(2*x)};
    \addplot[mark=*, mark size=2pt, blue] coordinates {(0,0)};
    \addplot[mark=*, mark size=2pt, blue] coordinates {(2,2)};
    \node[anchor=south west] at (axis cs:2,2.2) {$(2,2)$};
    \node[anchor=north east] at (axis cs:0.2,0.2) {$(0,0)$};
    \end{axis}
\end{tikzpicture}
\end{figure}

\bigskip

\exercicio
Na figura está representado o gráfico de uma função $g$ definida em $[-1, 3]$. Represente, no referencial dado, o gráfico da função inversa $g^{-1}$.

\begin{figure}[H]
\centering
\begin{tikzpicture}[scale=0.8]
    \begin{axis}[
        axis lines = middle,
        xlabel = $x$,
        ylabel = $y$,
        xmin = -2, xmax = 4,
        ymin = -2, ymax = 4,
        grid = major,
        grid style = {dashed, gray!30},
        width = 8cm,
        height = 6cm,
    ]
    \addplot[domain=-2:4, dashed, gray!70, thin] {x};
    
    \addplot[domain=-1:3, thick, red] {-0.5*x+1.5};
    \addplot[mark=*, mark size=2pt, red] coordinates {(-1,2)};
    \addplot[mark=*, mark size=2pt, red] coordinates {(3,0)};
    \node[anchor=south east] at (axis cs:-1,2) {$(-1,2)$};
    \node[anchor=north west] at (axis cs:3,0) {$(3,0)$};
    \end{axis}
\end{tikzpicture}
\end{figure}
\FloatBarrier

% MAT_P4FUNCOE_4FIN_GRA_004.tex
% Exercise ID: MAT_P4FUNCOE_4FIN_GRA_004
% Exercise ID: MAT_P4FUNCOE_4FIN_GRA_004
% Module: MÓDULO P4 - Funções | Concept: Função Inversa | Type: Determinação Gráfica
% Difficulty: 2/5 (Fácil) | Type: desenvolvimento
% Points: 10 | Time: 10 min
% Tags: inversa, grafico, simetria, funcao_quadratica
% Author: Professor | Date: 2025-11-18
% Status: active
% Description: Dado o gráfico de ramos de funções, desenhar o gráfico da inversa

\exercicio
Na figura está representado o gráfico de uma função $f$ definida em $[1, 4]$. Represente, no referencial dado, o gráfico da função inversa $f^{-1}$.

\begin{figure}[ht]
\centering
\begin{tikzpicture}[scale=0.8]
    \begin{axis}[
        axis lines = middle,
        xlabel = $x$,
        ylabel = $y$,
        xmin = -1, xmax = 5,
        ymin = -1, ymax = 5,
        grid = major,
        grid style = {dashed, gray!30},
        width = 8cm,
        height = 6cm,
    ]
    \addplot[domain=-1:5, dashed, gray!70, thin] {x};
    
    \addplot[domain=1:4, samples=100, thick, blue] {(x-1)^2 + 1};
    \addplot[mark=*, mark size=2pt, blue] coordinates {(1,1)};
    \addplot[mark=*, mark size=2pt, blue] coordinates {(4,10/2)};
    \node[anchor=west] at (axis cs:4,4.2) {$(4,4)$};
    \node[anchor=north east] at (axis cs:1,0.8) {$(1,1)$};
    \end{axis}
\end{tikzpicture}
\end{figure}

\bigskip

\exercicio
Na figura está representado o gráfico de uma função $g$ definida em $]0, 4]$. Represente, no referencial dado, o gráfico da função inversa $g^{-1}$.

\begin{figure}[H]
\centering
\begin{tikzpicture}[scale=0.8]
    \begin{axis}[
        axis lines = middle,
        xlabel = $x$,
        ylabel = $y$,
        xmin = -1, xmax = 5,
        ymin = -1, ymax = 5,
        grid = major,
        grid style = {dashed, gray!30},
        width = 8cm,
        height = 6cm,
    ]
    \addplot[domain=-1:5, dashed, gray!70, thin] {x};
    
    \addplot[domain=0.25:4, samples=100, thick, red] {1/x + 0.75};
    \addplot[mark=o, mark size=2pt, red] coordinates {(0.25,4.75)};
    \addplot[mark=*, mark size=2pt, red] coordinates {(4,1)};
    \node[anchor=south] at (axis cs:4,1.2) {$(4,1)$};
    \end{axis}
\end{tikzpicture}
\end{figure}
\FloatBarrier

% MAT_P4FUNCOE_4FIN_GRA_005.tex
% Exercise ID: MAT_P4FUNCOE_4FIN_GRA_005
% Exercise ID: MAT_P4FUNCOE_4FIN_GRA_005
% Module: MÓDULO P4 - Funções | Concept: Função Inversa | Type: Determinação Gráfica
% Difficulty: 2/5 (Fácil) | Type: desenvolvimento
% Points: 10 | Time: 10 min
% Tags: inversa, grafico, simetria, funcao_quadratica
% Author: Professor | Date: 2025-11-18
% Status: active
% Description: Dado o gráfico de ramos de funções, desenhar o gráfico da inversa

\exercicio
Na figura está representado o gráfico de uma função $f$ definida em $[-2, 2]$. Represente, no referencial dado, o gráfico da função inversa $f^{-1}$.

\begin{figure}[ht]
\centering
\begin{tikzpicture}[scale=0.8]
    \begin{axis}[
        axis lines = middle,
        xlabel = $x$,
        ylabel = $y$,
        xmin = -3, xmax = 3,
        ymin = -3, ymax = 3,
        grid = major,
        grid style = {dashed, gray!30},
        width = 8cm,
        height = 6cm,
    ]
    \addplot[domain=-3:3, dashed, gray!70, thin] {x};
    
    \addplot[domain=-2:2, samples=100, thick, blue] {x^3/4};
    \addplot[mark=*, mark size=2pt, blue] coordinates {(-2,-2)};
    \addplot[mark=*, mark size=2pt, blue] coordinates {(2,2)};
    \node[anchor=north east] at (axis cs:-2,-2) {$(-2,-2)$};
    \node[anchor=south west] at (axis cs:2,2) {$(2,2)$};
    \end{axis}
\end{tikzpicture}
\end{figure}

\bigskip

\exercicio
Na figura está representado o gráfico de uma função $g$ definida em $[0, 3]$. Represente, no referencial dado, o gráfico da função inversa $g^{-1}$.

\begin{figure}[H]
\centering
\begin{tikzpicture}[scale=0.8]
    \begin{axis}[
        axis lines = middle,
        xlabel = $x$,
        ylabel = $y$,
        xmin = -1, xmax = 4,
        ymin = -1, ymax = 4,
        grid = major,
        grid style = {dashed, gray!30},
        width = 8cm,
        height = 6cm,
    ]
    \addplot[domain=-1:4, dashed, gray!70, thin] {x};
    
    \addplot[domain=0:3, thick, red] {3*x/2};
    \addplot[mark=*, mark size=2pt, red] coordinates {(0,0)};
    \addplot[mark=*, mark size=2pt, red] coordinates {(2,3)};
    \node[anchor=south west] at (axis cs:2,3.1) {$(2,3)$};
    \node[anchor=north east] at (axis cs:0.2,0.2) {$(0,0)$};
    \end{axis}
\end{tikzpicture}
\end{figure}
\FloatBarrier

% MAT_P4FUNCOE_4FIN_TRH_001.tex
% Exercise ID: MAT_P4FUNCOE_4FIN_TRH_001
% Module: MÓDULO P4 - Funções | Concept: Função Inversa | Type: Teste da Reta Horizontal
% Difficulty: 2/5 (Fácil) | Type: desenvolvimento
% Points: 10 | Time: 10 min
% Tags: inversa, injetividade, teste_reta_horizontal, grafico
% Author: Professor | Date: 2025-11-18
% Status: active
% Description: Determinar quais funções são invertíveis usando teste da reta horizontal

\exercicio
Considere as funções representadas nas figuras seguintes:

\begin{figure}[H]
\centering
\begin{minipage}{0.45\textwidth}
\centering
\begin{tikzpicture}[scale=0.8]
    \begin{axis}[
        axis lines = middle,
        xlabel = $x$,
        ylabel = $y$,
        xmin = -3, xmax = 3,
        ymin = -1, ymax = 4,
        grid = major,
        grid style = {dashed, gray!30},
        width = 8cm,
        height = 6cm,
        title = {Função A},
        title style = {font=\bfseries},
    ]
    \addplot[domain=-2.5:2.5, samples=100, thick, red] {x^2};
    \end{axis}
\end{tikzpicture}
\end{minipage}
\hfill
\begin{minipage}{0.45\textwidth}
\centering
\begin{tikzpicture}[scale=0.8]
    \begin{axis}[
        axis lines = middle,
        xlabel = $x$,
        ylabel = $y$,
        xmin = -1, xmax = 4,
        ymin = -1, ymax = 4,
        grid = major,
        grid style = {dashed, gray!30},
        width = 8cm,
        height = 6cm,
        title = {Função B},
        title style = {font=\bfseries},
    ]
    \addplot[domain=0:3, samples=100, thick, blue] {x};
    \addplot[mark=*, mark size=2pt, blue] coordinates {(0,0)};
    \addplot[mark=*, mark size=2pt, blue] coordinates {(3,3)};
    \end{axis}
\end{tikzpicture}
\end{minipage}
\end{figure}

Quais das duas funções são invertíveis (isto é, cuja inversa também é uma função)? Justifique usando o teste da reta horizontal.
\FloatBarrier

% MAT_P4FUNCOE_4FIN_TRH_002.tex
% Exercise ID: MAT_P4FUNCOE_4FIN_TRH_002
% Exercise ID: MAT_P4FUNCOE_4FIN_TRH_002
% Module: MÓDULO P4 - Funções | Concept: Função Inversa | Type: Teste da Reta Horizontal
% Difficulty: 2/5 (Fácil) | Type: desenvolvimento
% Points: 10 | Time: 10 min
% Tags: inversa, injetividade, teste_reta_horizontal, grafico
% Author: Professor | Date: 2025-11-18
% Status: active
% Description: Determinar quais funções são invertíveis usando teste da reta horizontal

\exercicio
Considere as funções representadas nas figuras seguintes:

\begin{figure}[H]
\centering
\begin{minipage}{0.45\textwidth}
\centering
\begin{tikzpicture}[scale=0.8]
    \begin{axis}[
        axis lines = middle,
        xlabel = $x$,
        ylabel = $y$,
        xmin = -3, xmax = 3,
        ymin = -2, ymax = 4,
        grid = major,
        grid style = {dashed, gray!30},
        width = 8cm,
        height = 6cm,
        title = {Função C},
        title style = {font=\bfseries},
    ]
    \addplot[domain=-2:2, samples=100, thick, red] {x^2-4};
    \end{axis}
\end{tikzpicture}
\end{minipage}
\hfill
\begin{minipage}{0.45\textwidth}
\centering
\begin{tikzpicture}[scale=0.8]
    \begin{axis}[
        axis lines = middle,
        xlabel = $x$,
        ylabel = $y$,
        xmin = -1, xmax = 4,
        ymin = -1, ymax = 5,
        grid = major,
        grid style = {dashed, gray!30},
        width = 8cm,
        height = 6cm,
        title = {Função D},
        title style = {font=\bfseries},
    ]
    \addplot[domain=0:3.5, samples=100, thick, blue] {sqrt(x)};
    \addplot[mark=*, mark size=2pt, blue] coordinates {(0,0)};
    \end{axis}
\end{tikzpicture}
\end{minipage}
\end{figure}



Quais das duas funções são invertíveis (isto é, cuja inversa também é uma função)? Justifique usando o teste da reta horizontal.
\FloatBarrier

% MAT_P4FUNCOE_4FIN_TRH_003.tex
% Exercise ID: MAT_P4FUNCOE_4FIN_TRH_003
% Exercise ID: MAT_P4FUNCOE_4FIN_TRH_003
% Module: MÓDULO P4 - Funções | Concept: Função Inversa | Type: Teste da Reta Horizontal
% Difficulty: 2/5 (Fácil) | Type: desenvolvimento
% Points: 10 | Time: 10 min
% Tags: inversa, injetividade, teste_reta_horizontal, grafico
% Author: Professor | Date: 2025-11-18
% Status: active
% Description: Determinar quais funções são invertíveis usando teste da reta horizontal

\exercicio
Considere as funções representadas nas figuras seguintes:

\begin{figure}[H]
\centering
\begin{minipage}{0.45\textwidth}
\centering
\begin{tikzpicture}[scale=0.8]
    \begin{axis}[
        axis lines = middle,
        xlabel = $x$,
        ylabel = $y$,
        xmin = -4, xmax = 4,
        ymin = -2, ymax = 2,
        grid = major,
        grid style = {dashed, gray!30},
        width = 8cm,
        height = 6cm,
        title = {Função E},
        title style = {font=\bfseries},
    ]
    \addplot[domain=-3.14:3.14, samples=200, thick, red] {sin(deg(x))};
    \end{axis}
\end{tikzpicture}
\end{minipage}
\hfill
\begin{minipage}{0.45\textwidth}
\centering
\begin{tikzpicture}[scale=0.8]
    \begin{axis}[
        axis lines = middle,
        xlabel = $x$,
        ylabel = $y$,
        xmin = -1, xmax = 4,
        ymin = -1, ymax = 4,
        grid = major,
        grid style = {dashed, gray!30},
        width = 8cm,
        height = 6cm,
        title = {Função F},
        title style = {font=\bfseries},
    ]
    \addplot[domain=0:3, thick, blue] {2*x};
    \addplot[mark=*, mark size=2pt, blue] coordinates {(0,0)};
    \addplot[mark=*, mark size=2pt, blue] coordinates {(3,6)};
    \end{axis}
\end{tikzpicture}
\end{minipage}
\end{figure}



Quais das duas funções são invertíveis (isto é, cuja inversa também é uma função)? Justifique usando o teste da reta horizontal.
\FloatBarrier

% MAT_P4FUNCOE_4FIN_TRH_004.tex
% Exercise ID: MAT_P4FUNCOE_4FIN_TRH_004
% Exercise ID: MAT_P4FUNCOE_4FIN_TRH_004
% Module: MÓDULO P4 - Funções | Concept: Função Inversa | Type: Teste da Reta Horizontal
% Difficulty: 2/5 (Fácil) | Type: desenvolvimento
% Points: 10 | Time: 10 min
% Tags: inversa, injetividade, teste_reta_horizontal, grafico
% Author: Professor | Date: 2025-11-18
% Status: active
% Description: Determinar quais funções são invertíveis usando teste da reta horizontal

\exercicio
Considere as funções representadas nas figuras seguintes:

\begin{figure}[H]
\centering
\begin{minipage}{0.45\textwidth}
\centering
\begin{tikzpicture}[scale=0.8]
    \begin{axis}[
        axis lines = middle,
        xlabel = $x$,
        ylabel = $y$,
        xmin = -1, xmax = 5,
        ymin = -1, ymax = 5,
        grid = major,
        grid style = {dashed, gray!30},
        width = 8cm,
        height = 6cm,
        title = {Função G},
        title style = {font=\bfseries},
    ]
    \addplot[domain=0:4, samples=100, thick, red] {abs(x-2)+1};
    \end{axis}
\end{tikzpicture}
\end{minipage}
\hfill
\begin{minipage}{0.45\textwidth}
\centering
\begin{tikzpicture}[scale=0.8]
    \begin{axis}[
        axis lines = middle,
        xlabel = $x$,
        ylabel = $y$,
        xmin = -1, xmax = 4,
        ymin = -3, ymax = 1,
        grid = major,
        grid style = {dashed, gray!30},
        width = 8cm,
        height = 6cm,
        title = {Função H},
        title style = {font=\bfseries},
    ]
    \addplot[domain=-0.5:3.5, samples=100, thick, blue] {x^(3)/20};
    \end{axis}
\end{tikzpicture}
\end{minipage}
\end{figure}



Quais das duas funções são invertíveis (isto é, cuja inversa também é uma função)? Justifique usando o teste da reta horizontal.
\FloatBarrier

% MAT_P4FUNCOE_4FIN_TRH_005.tex
% Exercise ID: MAT_P4FUNCOE_4FIN_TRH_005
% Exercise ID: MAT_P4FUNCOE_4FIN_TRH_005
% Module: MÓDULO P4 - Funções | Concept: Função Inversa | Type: Teste da Reta Horizontal
% Difficulty: 2/5 (Fácil) | Type: desenvolvimento
% Points: 10 | Time: 10 min
% Tags: inversa, injetividade, teste_reta_horizontal, grafico
% Author: Professor | Date: 2025-11-18
% Status: active
% Description: Determinar quais funções são invertíveis usando teste da reta horizontal

\exercicio
Considere as funções representadas nas figuras seguintes:

\begin{figure}[H]
\centering
\begin{minipage}{0.45\textwidth}
\centering
\begin{tikzpicture}[scale=0.8]
    \begin{axis}[
        axis lines = middle,
        xlabel = $x$,
        ylabel = $y$,
        xmin = -1, xmax = 5,
        ymin = -1, ymax = 5,
        grid = major,
        grid style = {dashed, gray!30},
        width = 8cm,
        height = 6cm,
        title = {Função I},
        title style = {font=\bfseries},
    ]
    \addplot[domain=0:4, samples=100, thick, red] {x^2};
    \addplot[mark=*, mark size=2pt, red] coordinates {(0,0)};
    \end{axis}
\end{tikzpicture}
\end{minipage}
\hfill
\begin{minipage}{0.45\textwidth}
\centering
\begin{tikzpicture}[scale=0.8]
    \begin{axis}[
        axis lines = middle,
        xlabel = $x$,
        ylabel = $y$,
        xmin = -1, xmax = 5,
        ymin = -3, ymax = 1,
        grid = major,
        grid style = {dashed, gray!30},
        width = 8cm,
        height = 6cm,
        title = {Função J},
        title style = {font=\bfseries},
    ]
    \addplot[domain=0:4, samples=100, thick, blue] {-sqrt(x)};
    \addplot[mark=*, mark size=2pt, blue] coordinates {(0,0)};
    \end{axis}
\end{tikzpicture}
\end{minipage}
\end{figure}



Quais das duas funções são invertíveis (isto é, cuja inversa também é uma função)? Justifique usando o teste da reta horizontal.
\FloatBarrier



\end{document}

