% Template para geração automática de sebentas
% Gerado automaticamente - NÃO EDITAR MANUALMENTE
\documentclass[12pt,a4paper]{article}

% Encoding e idioma
\usepackage[utf8]{inputenc}
\usepackage[T1]{fontenc}
\usepackage[portuguese]{babel}

% Matemática
\usepackage{amsmath}
\usepackage{amssymb}
\usepackage{amsthm}
\usepackage{mathtools}

% Gráficos e figuras
\usepackage{graphicx}
\usepackage{tikz}
\usetikzlibrary{calc,patterns,angles,quotes}
\usepackage{pgfplots}
\pgfplotsset{compat=1.18}
% Force float barriers when needed to keep figures right after exercises
\usepackage{placeins}
\usepackage{float}  % Para figure[H] - força figuras exatamente onde definidas

% Layout e formatação
\usepackage{geometry}
\geometry{a4paper,margin=2.5cm,top=3cm,bottom=3cm}
\usepackage{fancyhdr}
\usepackage{enumitem}
\usepackage{multicol}
\usepackage{booktabs}

% Hyperlinks e referências
\usepackage{hyperref}
\hypersetup{
    colorlinks=true,
    linkcolor=blue,
    urlcolor=blue,
    citecolor=blue
}

% Sistema de exercícios - macros personalizadas
\newcounter{exerciciocount}
\newcounter{subexerciciocount}
\newcounter{optioncount}

\newcommand{\exercicio}[1][]{% 
    \par\vspace{1.5em}%
    \refstepcounter{exerciciocount}%
    \setcounter{subexerciciocount}{0}%
    \setcounter{optioncount}{0}%
    \noindent\textbf{Exercício~\theexerciciocount.}%
    \ifx&#1&%
        % Argumento vazio - o conteúdo vem depois
        \par\vspace{0.3em}%
    \else%
        % Argumento fornecido - incluir inline
        \ #1\par\vspace{0.5em}%
    \fi%
}

\newcommand{\subexercicio}[1]{%
    \par\vspace{0.8em}%
    \refstepcounter{subexerciciocount}%
    \noindent\textbf{\theexerciciocount.\thesubexerciciocount.} #1\par\vspace{0.3em}%
}

\newcommand{\option}[1]{%
    \par
    \refstepcounter{optioncount}%
    \noindent(\alph{optioncount}) #1%
}

% Cabeçalho completo do teste dentro de uma caixa simples
\newcommand{\espacoAluno}{%
    \vspace{0.5cm}
    \fbox{%
        \parbox{\textwidth}{%
            \noindent\textbf{Nome do Aluno:} \underline{\hspace{7cm}} \textbf{Turma:} \underline{\hspace{1cm}}\\[0.5cm]
            \noindent\textbf{Assinatura do Professor:} \underline{\hspace{3cm}} \hfill \textbf{Nota:} \underline{\hspace{2cm}}\\[0.5cm]
            \noindent\textbf{Assinatura do Encarregado de Educação:} \underline{\hspace{3cm}}
        }%
    }
    \vspace{1cm}
}

% Campos para respostas
\newcommand{\campo}[1][2.0cm]{\makebox[#1]{\hrulefill}}
\newcommand{\campoLetra}[1][1.2cm]{\makebox[#1]{\hrulefill}}
\newcommand{\campoCents}[1][3.0cm]{\makebox[#1]{\hrulefill}}% Cabeçalho e rodapé
\pagestyle{fancy}
\fancyhf{}
\fancyhead[L]{MÓDULO P4 - Funções}
\fancyhead[R]{Função Inversa}
\fancyfoot[C]{\thepage}

% Metadados do documento
\title{}
\author{}
\date{}

\begin{document}

% Remover título, autor e data - apenas conteúdo
\thispagestyle{fancy}

\section*{Função Inversa}

\textit{
Conceito de função inversa, condições de existência (injetividade), determinação analítica e gráfica, simetria e propriedades.
}

\vspace{1em}

\subsection*{Tipos de Exercícios}
\begin{itemize}
  \item \textbf{Determinação Analítica} --- Cálculo da expressão analítica da função inversa através de manipulação algébrica
\end{itemize}

\vspace{1em}

% Exercício 1: MAT_P4FUNCOE_4FX_DAX_001.tex
% Exercise ID: MAT_P4FUNCOE_4FX_DAX_001
% Module: MÓDULO P4 - Funções | Concept: Função Inversa | Type: Determinação Analítica da Função Inversa
% Difficulty: 2/5 (Fácil) | Format: standard
% Tags: inversa, expressao_analitica, algebra, calculo_analitico, injetividade, sobrejetividade, simetria, resolucao_equacao
% Author: Test Agent | Date: 2025-11-26
% Status: active

\exercicio{Determine a função inversa de f(x) = 2x + 3.}
\FloatBarrier

% Exercício 2: main.tex
% Exercise ID: MAT_P4FUNCOE_4FIN_ANA_001
% meta:
% id: MAT_P4FUNCOE_4FIN_ANA_001
% title: "Determinação Analítica da Função Inversa"
% difficulty: 2
% tags: funcao_inversa, determinacao_analitica
% author: Generated Example
% has_subvariants: true

\section{Determinação Analítica da Função Inversa}

\exercicio{
Determina analiticamente a função inversa das seguintes expressões:
}

\begin{enumerate}[label=\alph*)]

\item % Exercise ID: MAT_P4FUNCOE_4FIN_ANA_001
% Sub-variant 1 for MAT_P4FUNCOE_4FIN_ANA_001
% Function: f(x) = x + 4

$f(x) = x + 4$
\item % Exercise ID: MAT_P4FUNCOE_4FIN_ANA_001
% Sub-variant 2 for MAT_P4FUNCOE_4FIN_ANA_001
% Function: f(x) = 2x - 1

$f(x) = 2x - 1$
\item % Exercise ID: MAT_P4FUNCOE_4FIN_ANA_001
% Sub-variant 3 for MAT_P4FUNCOE_4FIN_ANA_001
% Function: f(x) = \frac{1}{x-1}

$f(x) = \frac{1}{x-1}$
\n\item % Exercise ID: MAT_P4FUNCOE_4FIN_ANA_001
% Exercise ID: MAT_P4FUNCOE_4FIN_ANA_001
% Exercise ID: MAT_P4FUNCOE_4FIN_ANA_001
% Exercise ID: MAT_P4FUNCOE_4FIN_ANA_001
% Sub-variant 3 for MAT_P4FUNCOE_4FIN_ANA_001
% Function: f(x) = \frac{x}{2}

$f(x) = \frac{x}{2}$
\n\item % Exercise ID: MAT_P4FUNCOE_4FIN_ANA_001
% Exercise ID: MAT_P4FUNCOE_4FIN_ANA_001
% Exercise ID: MAT_P4FUNCOE_4FIN_ANA_001
% Exercise ID: MAT_P4FUNCOE_4FIN_ANA_001
% Sub-variant 3 for MAT_P4FUNCOE_4FIN_ANA_001
% Function: f(x) = \frac{x}{2}

$f(x) = \frac{x}{2}$
\n\item % Exercise ID: MAT_P4FUNCOE_4FIN_ANA_001
% Exercise ID: MAT_P4FUNCOE_4FIN_ANA_001
% Exercise ID: MAT_P4FUNCOE_4FIN_ANA_001
% Exercise ID: MAT_P4FUNCOE_4FIN_ANA_001
% Exercise ID: MAT_P4FUNCOE_4FIN_ANA_001
% Sub-variant 3 for MAT_P4FUNCOE_4FIN_ANA_001
% Function: f(x) = 2x-9

$f(x) = x-9$
\n\item % Exercise ID: MAT_P4FUNCOE_4FIN_ANA_001
% Sub-variant 8 for MAT_P4FUNCOE_4FIN_ANA_001
% Date: 2025-11-24
% Function: f(x) = x-6

$f(x) = x-6$
\n\item % Exercise ID: MAT_P4FUNCOE_4FIN_ANA_001
% Sub-variant 8 for MAT_P4FUNCOE_4FIN_ANA_001
% Date: 2025-11-24
% Function: f(x) = x-6

$f(x) = x-6$
\n\item q% Exercise ID: MAT_P4FUNCOE_4FIN_ANA_001
% Sub-variant 9 for MAT_P4FUNCOE_4FIN_ANA_001
% Date: 2025-11-25
% Function: f(x) = x-7

$f(x) = x-7$
\n\item % Exercise ID: MAT_P4FUNCOE_4FIN_ANA_001
% Sub-variant 10 for MAT_P4FUNCOE_4FIN_ANA_001
% Date: 2025-11-25
% Function: f(x) = 2x - 1

$f(x) = 2x - 10$
\n\item % Exercise ID: MAT_P4FUNCOE_4FIN_ANA_001
% Sub-variant 11 for MAT_P4FUNCOE_4FIN_ANA_001
% Date: 2025-11-25
% Function: f(x) = 2x - 1

$f(x) = 7x - 2$
\end{enumerate}
\FloatBarrier

% Exercício 3: main.tex
% meta:
% id: MAT_P4FUNCOE_4FX_DAX_003
% title: "MÓDULO P4 - Funções - Função Inversa - Determinação Analítica da Função Inversa"
% difficulty: 3
% tags: 
% author: Test Agent
% has_subvariants: true

\section{MÓDULO P4 - Funções - Função Inversa - Determinação Analítica da Função Inversa}

\exercicio{
Determine analiticamente a função inversa das seguintes funções:
}

\begin{enumerate}[label=\alph*)]
\item % Sub-variant 1 for MAT_P4FUNCOE_4FX_DAX_003
% Content: x + 1

x + 1
\item % Sub-variant 2 for MAT_P4FUNCOE_4FX_DAX_003
% Content: 2x - 3

2x - 3
\item % Sub-variant 3 for MAT_P4FUNCOE_4FX_DAX_003
% Content: \\frac{1}{x-2}

\\frac{1}{x-2}
\end{enumerate}
\FloatBarrier

% Exercício 4: MAT_P4FUNCOE_4FX_DAX_002.tex
% Exercise ID: MAT_P4FUNCOE_4FX_DAX_002
% Module: MÓDULO P4 - Funções | Concept: Função Inversa | Type: Determinação Analítica da Função Inversa
% Difficulty: 2/5 (Fácil) | Format: standard
% Tags: algebra, simetria, resolucao_equacao, expressao_analitica, injetividade, inversa, sobrejetividade, calculo_analitico
% Author: Test Agent | Date: 2025-11-26
% Status: active

\exercicio{Determine a função inversa de f(x) = 2x + 3.}
\FloatBarrier

% Exercício 5: MAT_P4FUNCOE_4FX_DAX_005.tex
% Exercise ID: MAT_P4FUNCOE_4FX_DAX_005
% Created: 2025-11-26
% Difficulty: 2/5

\exercicio{Determine a função inversa de f(x) = (x-1)/(x+1).}
\FloatBarrier

% Exercício 6: MAT_P4FUNCOE_4FX_DAX_003.tex
% Exercise ID: MAT_P4FUNCOE_4FX_DAX_003
% Module: MÓDULO P4 - Funções | Concept: Função Inversa | Type: Determinação Analítica da Função Inversa
% Difficulty: 3/5 (Médio) | Format: standard
% Tags: resolucao_equacao, expressao_analitica, algebra, injetividade, simetria, sobrejetividade, calculo_analitico, inversa
% Author: Professor | Date: 2025-11-26
% Status: active

\exercicio{Teste de exercício}
\FloatBarrier

% Exercício 7: MAT_P4FUNCOE_FUNC_DA_001.tex
% Exercise ID: MAT_P4FUNCOE_FUNC_DA_001
% Module: P4_funcoes | Concept: 4-funcao_inversa
% Type: determinacao_analitica | Difficulty: 2/5
% Tags: funcao_inversa, determinacao analitica, funcoes, inversa
% Author: Professor | Date: 2025-11-26

\exercicio{
Determine a função inversa de $f(x) = 2x + 3$.
}

\subexercicio{Verifique que $f(f^{-1}
\FloatBarrier

% Exercício 8: MAT_P4FUNCOE_FUNC_DA_002.tex
% Exercise ID: MAT_P4FUNCOE_FUNC_DA_002
% Module: P4_funcoes | Concept: 4-funcao_inversa
% Type: determinacao_analitica | Difficulty: 2/5
% Tags: funcao_inversa, determinacao analitica, funcoes, inversa
% Author: Professor | Date: 2025-11-26

\exercicio{
Determine a função inversa de $f(x) = 2x + 3$.
}

\subexercicio{Verifique que $f(f^{-1}
\FloatBarrier

% Exercício 9: MAT_P4FUNCOE_4FX_DAX_004.tex
% Exercise ID: MAT_P4FUNCOE_4FX_DAX_004
% Created: 2025-11-26
% Difficulty: 2/5

\exercicio{Determine a função inversa de f(x) = 3x - 5.}
\FloatBarrier

% Exercício 10: MAT_P4FUNCOE_FUNC_DA_003.tex
% Exercise ID: MAT_P4FUNCOE_FUNC_DA_003
% Module: P4_funcoes | Concept: 4-funcao_inversa
% Type: determinacao_analitica | Difficulty: 2/5
% Tags: funcao_inversa, determinacao analitica, funcoes, inversa
% Author: Professor | Date: 2025-11-26

\exercicio{
Determine a função inversa de $f(x) = 2x + 3$.
}

\subexercicio{Verifique que $f(f^{-1}
\FloatBarrier


\end{document}
