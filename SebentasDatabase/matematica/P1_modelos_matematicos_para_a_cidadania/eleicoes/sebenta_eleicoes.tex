% Minimal sebenta template generated automatically
\documentclass[11pt,a4paper]{article}
\usepackage[utf8]{inputenc}
\usepackage[T1]{fontenc}
\usepackage{lmodern}
\usepackage{geometry}
\usepackage{fancyhdr}
\usepackage{hyperref}
\usepackage{graphicx}
\usepackage{float}
\usepackage{placeins}
\usepackage{bookmark}
\usepackage{booktabs}
\usepackage{amsmath,amssymb}
\usepackage{csquotes}
\usepackage{enumitem}
\usepackage{tikz}
\IfFileExists{pgfplots.sty}{\usepackage{pgfplots}\pgfplotsset{compat=1.17}}{}

\geometry{margin=2.5cm}

% Try to include project-specific style macros (containing \exercicio, \subexercicio, etc.)
% Try multiple relative locations to be robust across different generated output paths
\IfFileExists{../../../../Teste_modelo/config/style.tex}{% Sistema de exercícios com contadores automáticos
\newcounter{exerciciocount}          % Contador principal dos exercícios
\newcounter{subexerciciocount}       % Contador dos subexercícios
\newcounter{optioncount}             % Contador das opções

% Control whether the macro prints the automatic "Exercício N." heading.
% Default: show the heading. Call \showexerciciotitlefalse to suppress.
\newif\ifshowexerciciotitle
\showexerciciotitletrue

% Macro para exercício principal
\newcommand{\exercicio}[1]{%
        \par\vspace{1.5em}% Espaçamento antes
        \refstepcounter{exerciciocount}% Incrementa contador principal
        \setcounter{subexerciciocount}{0}% Reseta contador de subexercícios
        \setcounter{optioncount}{0}% Reseta contador de opções
        % Only print the automatic heading if the flag is true
        \ifshowexerciciotitle
            \noindent\textbf{Exercício~\theexerciciocount.}\space #1\par\vspace{0.5em}%
        \else
            % When suppressed, just print the content without the heading
            #1\par\vspace{0.5em}%
        \fi
}

% Macro para subexercício
\newcommand{\subexercicio}[1]{%
    \par\vspace{0.8em}% Espaçamento menor para subexercícios
    \refstepcounter{subexerciciocount}% Incrementa contador de subexercícios
    \noindent\textbf{\theexerciciocount.\thesubexerciciocount.} #1\par\vspace{0.3em}%
}

% Macro para opção
\newcommand{\option}[1]{%
    \par
    \refstepcounter{optioncount}%
    \noindent(\alph{optioncount}) #1%
}

% Título e informações do exame
\title{1ª Questão de aula do Módulo A10: Otimização}
\author{EPRALIMA - Escola Profissional Alto Lima}

\date{}

% Cabeçalho completo do teste dentro de uma caixa simples
\newcommand{\espacoAluno}{%
    \vspace{0.5cm}
    \fbox{%
        \parbox{\textwidth}{%
            \noindent\textbf{Nome do Aluno:} \underline{\hspace{7cm}} \textbf{Turma:} \underline{\hspace{1cm}}\\[0.5cm]
            \noindent\textbf{Assinatura do Professor:} \underline{\hspace{3cm}} \hfill \textbf{Nota:} \underline{\hspace{2cm}}\\[0.5cm]
            \noindent\textbf{Assinatura do Encarregado de Educação:} \underline{\hspace{3cm}}
        }%
    }
    \vspace{1cm}
}}{%
  \IfFileExists{../../../Teste_modelo/config/style.tex}{% Sistema de exercícios com contadores automáticos
\newcounter{exerciciocount}          % Contador principal dos exercícios
\newcounter{subexerciciocount}       % Contador dos subexercícios
\newcounter{optioncount}             % Contador das opções

% Control whether the macro prints the automatic "Exercício N." heading.
% Default: show the heading. Call \showexerciciotitlefalse to suppress.
\newif\ifshowexerciciotitle
\showexerciciotitletrue

% Macro para exercício principal
\newcommand{\exercicio}[1]{%
        \par\vspace{1.5em}% Espaçamento antes
        \refstepcounter{exerciciocount}% Incrementa contador principal
        \setcounter{subexerciciocount}{0}% Reseta contador de subexercícios
        \setcounter{optioncount}{0}% Reseta contador de opções
        % Only print the automatic heading if the flag is true
        \ifshowexerciciotitle
            \noindent\textbf{Exercício~\theexerciciocount.}\space #1\par\vspace{0.5em}%
        \else
            % When suppressed, just print the content without the heading
            #1\par\vspace{0.5em}%
        \fi
}

% Macro para subexercício
\newcommand{\subexercicio}[1]{%
    \par\vspace{0.8em}% Espaçamento menor para subexercícios
    \refstepcounter{subexerciciocount}% Incrementa contador de subexercícios
    \noindent\textbf{\theexerciciocount.\thesubexerciciocount.} #1\par\vspace{0.3em}%
}

% Macro para opção
\newcommand{\option}[1]{%
    \par
    \refstepcounter{optioncount}%
    \noindent(\alph{optioncount}) #1%
}

% Título e informações do exame
\title{1ª Questão de aula do Módulo A10: Otimização}
\author{EPRALIMA - Escola Profissional Alto Lima}

\date{}

% Cabeçalho completo do teste dentro de uma caixa simples
\newcommand{\espacoAluno}{%
    \vspace{0.5cm}
    \fbox{%
        \parbox{\textwidth}{%
            \noindent\textbf{Nome do Aluno:} \underline{\hspace{7cm}} \textbf{Turma:} \underline{\hspace{1cm}}\\[0.5cm]
            \noindent\textbf{Assinatura do Professor:} \underline{\hspace{3cm}} \hfill \textbf{Nota:} \underline{\hspace{2cm}}\\[0.5cm]
            \noindent\textbf{Assinatura do Encarregado de Educação:} \underline{\hspace{3cm}}
        }%
    }
    \vspace{1cm}
}}{%
    \IfFileExists{../../Teste_modelo/config/style.tex}{% Sistema de exercícios com contadores automáticos
\newcounter{exerciciocount}          % Contador principal dos exercícios
\newcounter{subexerciciocount}       % Contador dos subexercícios
\newcounter{optioncount}             % Contador das opções

% Control whether the macro prints the automatic "Exercício N." heading.
% Default: show the heading. Call \showexerciciotitlefalse to suppress.
\newif\ifshowexerciciotitle
\showexerciciotitletrue

% Macro para exercício principal
\newcommand{\exercicio}[1]{%
        \par\vspace{1.5em}% Espaçamento antes
        \refstepcounter{exerciciocount}% Incrementa contador principal
        \setcounter{subexerciciocount}{0}% Reseta contador de subexercícios
        \setcounter{optioncount}{0}% Reseta contador de opções
        % Only print the automatic heading if the flag is true
        \ifshowexerciciotitle
            \noindent\textbf{Exercício~\theexerciciocount.}\space #1\par\vspace{0.5em}%
        \else
            % When suppressed, just print the content without the heading
            #1\par\vspace{0.5em}%
        \fi
}

% Macro para subexercício
\newcommand{\subexercicio}[1]{%
    \par\vspace{0.8em}% Espaçamento menor para subexercícios
    \refstepcounter{subexerciciocount}% Incrementa contador de subexercícios
    \noindent\textbf{\theexerciciocount.\thesubexerciciocount.} #1\par\vspace{0.3em}%
}

% Macro para opção
\newcommand{\option}[1]{%
    \par
    \refstepcounter{optioncount}%
    \noindent(\alph{optioncount}) #1%
}

% Título e informações do exame
\title{1ª Questão de aula do Módulo A10: Otimização}
\author{EPRALIMA - Escola Profissional Alto Lima}

\date{}

% Cabeçalho completo do teste dentro de uma caixa simples
\newcommand{\espacoAluno}{%
    \vspace{0.5cm}
    \fbox{%
        \parbox{\textwidth}{%
            \noindent\textbf{Nome do Aluno:} \underline{\hspace{7cm}} \textbf{Turma:} \underline{\hspace{1cm}}\\[0.5cm]
            \noindent\textbf{Assinatura do Professor:} \underline{\hspace{3cm}} \hfill \textbf{Nota:} \underline{\hspace{2cm}}\\[0.5cm]
            \noindent\textbf{Assinatura do Encarregado de Educação:} \underline{\hspace{3cm}}
        }%
    }
    \vspace{1cm}
}}{%
      % style.tex not found - proceed without project macros
    }%
  }%
}

% Provide a robust fallback for macros that might be missing in style.tex
% This attempts to include the project style first (multiple relative paths),
% and only if none exist defines minimal counters and macros safely.
\IfFileExists{../../../../Teste_modelo/config/style.tex}{% Sistema de exercícios com contadores automáticos
\newcounter{exerciciocount}          % Contador principal dos exercícios
\newcounter{subexerciciocount}       % Contador dos subexercícios
\newcounter{optioncount}             % Contador das opções

% Control whether the macro prints the automatic "Exercício N." heading.
% Default: show the heading. Call \showexerciciotitlefalse to suppress.
\newif\ifshowexerciciotitle
\showexerciciotitletrue

% Macro para exercício principal
\newcommand{\exercicio}[1]{%
        \par\vspace{1.5em}% Espaçamento antes
        \refstepcounter{exerciciocount}% Incrementa contador principal
        \setcounter{subexerciciocount}{0}% Reseta contador de subexercícios
        \setcounter{optioncount}{0}% Reseta contador de opções
        % Only print the automatic heading if the flag is true
        \ifshowexerciciotitle
            \noindent\textbf{Exercício~\theexerciciocount.}\space #1\par\vspace{0.5em}%
        \else
            % When suppressed, just print the content without the heading
            #1\par\vspace{0.5em}%
        \fi
}

% Macro para subexercício
\newcommand{\subexercicio}[1]{%
    \par\vspace{0.8em}% Espaçamento menor para subexercícios
    \refstepcounter{subexerciciocount}% Incrementa contador de subexercícios
    \noindent\textbf{\theexerciciocount.\thesubexerciciocount.} #1\par\vspace{0.3em}%
}

% Macro para opção
\newcommand{\option}[1]{%
    \par
    \refstepcounter{optioncount}%
    \noindent(\alph{optioncount}) #1%
}

% Título e informações do exame
\title{1ª Questão de aula do Módulo A10: Otimização}
\author{EPRALIMA - Escola Profissional Alto Lima}

\date{}

% Cabeçalho completo do teste dentro de uma caixa simples
\newcommand{\espacoAluno}{%
    \vspace{0.5cm}
    \fbox{%
        \parbox{\textwidth}{%
            \noindent\textbf{Nome do Aluno:} \underline{\hspace{7cm}} \textbf{Turma:} \underline{\hspace{1cm}}\\[0.5cm]
            \noindent\textbf{Assinatura do Professor:} \underline{\hspace{3cm}} \hfill \textbf{Nota:} \underline{\hspace{2cm}}\\[0.5cm]
            \noindent\textbf{Assinatura do Encarregado de Educação:} \underline{\hspace{3cm}}
        }%
    }
    \vspace{1cm}
}}{%
  \IfFileExists{../../../Teste_modelo/config/style.tex}{% Sistema de exercícios com contadores automáticos
\newcounter{exerciciocount}          % Contador principal dos exercícios
\newcounter{subexerciciocount}       % Contador dos subexercícios
\newcounter{optioncount}             % Contador das opções

% Control whether the macro prints the automatic "Exercício N." heading.
% Default: show the heading. Call \showexerciciotitlefalse to suppress.
\newif\ifshowexerciciotitle
\showexerciciotitletrue

% Macro para exercício principal
\newcommand{\exercicio}[1]{%
        \par\vspace{1.5em}% Espaçamento antes
        \refstepcounter{exerciciocount}% Incrementa contador principal
        \setcounter{subexerciciocount}{0}% Reseta contador de subexercícios
        \setcounter{optioncount}{0}% Reseta contador de opções
        % Only print the automatic heading if the flag is true
        \ifshowexerciciotitle
            \noindent\textbf{Exercício~\theexerciciocount.}\space #1\par\vspace{0.5em}%
        \else
            % When suppressed, just print the content without the heading
            #1\par\vspace{0.5em}%
        \fi
}

% Macro para subexercício
\newcommand{\subexercicio}[1]{%
    \par\vspace{0.8em}% Espaçamento menor para subexercícios
    \refstepcounter{subexerciciocount}% Incrementa contador de subexercícios
    \noindent\textbf{\theexerciciocount.\thesubexerciciocount.} #1\par\vspace{0.3em}%
}

% Macro para opção
\newcommand{\option}[1]{%
    \par
    \refstepcounter{optioncount}%
    \noindent(\alph{optioncount}) #1%
}

% Título e informações do exame
\title{1ª Questão de aula do Módulo A10: Otimização}
\author{EPRALIMA - Escola Profissional Alto Lima}

\date{}

% Cabeçalho completo do teste dentro de uma caixa simples
\newcommand{\espacoAluno}{%
    \vspace{0.5cm}
    \fbox{%
        \parbox{\textwidth}{%
            \noindent\textbf{Nome do Aluno:} \underline{\hspace{7cm}} \textbf{Turma:} \underline{\hspace{1cm}}\\[0.5cm]
            \noindent\textbf{Assinatura do Professor:} \underline{\hspace{3cm}} \hfill \textbf{Nota:} \underline{\hspace{2cm}}\\[0.5cm]
            \noindent\textbf{Assinatura do Encarregado de Educação:} \underline{\hspace{3cm}}
        }%
    }
    \vspace{1cm}
}}{%
    \IfFileExists{../../Teste_modelo/config/style.tex}{% Sistema de exercícios com contadores automáticos
\newcounter{exerciciocount}          % Contador principal dos exercícios
\newcounter{subexerciciocount}       % Contador dos subexercícios
\newcounter{optioncount}             % Contador das opções

% Control whether the macro prints the automatic "Exercício N." heading.
% Default: show the heading. Call \showexerciciotitlefalse to suppress.
\newif\ifshowexerciciotitle
\showexerciciotitletrue

% Macro para exercício principal
\newcommand{\exercicio}[1]{%
        \par\vspace{1.5em}% Espaçamento antes
        \refstepcounter{exerciciocount}% Incrementa contador principal
        \setcounter{subexerciciocount}{0}% Reseta contador de subexercícios
        \setcounter{optioncount}{0}% Reseta contador de opções
        % Only print the automatic heading if the flag is true
        \ifshowexerciciotitle
            \noindent\textbf{Exercício~\theexerciciocount.}\space #1\par\vspace{0.5em}%
        \else
            % When suppressed, just print the content without the heading
            #1\par\vspace{0.5em}%
        \fi
}

% Macro para subexercício
\newcommand{\subexercicio}[1]{%
    \par\vspace{0.8em}% Espaçamento menor para subexercícios
    \refstepcounter{subexerciciocount}% Incrementa contador de subexercícios
    \noindent\textbf{\theexerciciocount.\thesubexerciciocount.} #1\par\vspace{0.3em}%
}

% Macro para opção
\newcommand{\option}[1]{%
    \par
    \refstepcounter{optioncount}%
    \noindent(\alph{optioncount}) #1%
}

% Título e informações do exame
\title{1ª Questão de aula do Módulo A10: Otimização}
\author{EPRALIMA - Escola Profissional Alto Lima}

\date{}

% Cabeçalho completo do teste dentro de uma caixa simples
\newcommand{\espacoAluno}{%
    \vspace{0.5cm}
    \fbox{%
        \parbox{\textwidth}{%
            \noindent\textbf{Nome do Aluno:} \underline{\hspace{7cm}} \textbf{Turma:} \underline{\hspace{1cm}}\\[0.5cm]
            \noindent\textbf{Assinatura do Professor:} \underline{\hspace{3cm}} \hfill \textbf{Nota:} \underline{\hspace{2cm}}\\[0.5cm]
            \noindent\textbf{Assinatura do Encarregado de Educação:} \underline{\hspace{3cm}}
        }%
    }
    \vspace{1cm}
}}{%
      % style.tex not found - define minimal counters/macros defensively
      \makeatletter
      \@ifundefined{exerciciocount}{\newcounter{exerciciocount}}{}
      \@ifundefined{subexerciciocount}{\newcounter{subexerciciocount}}{}
      \@ifundefined{optioncount}{\newcounter{optioncount}}{}

      \newcommand{\exercicio}[1]{%
        \par\vspace{1.5em}%
        \refstepcounter{exerciciocount}%
        \setcounter{subexerciciocount}{0}%
        \setcounter{optioncount}{0}%
        \noindent\textbf{Exercício~\theexerciciocount.} #1\par\vspace{0.5em}%
      }

      \newcommand{\subexercicio}[1]{%
        \par\vspace{0.8em}%
        \refstepcounter{subexerciciocount}%
        \noindent\textbf{\theexerciciocount.\thesubexerciciocount.} #1\par\vspace{0.3em}%
      }

      \newcommand{\exercicioDesenvolvimento}[1]{\par\noindent #1\par}
      \newcommand{\option}[1]{%
        \par\refstepcounter{optioncount}%
        \noindent(\alph{optioncount}) #1%
      }
      \makeatother
    }%
  }%
}

% ========== IP-BASED TEST SYSTEM MACROS (v3.5) ==========
% Support for modular exercise inclusion with numbered headings
% Provide a boolean flag to control whether the automatic heading is shown
\makeatletter
\@ifundefined{showexerciciotitletrue}{%
    \newif\ifshowexerciciotitle
    \showexerciciotitletrue
}{}
\makeatother

% Override \exercicio to respect the \ifshowexerciciotitle flag
% When false, it prints only the content without automatic heading
\renewcommand{\exercicio}[1]{%
    \ifshowexerciciotitle
        \par\vspace{1.5em}%
        \refstepcounter{exerciciocount}%
        \setcounter{subexerciciocount}{0}%
        \setcounter{optioncount}{0}%
        \noindent\textbf{Exercício~\theexerciciocount.} #1\par\vspace{0.5em}%
    \else
        #1\par
    \fi
}

\pagestyle{fancy}
\fancyhf{}
\lhead{Módulo P1 - Modelos Matemáticos para a Cidadania}
\rhead{eleicoes}
\cfoot{\thepage}

\title{}
\author{}
\date{}

\begin{document}
\maketitle

\section*{eleicoes}

\subsection*{Tipos de Exercícios}
\begin{itemize}
  \item \textbf{Análise de Tabelas Eleitorais}
  \item \textbf{Compreensão de Termos Eleitorais}
  \item \textbf{Conceitos Eleitorais Básicos}
\end{itemize}

\vspace{1em}

% Exercício 1: MAT_P1MODELO_ELE_MATCH_001.tex
% meta:
% id: MAT_P1MODELO_ELE_MATCH_001
% title: "Associação: termos eleitorais e definições"
% difficulty: "easy"
% tags: "eleições,definições,compreensão"
% author: "Auto-generated"
% created_at: 2025-11-20T00:00:00Z
% version: 1
\exercicio{
Liga as colunas: (à esquerda termos; à direita definições)

\begin{tabular}{@{}p{0.55\textwidth}p{0.35\textwidth}@{}}
\textbf{Termos} & \textbf{Definições} \\
\midrule
\begin{enumerate}[label=\textbf{(\arabic*)}, leftmargin=*, itemsep=10pt]
  \item Maioria simples
  \item Maioria absoluta
  \item Voto nulo
  \item Voto em branco
  \item Abstenção
\end{enumerate}
&
\begin{enumerate}[label=\textbf{\Alph*}, leftmargin=*, itemsep=8pt]
  \item Exige mais votos que qualquer outro concorrente; não precisa de exceder metade dos votos válidos.
  \item Votos expressos sem contar abstenções; maioria superior a metade dos votos válidos.
  \item Voto inválido por irregularidade na folha (marcações contraditórias, sinais identificadores, etc.).
  \item Voto depositado que não assinala qualquer opção; em muitos sistemas é contabilizado separadamente dos nulos.
  \item Eleitor inscrito que não comparece à mesa de voto no dia da eleição.
\end{enumerate}
\\
\end{tabular}

\vspace{8pt}

oindent Respostas: 1. \campoLetra \quad 2. \campoLetra \quad 3. \campoLetra \quad 4. \campoLetra \quad 5. \campoLetra

\vspace{3cm}
}
\FloatBarrier

% Exercício 2: MAT_P1MODELO_EXX_CTX_001.tex
% Exercise ID: MAT_P1MODELO_EXX_CTX_001
% Module: Módulo P1 - Modelos Matemáticos para a Cidadania | Concept: Eleições | Type: Compreensão de Termos Eleitorais
% Difficulty: 4/5 (Difícil) | Format: desenvolvimento
% Tags: democracia, votacao, percentagens, leitura, definições, associação, teste, automacao
% Author: Teste Automático | Date: 2025-11-26
% Status: active

\exercicio{Exercício de teste para compreensao_termos}

% Solution:
% \begin{solucao}
% Solução de exemplo
% \end{solucao}
\FloatBarrier

% Exercício 3: MAT_P1MODELO_EXX_CTX_002.tex
% Exercise ID: MAT_P1MODELO_EXX_CTX_002
% Module: Módulo P1 - Modelos Matemáticos para a Cidadania | Concept: Eleições | Type: Compreensão de Termos Eleitorais
% Difficulty: 2/5 (Fácil) | Format: desenvolvimento
% Tags: democracia, associação, percentagens, leitura, votacao, definições, teste, automacao
% Author: Teste Automático | Date: 2025-11-26
% Status: active

\exercicio{Exercício de teste para compreensao_termos}
\FloatBarrier

% Exercício 4: MAT_P1MODELO_ELE_CONC_001.tex
% meta:
% id: MAT_P1MODELO_ELE_CONC_001
% title: "Verdadeiro/Falso — conceitos eleitorais"
% difficulty: "easy"
% tags: "eleições,maioria,abstenção,votos"
% author: "Auto-generated"
% created_at: 2025-11-20T00:00:00Z
% version: 1
\exercicio{
Instrucções: Para cada afirmação, escreve ``V'' se considerares a afirmação verdadeira ou ``F'' se considerares falsa.

\begin{enumerate}[label=\textbf{(\arabic*)}, leftmargin=*, itemsep=1.0\baselineskip]
  \item A maioria simples exige que um candidato obtenha mais votos do que qualquer outro, independentemente do número total de votos válidos.
  \item A maioria absoluta significa obter mais de metade dos votos válidos.
  \item Votos nulos não entram na contagem dos votos válidos.
  \item A abstenção corresponde a eleitores que se registaram mas não compareceram à votação.
  \item Para uma eleição que exige maioria absoluta, se nenhum candidato obtiver mais de metade dos votos válidos, realiza-se um segundo turno entre os dois mais votados.
  \item Se um eleitor deposita uma folha de voto em que não marca nenhum candidato, esse voto é considerado voto em branco.
\end{enumerate}

\vspace{3cm}
}
\FloatBarrier

% Exercício 5: MAT_P1MODELO_ELE_PERC_001.tex
% meta:
% id: MAT_P1MODELO_ELE_PERC_001
% title: "Calcular votos a partir de percentagens — vários exemplos"
% difficulty: "medium"
% tags: "percentagens,total,votos,conversão"
% author: "Auto-generated"
% created_at: 2025-11-20T00:00:00Z
% version: 1
\exercicio{
Preenche a tabela com os votos calculados a partir das percentagens e do total, e responde:
\begin{enumerate}
\item Considera a seguinte tabela com percentagens de votos. Preenche os votos de cada candidato, sabendo que o total de votos válidos é 1200, com 10 votos em branco e 5 nulos.
\begin{center}
\begin{tabular}{lrr}
\toprule
\textbf{Candidato} & \textbf{\%} & \textbf{Votos (para preencher)}\\
\midrule
A & 38\% & \\
B & 34\% & \\
C & 25\% & \\
D & 3\% & \\
\midrule
\textbf{Total votos válidos} & & 1\,200\\
Votos em branco & & 10\\
Votos nulos & & 5\\
\bottomrule
\end{tabular}
\end{center}
\begin{enumerate}
  \item Calcula o número de votos de cada candidato. \campo[6.0cm]
  \item Alguém obteve maioria absoluta? \campo[6.0cm]
  \item Percentagens de brancos e nulos. \campo[6.0cm]
\end{enumerate}
\item Considera a seguinte tabela com percentagens de votos. Preenche os votos de cada candidato, sabendo que o total de votos válidos é 2000, com 15 votos em branco e 5 nulos.
\begin{center}
\begin{tabular}{lrr}
\toprule
\textbf{Candidato} & \textbf{\%} & \textbf{Votos (para preencher)}\\
\midrule
E & 40\% & \\
F & 30\% & \\
G & 20\% & \\
H & 10\% & \\
\midrule
\textbf{Total votos válidos} & & 2\,000\\
Votos em branco & & 15\\
Votos nulos & & 5\\
\bottomrule
\end{tabular}
\end{center}
\begin{enumerate}
  \item Calcula o número de votos de cada candidato. \campo[6.0cm]
  \item Alguém obteve maioria absoluta? \campo[6.0cm]
  \item Percentagens de brancos e nulos. \campo[6.0cm]
\end{enumerate}
\item Considera a seguinte tabela com percentagens de votos. Preenche os votos de cada candidato, sabendo que o total de votos válidos é 800, com 20 votos em branco e 0 nulos.
\begin{center}
\begin{tabular}{lrr}
\toprule
\textbf{Candidato} & \textbf{\%} & \textbf{Votos (para preencher)}\\
\midrule
I & 45\% & \\
J & 35\% & \\
K & 15\% & \\
L & 5\% & \\
\midrule
\textbf{Total votos válidos} & & 800\\
Votos em branco & & 20\\
Votos nulos & & 0\\
\bottomrule
\end{tabular}
\end{center}
\begin{enumerate}
  \item Calcula o número de votos de cada candidato. \campo[6.0cm]
  \item Alguém obteve maioria absoluta? \campo[6.0cm]
  \item Percentagens de brancos e nulos. \campo[6.0cm]
\end{enumerate}
\end{enumerate}

\vspace{3cm}
}
\FloatBarrier

% Exercício 6: MAT_P1MODELO_ELE_TAB1_001.tex
% meta:
% id: MAT_P1MODELO_ELE_TAB1_001
% title: "Analisar tabela eleitoral — exemplo A/B/C"
% difficulty: "medium"
% tags: "tabelas,percentagens,maioria,abstenção"
% author: "Auto-generated"
% created_at: 2025-11-20T00:00:00Z
% version: 1
\exercicio{
Tabela:
\begin{center}
\begin{tabular}{lrr}
\toprule
Candidato & Votos & \% (para preencher) \\
\midrule
A & 420 & \\
B & 310 & \\
C & 170 & \\
\midrule
Total votos válidos & 900 & \\
Votos em branco & 25 & \\
Votos nulos & 10 & \\
\bottomrule
\end{tabular}
\end{center}

Perguntas:
\begin{enumerate}
  \item Alguém obteve maioria simples? Resposta e cálculo: \campo[6.0cm]
  \item Alguém obteve maioria absoluta? Resposta e cálculo: \campo[6.0cm]
  \item Calcula as percentagens de cada candidato (com duas casas decimais). \campo[6.0cm]
\end{enumerate}

\vspace{3cm}
}
\FloatBarrier

% Exercício 7: MAT_P1MODELO_ELE_TAB2_001.tex
% meta:
% id: MAT_P1MODELO_ELE_TAB2_001
% title: "Analisar tabela eleitoral — exemplo D/E/F/G"
% difficulty: "medium"
% tags: "tabelas,percentagens,maioria,brancos,nulos"
% author: "Auto-generated"
% created_at: 2025-11-20T00:00:00Z
% version: 1
\exercicio{
Tabela:
\begin{center}
\begin{tabular}{lrr}
\toprule
Candidato & Votos & \% (para preencher) \\
\midrule
D & 1\,150 & \\
E & 980 & \\
F & 720 & \\
G & 150 & \\
\midrule
Total votos válidos & 3\,000 &  \\
Votos em branco & 60 & \\
Votos nulos & 40 & \\
\bottomrule
\end{tabular}
\end{center}

Perguntas:
\begin{enumerate}
  \item Alguém obteve maioria simples? Resposta e cálculo: \campo[6.0cm]
  \item Alguém obteve maioria absoluta? Resposta e cálculo: \campo[6.0cm]
  \item Qual foi a percentagem de votos em branco e de votos nulos? \campo[6.0cm]
\end{enumerate}

\vspace{3cm}
}
\FloatBarrier

% Exercício 8: MAT_P1MODELO_ELE_TAB3_001.tex
% meta:
% id: MAT_P1MODELO_ELE_TAB3_001
% title: "Tabelas com abstenção e percentagens — exemplo H/I"
% difficulty: "medium"
% tags: "abstenção,percentagens,tabelas"
% author: "Auto-generated"
% created_at: 2025-11-20T00:00:00Z
% version: 1
\exercicio{
Contém um exemplo com abstenção. Tabela:
\begin{tabular}{lrr}
\toprule
Candidato & Votos & \% \\
\midrule
H & 2\,300 & \\
I & 1\,900 & \\
\midrule
Total votos válidos & 4\,200 & \\
Votos em branco & 150 & \\
Votos nulos & 50 & \\
Eleitores inscritos & 6\,000 & \\
\bottomrule
\end{tabular}

Perguntas:
\begin{enumerate}
  \item Alguém obteve maioria simples? Resposta e cálculo: \campo[6.0cm]
  \item Alguém obteve maioria absoluta? Resposta e cálculo: \campo[6.0cm]
  \item Qual é a taxa de abstenção (em percentagem)? Resposta e cálculo: \campo[6.0cm]
  \item Percentagem de votos em branco e nulos? Resposta e cálculo: \campo[6.0cm]
\end{enumerate}

\vspace{3cm}
}
\FloatBarrier

% Exercício 9: MAT_P1MODELO_EXX_ATE_001.tex
% Exercise ID: MAT_P1MODELO_EXX_ATE_001
% Module: Módulo P1 - Modelos Matemáticos para a Cidadania | Concept: Eleições | Type: Análise de Tabelas Eleitorais
% Difficulty: 5/5 (Muito Difícil) | Format: desenvolvimento
% Tags: percentagens, tabelas, maioria, abstenção, brancos, democracia, votacao, nulos, teste, automacao
% Author: Teste Automático | Date: 2025-11-26
% Status: active

\exercicio{Exercício de teste para analise_tabelas_eleitorais}
\FloatBarrier

\end{document}
