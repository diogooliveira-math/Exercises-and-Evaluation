S% Template para geração automática de sebentas
% Gerado automaticamente - NÃO EDITAR MANUALMENTE
\documentclass[12pt,a4paper]{article}

% Encoding e idioma
\usepackage[utf8]{inputenc}
\usepackage[T1]{fontenc}
\usepackage[portuguese]{babel}

% Matemática
\usepackage{amsmath}
\usepackage{amssymb}
\usepackage{amsthm}
\usepackage{mathtools}

% Gráficos e figuras
\usepackage{graphicx}
\usepackage{tikz}
\usetikzlibrary{calc,patterns,angles,quotes}
\usepackage{pgfplots}
\pgfplotsset{compat=1.18}
% Force float barriers when needed to keep figures right after exercises
\usepackage{placeins}
\usepackage{float}  % Para figure[H] - força figuras exatamente onde definidas

% Layout e formatação
\usepackage{geometry}
\geometry{a4paper,margin=2.5cm,top=3cm,bottom=3cm}
\usepackage{fancyhdr}
\usepackage{enumitem}
\usepackage{multicol}
\usepackage{booktabs}

% Hyperlinks e referências
\usepackage{hyperref}
\hypersetup{
    colorlinks=true,
    linkcolor=blue,
    urlcolor=blue,
    citecolor=blue
}

% Sistema de exercícios - macros personalizadas
\newcounter{exerciciocount}
\newcounter{subexerciciocount}
\newcounter{optioncount}

\newcommand{\exercicio}[1][]{% 
    \par\vspace{1.5em}%
    \refstepcounter{exerciciocount}%
    \setcounter{subexerciciocount}{0}%
    \setcounter{optioncount}{0}%
    \noindent\textbf{Exercício~\theexerciciocount.}%
    \ifx&#1&%
        % Argumento vazio - o conteúdo vem depois
        \par\vspace{0.3em}%
    \else%
        % Argumento fornecido - incluir inline
        \ #1\par\vspace{0.5em}%
    \fi%
}

\newcommand{\subexercicio}[1]{%
    \par\vspace{0.8em}%
    \refstepcounter{subexerciciocount}%
    \noindent\textbf{\theexerciciocount.\thesubexerciciocount.} #1\par\vspace{0.3em}%
}

\newcommand{\option}[1]{%
    \par
    \refstepcounter{optioncount}%
    \noindent(\alph{optioncount}) #1%
}

% Campos para respostas
\newcommand{\campo}[1][2.0cm]{\makebox[#1]{\hrulefill}}
\newcommand{\campoLetra}[1][1.2cm]{\makebox[#1]{\hrulefill}}
\newcommand{\campoCents}[1][3.0cm]{\makebox[#1]{\hrulefill}}% Cabeçalho e rodapé
\pagestyle{fancy}
\fancyhf{}
\fancyhead[L]{Módulo A9 - Funções de Crescimento}
\fancyhead[R]{0 - revisoes}
\fancyfoot[C]{\thepage}

% Metadados do documento
\title{}
\author{}
\date{}

\begin{document}

% Remover título, autor e data - apenas conteúdo
\thispagestyle{fancy}

\section*{0 - revisoes}

\subsection*{Tipos de Exercícios}
\begin{itemize}
  \item \textbf{Aplicações Práticas} --- Exercícios de aplicação de conceitos matemáticos básicos (proporcionalidade, percentagens) em contextos práticos do quotidiano.
  \item \textbf{Cálculo com Funções} --- Exercícios de cálculo de imagens, objetos e manipulação algébrica básica de funções.
  \item \textbf{Conceito de Função} --- Exercícios sobre o conceito fundamental de função, verificação de correspondências e identificação de funções.
\end{itemize}

\vspace{1em}

% Exercício 1: MAT_A9FUNCOE_0RX_APX_001.tex
% Exercise ID: MAT_A9FUNCOE_0RX_APX_001
% Module: Módulo A9 - Funções de Crescimento | Concept: Revisões de Crescimento | Type: Aplicações Práticas
% Difficulty: 1/5 (Muito Fácil) | Format: desenvolvimento
% Tags: crescimento, aplicacao_pratica, contexto_real, consolidacao, revisao, proporcionalidade, percentagens
% Author: Professor | Date: 2025-11-20
% Status: active

\exercicio{
Uma família de 4 pessoas consome 2 kg de arroz por semana. Quantos kg de arroz serão necessários para uma família de 6 pessoas, mantendo a mesma proporção?
}
\FloatBarrier

% Exercício 2: MAT_A9FUNCRE_0REV_APL_001.tex
% meta:
% id: MAT_A9FUNCRE_0REV_APL_101
% module: A9_funcoes_crescimento
% concept: 0-revisoes
% concept_name: Revisões de Crescimento
% tipo: aplicacoes_praticas
% tipo_nome: Aplicações Práticas
% difficulty: 2
% tags: revisao,proporcionalidade,aplicacao_pratica
% author: sistema
\exercicio{
Uma receita de bolo de iogurte para 2 pessoas usa:
\begin{itemize}
  \item 1 iogurte natural
  \item 2 copos de açúcar
  \item 3 copos de farinha
  \item $\frac{1}{2}$ copo de óleo
  \item 3 ovos
\end{itemize}
Pretende-se ajustar para 12 pessoas. Indica as quantidades proporcionais.
}
\FloatBarrier

% Exercício 3: MAT_A9FUNCRE_0REV_APL_002.tex
% meta:
% id: MAT_A9FUNCRE_0REV_APL_102
% module: A9_funcoes_crescimento
% concept: 0-revisoes
% concept_name: Revisões de Crescimento
% tipo: aplicacoes_praticas
% tipo_nome: Aplicações Práticas
% difficulty: 2
% tags: revisao,percentagens,aplicacao_pratica
% author: sistema
\exercicio{
O Clube A comprou um jogador por 13 milhões de euros, o que representa 18\% do seu orçamento anual. Determina o orçamento total e discute se a percentagem é sustentável face a uma segunda contratação igual.
}
\FloatBarrier

% Exercício 4: MAT_A9FUNCRES_0REV_003.tex
% meta:
% id: MAT_A9FUNCRE_0REV_APL_001
% module: A9_funcoes_crescimento
% concept: 0-revisoes
% concept_name: Revisões de Crescimento
% tipo: aplicacoes_praticas
% tipo_nome: Aplicações Práticas
% difficulty: 2
% tags: revisao,proporcionalidade,aplicacao_pratica
% author: sistema
\exercicio{
Uma receita de bolo de iogurte para 2 pessoas usa:
\begin{itemize}
  \item 1 iogurte natural
  \item 2 copos de açúcar
  \item 3 copos de farinha
  \item $\frac{1}{2}$ copo de óleo
  \item 3 ovos
\end{itemize}
Pretende-se ajustar para 10 pessoas. Indica as quantidades proporcionais.
}
\FloatBarrier

% Exercício 5: MAT_A9FUNCRES_0REV_004.tex
% meta:
% id: MAT_A9FUNCRE_0REV_APL_002
% module: A9_funcoes_crescimento
% concept: 0-revisoes
% concept_name: Revisões de Crescimento
% tipo: aplicacoes_praticas
% tipo_nome: Aplicações Práticas
% difficulty: 2
% tags: revisao,percentagens,aplicacao_pratica
% author: sistema
\exercicio{
O Clube A comprou um jogador por 12 milhões de euros, o que representa 20\% do seu orçamento anual. Determina o orçamento total e discute se a percentagem é sustentável face a uma segunda contratação igual.
}
\FloatBarrier

% Exercício 6: MAT_A9FUNCRE_0REV_CAL_001.tex
% meta:
% id: MAT_A9FUNCRE_0REV_CAL_103
% module: A9_funcoes_crescimento
% concept: 0-revisoes
% concept_name: Revisões de Crescimento
% tipo: calculo_funcoes
% tipo_nome: Cálculo com Funções
% difficulty: 2
% tags: revisao,funcoes,calculo_imagens
% author: sistema
\exercicio{
Considera a função $f(x)=2x-4$.
\begin{itemize}
  \item[a)] Calcula $f(0)$, $f(2)$ e $f(5)$.
  \item[b)] Qual é o valor de $x$ tal que $f(x)=1$?
\end{itemize}
}
\FloatBarrier

% Exercício 7: MAT_A9FUNCRES_0REV_002.tex
% meta:
% id: MAT_A9FUNCRE_0REV_CAL_001
% module: A9_funcoes_crescimento
% concept: 0-revisoes
% concept_name: Revisões de Crescimento
% tipo: calculo_funcoes
% tipo_nome: Cálculo com Funções
% difficulty: 2
% tags: revisao,funcoes,calculo_imagens
% author: sistema
\exercicio{
Considera a função $f(x)=2x-3$.
\begin{itemize}
  \item[a)] Calcula $f(0)$, $f(2)$ e $f(5)$.
  \item[b)] Qual é o valor de $x$ tal que $f(x)=1$?
\end{itemize}
}
\FloatBarrier

% Exercício 8: MAT_A9FUNCRE_0REV_CFU_001.tex
% meta:
% id: MAT_A9FUNCRE_0REV_CFU_001
% module: A9_funcoes_crescimento
% concept: 0-revisoes
% concept_name: Revisões de Crescimento
% tipo: conceito_funcao
% tipo_nome: Conceito de Função
% difficulty: 2
% tags: revisao,funcoes,conceito_funcao
% author: sistema
\exercicio{
Considera a correspondência seguinte entre pessoas e o número de sapatos que calçam:
\[
\begin{tabular}{|c|c|}
\hline
\text{Pessoa} & \text{Número de sapatos que calça} \\
\hline
\text{Ana} & 37 \\
\text{Bruno} & 42 \\
\text{Carla} & 39 \\
\text{David} & 42 \\
\hline
\end{tabular}
\]
\textbf{Pergunta:} Esta correspondência é uma função? Justifica escolhendo a opção correta e explicando por que as outras estão erradas.
\begin{enumerate}
 \item[(A)] Não é uma função, porque o número 42 aparece duas vezes.
 \item[(B)] É uma função, porque cada pessoa está associada a um único número.
 \item[(C)] Não é uma função, porque Bruno e David calçam o mesmo número.
 \item[(D)] Não é uma função, porque há números repetidos na segunda coluna.
\end{enumerate}
}
\FloatBarrier

% Exercício 9: MAT_A9FUNCRES_0REV_001.tex
% meta:
% id: MAT_A9FUNCRE_0REV_CFU_001
% module: A9_funcoes_crescimento
% concept: 0-revisoes
% concept_name: Revisões de Crescimento
% tipo: conceito_funcao
% tipo_nome: Conceito de Função
% difficulty: 2
% tags: revisao,funcoes,conceito_funcao
% author: sistema
\exercicio{
Considera a correspondência seguinte entre pessoas e o número de sapatos que calçam:
\[
\begin{array}{c|c}
\text{Pessoa} & \text{Número de sapatos que calça} \\
\hline
\text{Ana} & 37 \\
\text{Bruno} & 42 \\
\text{Carla} & 39 \\
\text{David} & 42 \\
\end{array}
\]
\textbf{Pergunta:} Esta correspondência é uma função? Justifica escolhendo a opção correta e explicando por que as outras estão erradas.
\begin{enumerate}
 \item[(A)] Não é uma função, porque o número 42 aparece duas vezes.
 \item[(B)] É uma função, porque cada pessoa está associada a um único número.
 \item[(C)] Não é uma função, porque Bruno e David calçam o mesmo número.
 \item[(D)] Não é uma função, porque há números repetidos na segunda coluna.
\end{enumerate}
}
\FloatBarrier


\end{document}
