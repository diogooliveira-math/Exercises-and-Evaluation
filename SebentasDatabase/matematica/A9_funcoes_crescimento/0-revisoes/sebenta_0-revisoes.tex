% Minimal sebenta template generated automatically
\documentclass[11pt,a4paper]{article}
\usepackage[utf8]{inputenc}
\usepackage[T1]{fontenc}
\usepackage{lmodern}
\usepackage{geometry}
\usepackage{fancyhdr}
\usepackage{hyperref}
\usepackage{graphicx}
\usepackage{float}
\usepackage{placeins}
\usepackage{bookmark}
\usepackage{booktabs}
\usepackage{amsmath,amssymb}
\usepackage{csquotes}
\usepackage{enumitem}
\usepackage{tikz}
\IfFileExists{pgfplots.sty}{\usepackage{pgfplots}\pgfplotsset{compat=1.17}}{}

\geometry{margin=2.5cm}

% Try to include project-specific style macros (containing \exercicio, \subexercicio, etc.)
% Try multiple relative locations to be robust across different generated output paths
\IfFileExists{../../../../Teste_modelo/config/style.tex}{% Sistema de exercícios com contadores automáticos
\newcounter{exerciciocount}          % Contador principal dos exercícios
\newcounter{subexerciciocount}       % Contador dos subexercícios
\newcounter{optioncount}             % Contador das opções

% Control whether the macro prints the automatic "Exercício N." heading.
% Default: show the heading. Call \showexerciciotitlefalse to suppress.
\newif\ifshowexerciciotitle
\showexerciciotitletrue

% Macro para exercício principal
\newcommand{\exercicio}[1]{%
        \par\vspace{1.5em}% Espaçamento antes
        \refstepcounter{exerciciocount}% Incrementa contador principal
        \setcounter{subexerciciocount}{0}% Reseta contador de subexercícios
        \setcounter{optioncount}{0}% Reseta contador de opções
        % Only print the automatic heading if the flag is true
        \ifshowexerciciotitle
            \noindent\textbf{Exercício~\theexerciciocount.}\space #1\par\vspace{0.5em}%
        \else
            % When suppressed, just print the content without the heading
            #1\par\vspace{0.5em}%
        \fi
}

% Macro para subexercício
\newcommand{\subexercicio}[1]{%
    \par\vspace{0.8em}% Espaçamento menor para subexercícios
    \refstepcounter{subexerciciocount}% Incrementa contador de subexercícios
    \noindent\textbf{\theexerciciocount.\thesubexerciciocount.} #1\par\vspace{0.3em}%
}

% Macro para opção
\newcommand{\option}[1]{%
    \par
    \refstepcounter{optioncount}%
    \noindent(\alph{optioncount}) #1%
}

% Título e informações do exame
\title{1ª Questão de aula do Módulo A10: Otimização}
\author{EPRALIMA - Escola Profissional Alto Lima}

\date{}

% Cabeçalho completo do teste dentro de uma caixa simples
\newcommand{\espacoAluno}{%
    \vspace{0.5cm}
    \fbox{%
        \parbox{\textwidth}{%
            \noindent\textbf{Nome do Aluno:} \underline{\hspace{7cm}} \textbf{Turma:} \underline{\hspace{1cm}}\\[0.5cm]
            \noindent\textbf{Assinatura do Professor:} \underline{\hspace{3cm}} \hfill \textbf{Nota:} \underline{\hspace{2cm}}\\[0.5cm]
            \noindent\textbf{Assinatura do Encarregado de Educação:} \underline{\hspace{3cm}}
        }%
    }
    \vspace{1cm}
}}{%
  \IfFileExists{../../../Teste_modelo/config/style.tex}{% Sistema de exercícios com contadores automáticos
\newcounter{exerciciocount}          % Contador principal dos exercícios
\newcounter{subexerciciocount}       % Contador dos subexercícios
\newcounter{optioncount}             % Contador das opções

% Control whether the macro prints the automatic "Exercício N." heading.
% Default: show the heading. Call \showexerciciotitlefalse to suppress.
\newif\ifshowexerciciotitle
\showexerciciotitletrue

% Macro para exercício principal
\newcommand{\exercicio}[1]{%
        \par\vspace{1.5em}% Espaçamento antes
        \refstepcounter{exerciciocount}% Incrementa contador principal
        \setcounter{subexerciciocount}{0}% Reseta contador de subexercícios
        \setcounter{optioncount}{0}% Reseta contador de opções
        % Only print the automatic heading if the flag is true
        \ifshowexerciciotitle
            \noindent\textbf{Exercício~\theexerciciocount.}\space #1\par\vspace{0.5em}%
        \else
            % When suppressed, just print the content without the heading
            #1\par\vspace{0.5em}%
        \fi
}

% Macro para subexercício
\newcommand{\subexercicio}[1]{%
    \par\vspace{0.8em}% Espaçamento menor para subexercícios
    \refstepcounter{subexerciciocount}% Incrementa contador de subexercícios
    \noindent\textbf{\theexerciciocount.\thesubexerciciocount.} #1\par\vspace{0.3em}%
}

% Macro para opção
\newcommand{\option}[1]{%
    \par
    \refstepcounter{optioncount}%
    \noindent(\alph{optioncount}) #1%
}

% Título e informações do exame
\title{1ª Questão de aula do Módulo A10: Otimização}
\author{EPRALIMA - Escola Profissional Alto Lima}

\date{}

% Cabeçalho completo do teste dentro de uma caixa simples
\newcommand{\espacoAluno}{%
    \vspace{0.5cm}
    \fbox{%
        \parbox{\textwidth}{%
            \noindent\textbf{Nome do Aluno:} \underline{\hspace{7cm}} \textbf{Turma:} \underline{\hspace{1cm}}\\[0.5cm]
            \noindent\textbf{Assinatura do Professor:} \underline{\hspace{3cm}} \hfill \textbf{Nota:} \underline{\hspace{2cm}}\\[0.5cm]
            \noindent\textbf{Assinatura do Encarregado de Educação:} \underline{\hspace{3cm}}
        }%
    }
    \vspace{1cm}
}}{%
    \IfFileExists{../../Teste_modelo/config/style.tex}{% Sistema de exercícios com contadores automáticos
\newcounter{exerciciocount}          % Contador principal dos exercícios
\newcounter{subexerciciocount}       % Contador dos subexercícios
\newcounter{optioncount}             % Contador das opções

% Control whether the macro prints the automatic "Exercício N." heading.
% Default: show the heading. Call \showexerciciotitlefalse to suppress.
\newif\ifshowexerciciotitle
\showexerciciotitletrue

% Macro para exercício principal
\newcommand{\exercicio}[1]{%
        \par\vspace{1.5em}% Espaçamento antes
        \refstepcounter{exerciciocount}% Incrementa contador principal
        \setcounter{subexerciciocount}{0}% Reseta contador de subexercícios
        \setcounter{optioncount}{0}% Reseta contador de opções
        % Only print the automatic heading if the flag is true
        \ifshowexerciciotitle
            \noindent\textbf{Exercício~\theexerciciocount.}\space #1\par\vspace{0.5em}%
        \else
            % When suppressed, just print the content without the heading
            #1\par\vspace{0.5em}%
        \fi
}

% Macro para subexercício
\newcommand{\subexercicio}[1]{%
    \par\vspace{0.8em}% Espaçamento menor para subexercícios
    \refstepcounter{subexerciciocount}% Incrementa contador de subexercícios
    \noindent\textbf{\theexerciciocount.\thesubexerciciocount.} #1\par\vspace{0.3em}%
}

% Macro para opção
\newcommand{\option}[1]{%
    \par
    \refstepcounter{optioncount}%
    \noindent(\alph{optioncount}) #1%
}

% Título e informações do exame
\title{1ª Questão de aula do Módulo A10: Otimização}
\author{EPRALIMA - Escola Profissional Alto Lima}

\date{}

% Cabeçalho completo do teste dentro de uma caixa simples
\newcommand{\espacoAluno}{%
    \vspace{0.5cm}
    \fbox{%
        \parbox{\textwidth}{%
            \noindent\textbf{Nome do Aluno:} \underline{\hspace{7cm}} \textbf{Turma:} \underline{\hspace{1cm}}\\[0.5cm]
            \noindent\textbf{Assinatura do Professor:} \underline{\hspace{3cm}} \hfill \textbf{Nota:} \underline{\hspace{2cm}}\\[0.5cm]
            \noindent\textbf{Assinatura do Encarregado de Educação:} \underline{\hspace{3cm}}
        }%
    }
    \vspace{1cm}
}}{%
      % style.tex not found - proceed without project macros
    }%
  }%
}

% Provide a robust fallback for macros that might be missing in style.tex
% This attempts to include the project style first (multiple relative paths),
% and only if none exist defines minimal counters and macros safely.
\IfFileExists{../../../../Teste_modelo/config/style.tex}{% Sistema de exercícios com contadores automáticos
\newcounter{exerciciocount}          % Contador principal dos exercícios
\newcounter{subexerciciocount}       % Contador dos subexercícios
\newcounter{optioncount}             % Contador das opções

% Control whether the macro prints the automatic "Exercício N." heading.
% Default: show the heading. Call \showexerciciotitlefalse to suppress.
\newif\ifshowexerciciotitle
\showexerciciotitletrue

% Macro para exercício principal
\newcommand{\exercicio}[1]{%
        \par\vspace{1.5em}% Espaçamento antes
        \refstepcounter{exerciciocount}% Incrementa contador principal
        \setcounter{subexerciciocount}{0}% Reseta contador de subexercícios
        \setcounter{optioncount}{0}% Reseta contador de opções
        % Only print the automatic heading if the flag is true
        \ifshowexerciciotitle
            \noindent\textbf{Exercício~\theexerciciocount.}\space #1\par\vspace{0.5em}%
        \else
            % When suppressed, just print the content without the heading
            #1\par\vspace{0.5em}%
        \fi
}

% Macro para subexercício
\newcommand{\subexercicio}[1]{%
    \par\vspace{0.8em}% Espaçamento menor para subexercícios
    \refstepcounter{subexerciciocount}% Incrementa contador de subexercícios
    \noindent\textbf{\theexerciciocount.\thesubexerciciocount.} #1\par\vspace{0.3em}%
}

% Macro para opção
\newcommand{\option}[1]{%
    \par
    \refstepcounter{optioncount}%
    \noindent(\alph{optioncount}) #1%
}

% Título e informações do exame
\title{1ª Questão de aula do Módulo A10: Otimização}
\author{EPRALIMA - Escola Profissional Alto Lima}

\date{}

% Cabeçalho completo do teste dentro de uma caixa simples
\newcommand{\espacoAluno}{%
    \vspace{0.5cm}
    \fbox{%
        \parbox{\textwidth}{%
            \noindent\textbf{Nome do Aluno:} \underline{\hspace{7cm}} \textbf{Turma:} \underline{\hspace{1cm}}\\[0.5cm]
            \noindent\textbf{Assinatura do Professor:} \underline{\hspace{3cm}} \hfill \textbf{Nota:} \underline{\hspace{2cm}}\\[0.5cm]
            \noindent\textbf{Assinatura do Encarregado de Educação:} \underline{\hspace{3cm}}
        }%
    }
    \vspace{1cm}
}}{%
  \IfFileExists{../../../Teste_modelo/config/style.tex}{% Sistema de exercícios com contadores automáticos
\newcounter{exerciciocount}          % Contador principal dos exercícios
\newcounter{subexerciciocount}       % Contador dos subexercícios
\newcounter{optioncount}             % Contador das opções

% Control whether the macro prints the automatic "Exercício N." heading.
% Default: show the heading. Call \showexerciciotitlefalse to suppress.
\newif\ifshowexerciciotitle
\showexerciciotitletrue

% Macro para exercício principal
\newcommand{\exercicio}[1]{%
        \par\vspace{1.5em}% Espaçamento antes
        \refstepcounter{exerciciocount}% Incrementa contador principal
        \setcounter{subexerciciocount}{0}% Reseta contador de subexercícios
        \setcounter{optioncount}{0}% Reseta contador de opções
        % Only print the automatic heading if the flag is true
        \ifshowexerciciotitle
            \noindent\textbf{Exercício~\theexerciciocount.}\space #1\par\vspace{0.5em}%
        \else
            % When suppressed, just print the content without the heading
            #1\par\vspace{0.5em}%
        \fi
}

% Macro para subexercício
\newcommand{\subexercicio}[1]{%
    \par\vspace{0.8em}% Espaçamento menor para subexercícios
    \refstepcounter{subexerciciocount}% Incrementa contador de subexercícios
    \noindent\textbf{\theexerciciocount.\thesubexerciciocount.} #1\par\vspace{0.3em}%
}

% Macro para opção
\newcommand{\option}[1]{%
    \par
    \refstepcounter{optioncount}%
    \noindent(\alph{optioncount}) #1%
}

% Título e informações do exame
\title{1ª Questão de aula do Módulo A10: Otimização}
\author{EPRALIMA - Escola Profissional Alto Lima}

\date{}

% Cabeçalho completo do teste dentro de uma caixa simples
\newcommand{\espacoAluno}{%
    \vspace{0.5cm}
    \fbox{%
        \parbox{\textwidth}{%
            \noindent\textbf{Nome do Aluno:} \underline{\hspace{7cm}} \textbf{Turma:} \underline{\hspace{1cm}}\\[0.5cm]
            \noindent\textbf{Assinatura do Professor:} \underline{\hspace{3cm}} \hfill \textbf{Nota:} \underline{\hspace{2cm}}\\[0.5cm]
            \noindent\textbf{Assinatura do Encarregado de Educação:} \underline{\hspace{3cm}}
        }%
    }
    \vspace{1cm}
}}{%
    \IfFileExists{../../Teste_modelo/config/style.tex}{% Sistema de exercícios com contadores automáticos
\newcounter{exerciciocount}          % Contador principal dos exercícios
\newcounter{subexerciciocount}       % Contador dos subexercícios
\newcounter{optioncount}             % Contador das opções

% Control whether the macro prints the automatic "Exercício N." heading.
% Default: show the heading. Call \showexerciciotitlefalse to suppress.
\newif\ifshowexerciciotitle
\showexerciciotitletrue

% Macro para exercício principal
\newcommand{\exercicio}[1]{%
        \par\vspace{1.5em}% Espaçamento antes
        \refstepcounter{exerciciocount}% Incrementa contador principal
        \setcounter{subexerciciocount}{0}% Reseta contador de subexercícios
        \setcounter{optioncount}{0}% Reseta contador de opções
        % Only print the automatic heading if the flag is true
        \ifshowexerciciotitle
            \noindent\textbf{Exercício~\theexerciciocount.}\space #1\par\vspace{0.5em}%
        \else
            % When suppressed, just print the content without the heading
            #1\par\vspace{0.5em}%
        \fi
}

% Macro para subexercício
\newcommand{\subexercicio}[1]{%
    \par\vspace{0.8em}% Espaçamento menor para subexercícios
    \refstepcounter{subexerciciocount}% Incrementa contador de subexercícios
    \noindent\textbf{\theexerciciocount.\thesubexerciciocount.} #1\par\vspace{0.3em}%
}

% Macro para opção
\newcommand{\option}[1]{%
    \par
    \refstepcounter{optioncount}%
    \noindent(\alph{optioncount}) #1%
}

% Título e informações do exame
\title{1ª Questão de aula do Módulo A10: Otimização}
\author{EPRALIMA - Escola Profissional Alto Lima}

\date{}

% Cabeçalho completo do teste dentro de uma caixa simples
\newcommand{\espacoAluno}{%
    \vspace{0.5cm}
    \fbox{%
        \parbox{\textwidth}{%
            \noindent\textbf{Nome do Aluno:} \underline{\hspace{7cm}} \textbf{Turma:} \underline{\hspace{1cm}}\\[0.5cm]
            \noindent\textbf{Assinatura do Professor:} \underline{\hspace{3cm}} \hfill \textbf{Nota:} \underline{\hspace{2cm}}\\[0.5cm]
            \noindent\textbf{Assinatura do Encarregado de Educação:} \underline{\hspace{3cm}}
        }%
    }
    \vspace{1cm}
}}{%
      % style.tex not found - define minimal counters/macros defensively
      \makeatletter
      \@ifundefined{exerciciocount}{\newcounter{exerciciocount}}{}
      \@ifundefined{subexerciciocount}{\newcounter{subexerciciocount}}{}
      \@ifundefined{optioncount}{\newcounter{optioncount}}{}

      \newcommand{\exercicio}[1]{%
        \par\vspace{1.5em}%
        \refstepcounter{exerciciocount}%
        \setcounter{subexerciciocount}{0}%
        \setcounter{optioncount}{0}%
        \noindent\textbf{Exercício~\theexerciciocount.} #1\par\vspace{0.5em}%
      }

      \newcommand{\subexercicio}[1]{%
        \par\vspace{0.8em}%
        \refstepcounter{subexerciciocount}%
        \noindent\textbf{\theexerciciocount.\thesubexerciciocount.} #1\par\vspace{0.3em}%
      }

      \newcommand{\exercicioDesenvolvimento}[1]{\par\noindent #1\par}
      \newcommand{\option}[1]{%
        \par\refstepcounter{optioncount}%
        \noindent(\alph{optioncount}) #1%
      }
      \makeatother
    }%
  }%
}

% ========== IP-BASED TEST SYSTEM MACROS (v3.5) ==========
% Support for modular exercise inclusion with numbered headings
% Provide a boolean flag to control whether the automatic heading is shown
\makeatletter
\@ifundefined{showexerciciotitletrue}{%
    \newif\ifshowexerciciotitle
    \showexerciciotitletrue
}{}
\makeatother

% Override \exercicio to respect the \ifshowexerciciotitle flag
% When false, it prints only the content without automatic heading
\renewcommand{\exercicio}[1]{%
    \ifshowexerciciotitle
        \par\vspace{1.5em}%
        \refstepcounter{exerciciocount}%
        \setcounter{subexerciciocount}{0}%
        \setcounter{optioncount}{0}%
        \noindent\textbf{Exercício~\theexerciciocount.} #1\par\vspace{0.5em}%
    \else
        #1\par
    \fi
}

\pagestyle{fancy}
\fancyhf{}
\lhead{Módulo A9 - Funções de Crescimento}
\rhead{0 - revisoes}
\cfoot{\thepage}

\title{}
\author{}
\date{}

\begin{document}
\maketitle

\section*{0 - revisoes}

\subsection*{Tipos de Exercícios}
\begin{itemize}
  \item \textbf{Aplicações Práticas} --- Exercícios de aplicação de conceitos matemáticos básicos (proporcionalidade, percentagens) em contextos práticos do quotidiano.
  \item \textbf{Cálculo com Funções} --- Exercícios de cálculo de imagens, objetos e manipulação algébrica básica de funções.
  \item \textbf{Conceito de Função} --- Exercícios sobre o conceito fundamental de função, verificação de correspondências e identificação de funções.
\end{itemize}

\vspace{1em}

% Exercício 1: MAT_A9FUNCOE_0RX_APX_001.tex
% Exercise ID: MAT_A9FUNCOE_0RX_APX_001
% Module: Módulo A9 - Funções de Crescimento | Concept: Revisões de Crescimento | Type: Aplicações Práticas
% Difficulty: 1/5 (Muito Fácil) | Format: desenvolvimento
% Tags: crescimento, aplicacao_pratica, contexto_real, consolidacao, revisao, proporcionalidade, percentagens
% Author: Professor | Date: 2025-11-20
% Status: active

\exercicio{
Uma família de 4 pessoas consome 2 kg de arroz por semana. Quantos kg de arroz serão necessários para uma família de 6 pessoas, mantendo a mesma proporção?

\vspace{3cm}
}
\FloatBarrier

% Exercício 2: MAT_A9FUNCOE_0RX_APX_002.tex
% Exercise ID: MAT_A9FUNCOE_0RX_APX_002
% Module: Módulo A9 - Funções de Crescimento | Concept: Revisões de Crescimento | Type: Aplicações Práticas
% Difficulty: 5/5 (Muito Difícil) | Format: desenvolvimento
% Tags: crescimento, proporcionalidade, contexto_real, percentagens, revisao, consolidacao, aplicacao_pratica, teste, automacao
% Author: Teste Automático | Date: 2025-11-26
% Status: active

\exercicio{Exercício de teste para aplicacoes_praticas}

% Solution:
% \begin{solucao}
% Solução de exemplo
% \end{solucao}
\FloatBarrier

% Exercício 3: MAT_A9FUNCRE_0REV_APL_001.tex
% meta:
% id: MAT_A9FUNCRE_0REV_APL_101
% module: A9_funcoes_crescimento
% concept: 0-revisoes
% concept_name: Revisões de Crescimento
% tipo: aplicacoes_praticas
% tipo_nome: Aplicações Práticas
% difficulty: 2
% tags: revisao,proporcionalidade,aplicacao_pratica
% author: sistema
\exercicio{
Uma receita de bolo de iogurte para 2 pessoas usa:
\begin{itemize}
  \item 1 iogurte natural
  \item 2 copos de açúcar
  \item 3 copos de farinha
  \item $\frac{1}{2}$ copo de óleo
  \item 3 ovos
\end{itemize}
Pretende-se ajustar para 12 pessoas. Indica as quantidades proporcionais.

\vspace{3cm}
}
\FloatBarrier

% Exercício 4: MAT_A9FUNCRE_0REV_APL_002.tex
% meta:
% id: MAT_A9FUNCRE_0REV_APL_102
% module: A9_funcoes_crescimento
% concept: 0-revisoes
% concept_name: Revisões de Crescimento
% tipo: aplicacoes_praticas
% tipo_nome: Aplicações Práticas
% difficulty: 2
% tags: revisao,percentagens,aplicacao_pratica
% author: sistema
\exercicio{
O Clube A comprou um jogador por 13 milhões de euros, o que representa 18\% do seu orçamento anual. Determina o orçamento total e discute se a percentagem é sustentável face a uma segunda contratação igual.

\vspace{3cm}
}
\FloatBarrier

% Exercício 5: MAT_A9FUNCRES_0REV_003.tex
% meta:
% id: MAT_A9FUNCRE_0REV_APL_001
% module: A9_funcoes_crescimento
% concept: 0-revisoes
% concept_name: Revisões de Crescimento
% tipo: aplicacoes_praticas
% tipo_nome: Aplicações Práticas
% difficulty: 2
% tags: revisao,proporcionalidade,aplicacao_pratica
% author: sistema
\exercicio{
Uma receita de bolo de iogurte para 2 pessoas usa:
\begin{itemize}
  \item 1 iogurte natural
  \item 2 copos de açúcar
  \item 3 copos de farinha
  \item $\frac{1}{2}$ copo de óleo
  \item 3 ovos
\end{itemize}
Pretende-se ajustar para 10 pessoas. Indica as quantidades proporcionais.

\vspace{3cm}
}
\FloatBarrier

% Exercício 6: MAT_A9FUNCRES_0REV_004.tex
% meta:
% id: MAT_A9FUNCRE_0REV_APL_002
% module: A9_funcoes_crescimento
% concept: 0-revisoes
% concept_name: Revisões de Crescimento
% tipo: aplicacoes_praticas
% tipo_nome: Aplicações Práticas
% difficulty: 2
% tags: revisao,percentagens,aplicacao_pratica
% author: sistema
\exercicio{
O Clube A comprou um jogador por 12 milhões de euros, o que representa 20\% do seu orçamento anual. Determina o orçamento total e discute se a percentagem é sustentável face a uma segunda contratação igual.

\vspace{3cm}
}
\FloatBarrier

% Exercício 7: MAT_A9FUNCRE_0REV_CAL_001.tex
% meta:
% id: MAT_A9FUNCRE_0REV_CAL_103
% module: A9_funcoes_crescimento
% concept: 0-revisoes
% concept_name: Revisões de Crescimento
% tipo: calculo_funcoes
% tipo_nome: Cálculo com Funções
% difficulty: 2
% tags: revisao,funcoes,calculo_imagens
% author: sistema
\exercicio{
Considera a função $f(x)=2x-4$.
\begin{itemize}
  \item[a)] Calcula $f(0)$, $f(2)$ e $f(5)$.
  
  \vspace{3cm}
  
  \item[b)] Qual é o valor de $x$ tal que $f(x)=1$?
\end{itemize}

\vspace{3cm}
}
\FloatBarrier

% Exercício 8: MAT_A9FUNCRES_0REV_002.tex
% meta:
% id: MAT_A9FUNCRE_0REV_CAL_001
% module: A9_funcoes_crescimento
% concept: 0-revisoes
% concept_name: Revisões de Crescimento
% tipo: calculo_funcoes
% tipo_nome: Cálculo com Funções
% difficulty: 2
% tags: revisao,funcoes,calculo_imagens
% author: sistema
\exercicio{
Considera a função $f(x)=2x-3$.
\begin{itemize}
  \item[a)] Calcula $f(0)$, $f(2)$ e $f(5)$.
  \vspace{3cm}
  \item[b)] Qual é o valor de $x$ tal que $f(x)=1$?
\end{itemize}

\vspace{3cm}
}
\FloatBarrier

% Exercício 9: MAT_A9FUNCOE_0RX_CFX_001.tex
% Exercise ID: MAT_A9FUNCOE_0RX_CFX_001
% Module: Módulo A9 - Funções de Crescimento | Concept: Revisões de Crescimento | Type: Conceito de Função
% Difficulty: 1/5 (Muito Fácil) | Format: desenvolvimento
% Tags: correspondencia, definicao, conceito_funcao, consolidacao, crescimento, revisao, verificacao, teste, automacao
% Author: Teste Automático | Date: 2025-11-26
% Status: active

\exercicio{Exercício de teste para conceito_funcao}
\FloatBarrier

% Exercício 10: MAT_A9FUNCOE_0RX_CFX_002.tex
% Exercise ID: MAT_A9FUNCOE_0RX_CFX_002
% Module: Módulo A9 - Funções de Crescimento | Concept: Revisões de Crescimento | Type: Conceito de Função
% Difficulty: 1/5 (Muito Fácil) | Format: desenvolvimento
% Tags: conceito_funcao, consolidacao, revisao, crescimento, correspondencia, definicao, verificacao, teste, automacao
% Author: Teste Automático | Date: 2025-11-26
% Status: active

\exercicio{Exercício de teste para conceito_funcao}

% Solution:
% \begin{solucao}
% Solução de exemplo
% \end{solucao}
\FloatBarrier

% Exercício 11: MAT_A9FUNCOE_0RX_CFX_003.tex
% Exercise ID: MAT_A9FUNCOE_0RX_CFX_003
% Module: Módulo A9 - Funções de Crescimento | Concept: Revisões de Crescimento | Type: Conceito de Função
% Difficulty: 3/5 (Médio) | Format: desenvolvimento
% Tags: consolidacao, conceito_funcao, revisao, correspondencia, crescimento, verificacao, definicao, teste, automacao
% Author: Teste Automático | Date: 2025-11-26
% Status: active

\exercicio{Exercício de teste para conceito_funcao}

% Solution:
% \begin{solucao}
% Solução de exemplo
% \end{solucao}
\FloatBarrier

% Exercício 12: MAT_A9FUNCOE_0RX_CFX_004.tex
% Exercise ID: MAT_A9FUNCOE_0RX_CFX_004
% Module: Módulo A9 - Funções de Crescimento | Concept: Revisões de Crescimento | Type: Conceito de Função
% Difficulty: 5/5 (Muito Difícil) | Format: desenvolvimento
% Tags: verificacao, crescimento, definicao, conceito_funcao, consolidacao, revisao, correspondencia, teste, automacao
% Author: Teste Automático | Date: 2025-11-26
% Status: active

\exercicio{Exercício de teste para conceito_funcao}

% Solution:
% \begin{solucao}
% Solução de exemplo
% \end{solucao}
\FloatBarrier

% Exercício 13: MAT_A9FUNCRE_0REV_CFU_001.tex
% meta:
% id: MAT_A9FUNCRE_0REV_CFU_001
% module: A9_funcoes_crescimento
% concept: 0-revisoes
% concept_name: Revisões de Crescimento
% tipo: conceito_funcao
% tipo_nome: Conceito de Função
% difficulty: 2
% tags: revisao,funcoes,conceito_funcao
% author: sistema
\exercicio{
Considera a correspondência seguinte entre pessoas e o número de sapatos que calçam:
\[
\begin{tabular}{|c|c|}
\hline
\text{Pessoa} & \text{Número de sapatos que calça} \\
\hline
\text{Ana} & 37 \\
\text{Bruno} & 42 \\
\text{Carla} & 39 \\
\text{David} & 42 \\
\hline
\end{tabular}
\]
\textbf{Pergunta:} Esta correspondência é uma função? Justifica escolhendo a opção correta e explicando por que as outras estão erradas.
\begin{enumerate}
 \item[(A)] Não é uma função, porque o número 42 aparece duas vezes.
 \item[(B)] É uma função, porque cada pessoa está associada a um único número.
 \item[(C)] Não é uma função, porque Bruno e David calçam o mesmo número.
 \item[(D)] Não é uma função, porque há números repetidos na segunda coluna.
\end{enumerate}

}
\FloatBarrier

% Exercício 14: MAT_A9FUNCRES_0REV_001.tex
% meta:
% id: MAT_A9FUNCRE_0REV_CFU_001
% module: A9_funcoes_crescimento
% concept: 0-revisoes
% concept_name: Revisões de Crescimento
% tipo: conceito_funcao
% tipo_nome: Conceito de Função
% difficulty: 2
% tags: revisao,funcoes,conceito_funcao
% author: sistema
\exercicio{
Considera a correspondência seguinte entre pessoas e o número de sapatos que calçam:
\[
\begin{array}{c|c}
\text{Pessoa} & \text{Número de sapatos que calça} \\
\hline
\text{Ana} & 50 \\
\text{Bruno} & 20 \\
\text{Carla} & 39 \\
\text{David} & 42 \\
\end{array}
\]
\textbf{Pergunta:} Esta correspondência é uma função? Justifica escolhendo a opção correta e explicando por que as outras estão erradas.
\begin{enumerate}
 \item[(A)] Não é uma função, porque o número 42 aparece duas vezes.
 \item[(B)] É uma função, porque cada pessoa está associada a um único número.
 \item[(C)] Não é uma função, porque Bruno e David calçam o mesmo número.
 \item[(D)] Não é uma função, porque há números repetidos na segunda coluna.
\end{enumerate}
}
\FloatBarrier

\end{document}
