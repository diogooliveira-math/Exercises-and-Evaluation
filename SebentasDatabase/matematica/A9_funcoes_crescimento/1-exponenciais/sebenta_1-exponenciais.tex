% Minimal sebenta template generated automatically
\documentclass[11pt,a4paper]{article}
\usepackage[utf8]{inputenc}
\usepackage[T1]{fontenc}
\usepackage{lmodern}
\usepackage{geometry}
\usepackage{fancyhdr}
\usepackage{hyperref}
\usepackage{graphicx}
\usepackage{float}
\usepackage{placeins}
\usepackage{bookmark}
\usepackage{booktabs}
\usepackage{amsmath,amssymb}
\usepackage{csquotes}
\usepackage{enumitem}
\usepackage{tikz}
\IfFileExists{pgfplots.sty}{\usepackage{pgfplots}\pgfplotsset{compat=1.17}}{}

\geometry{margin=2.5cm}

% Try to include project-specific style macros (containing \exercicio, \subexercicio, etc.)
% Try multiple relative locations to be robust across different generated output paths
\IfFileExists{../../../../Teste_modelo/config/style.tex}{% Sistema de exercícios com contadores automáticos
\newcounter{exerciciocount}          % Contador principal dos exercícios
\newcounter{subexerciciocount}       % Contador dos subexercícios
\newcounter{optioncount}             % Contador das opções

% Control whether the macro prints the automatic "Exercício N." heading.
% Default: show the heading. Call \showexerciciotitlefalse to suppress.
\newif\ifshowexerciciotitle
\showexerciciotitletrue

% Macro para exercício principal
\newcommand{\exercicio}[1]{%
        \par\vspace{1.5em}% Espaçamento antes
        \refstepcounter{exerciciocount}% Incrementa contador principal
        \setcounter{subexerciciocount}{0}% Reseta contador de subexercícios
        \setcounter{optioncount}{0}% Reseta contador de opções
        % Only print the automatic heading if the flag is true
        \ifshowexerciciotitle
            \noindent\textbf{Exercício~\theexerciciocount.}\space #1\par\vspace{0.5em}%
        \else
            % When suppressed, just print the content without the heading
            #1\par\vspace{0.5em}%
        \fi
}

% Macro para subexercício
\newcommand{\subexercicio}[1]{%
    \par\vspace{0.8em}% Espaçamento menor para subexercícios
    \refstepcounter{subexerciciocount}% Incrementa contador de subexercícios
    \noindent\textbf{\theexerciciocount.\thesubexerciciocount.} #1\par\vspace{0.3em}%
}

% Macro para opção
\newcommand{\option}[1]{%
    \par
    \refstepcounter{optioncount}%
    \noindent(\alph{optioncount}) #1%
}

% Título e informações do exame
\title{1ª Questão de aula do Módulo A10: Otimização}
\author{EPRALIMA - Escola Profissional Alto Lima}

\date{}

% Cabeçalho completo do teste dentro de uma caixa simples
\newcommand{\espacoAluno}{%
    \vspace{0.5cm}
    \fbox{%
        \parbox{\textwidth}{%
            \noindent\textbf{Nome do Aluno:} \underline{\hspace{7cm}} \textbf{Turma:} \underline{\hspace{1cm}}\\[0.5cm]
            \noindent\textbf{Assinatura do Professor:} \underline{\hspace{3cm}} \hfill \textbf{Nota:} \underline{\hspace{2cm}}\\[0.5cm]
            \noindent\textbf{Assinatura do Encarregado de Educação:} \underline{\hspace{3cm}}
        }%
    }
    \vspace{1cm}
}}{%
  \IfFileExists{../../../Teste_modelo/config/style.tex}{% Sistema de exercícios com contadores automáticos
\newcounter{exerciciocount}          % Contador principal dos exercícios
\newcounter{subexerciciocount}       % Contador dos subexercícios
\newcounter{optioncount}             % Contador das opções

% Control whether the macro prints the automatic "Exercício N." heading.
% Default: show the heading. Call \showexerciciotitlefalse to suppress.
\newif\ifshowexerciciotitle
\showexerciciotitletrue

% Macro para exercício principal
\newcommand{\exercicio}[1]{%
        \par\vspace{1.5em}% Espaçamento antes
        \refstepcounter{exerciciocount}% Incrementa contador principal
        \setcounter{subexerciciocount}{0}% Reseta contador de subexercícios
        \setcounter{optioncount}{0}% Reseta contador de opções
        % Only print the automatic heading if the flag is true
        \ifshowexerciciotitle
            \noindent\textbf{Exercício~\theexerciciocount.}\space #1\par\vspace{0.5em}%
        \else
            % When suppressed, just print the content without the heading
            #1\par\vspace{0.5em}%
        \fi
}

% Macro para subexercício
\newcommand{\subexercicio}[1]{%
    \par\vspace{0.8em}% Espaçamento menor para subexercícios
    \refstepcounter{subexerciciocount}% Incrementa contador de subexercícios
    \noindent\textbf{\theexerciciocount.\thesubexerciciocount.} #1\par\vspace{0.3em}%
}

% Macro para opção
\newcommand{\option}[1]{%
    \par
    \refstepcounter{optioncount}%
    \noindent(\alph{optioncount}) #1%
}

% Título e informações do exame
\title{1ª Questão de aula do Módulo A10: Otimização}
\author{EPRALIMA - Escola Profissional Alto Lima}

\date{}

% Cabeçalho completo do teste dentro de uma caixa simples
\newcommand{\espacoAluno}{%
    \vspace{0.5cm}
    \fbox{%
        \parbox{\textwidth}{%
            \noindent\textbf{Nome do Aluno:} \underline{\hspace{7cm}} \textbf{Turma:} \underline{\hspace{1cm}}\\[0.5cm]
            \noindent\textbf{Assinatura do Professor:} \underline{\hspace{3cm}} \hfill \textbf{Nota:} \underline{\hspace{2cm}}\\[0.5cm]
            \noindent\textbf{Assinatura do Encarregado de Educação:} \underline{\hspace{3cm}}
        }%
    }
    \vspace{1cm}
}}{%
    \IfFileExists{../../Teste_modelo/config/style.tex}{% Sistema de exercícios com contadores automáticos
\newcounter{exerciciocount}          % Contador principal dos exercícios
\newcounter{subexerciciocount}       % Contador dos subexercícios
\newcounter{optioncount}             % Contador das opções

% Control whether the macro prints the automatic "Exercício N." heading.
% Default: show the heading. Call \showexerciciotitlefalse to suppress.
\newif\ifshowexerciciotitle
\showexerciciotitletrue

% Macro para exercício principal
\newcommand{\exercicio}[1]{%
        \par\vspace{1.5em}% Espaçamento antes
        \refstepcounter{exerciciocount}% Incrementa contador principal
        \setcounter{subexerciciocount}{0}% Reseta contador de subexercícios
        \setcounter{optioncount}{0}% Reseta contador de opções
        % Only print the automatic heading if the flag is true
        \ifshowexerciciotitle
            \noindent\textbf{Exercício~\theexerciciocount.}\space #1\par\vspace{0.5em}%
        \else
            % When suppressed, just print the content without the heading
            #1\par\vspace{0.5em}%
        \fi
}

% Macro para subexercício
\newcommand{\subexercicio}[1]{%
    \par\vspace{0.8em}% Espaçamento menor para subexercícios
    \refstepcounter{subexerciciocount}% Incrementa contador de subexercícios
    \noindent\textbf{\theexerciciocount.\thesubexerciciocount.} #1\par\vspace{0.3em}%
}

% Macro para opção
\newcommand{\option}[1]{%
    \par
    \refstepcounter{optioncount}%
    \noindent(\alph{optioncount}) #1%
}

% Título e informações do exame
\title{1ª Questão de aula do Módulo A10: Otimização}
\author{EPRALIMA - Escola Profissional Alto Lima}

\date{}

% Cabeçalho completo do teste dentro de uma caixa simples
\newcommand{\espacoAluno}{%
    \vspace{0.5cm}
    \fbox{%
        \parbox{\textwidth}{%
            \noindent\textbf{Nome do Aluno:} \underline{\hspace{7cm}} \textbf{Turma:} \underline{\hspace{1cm}}\\[0.5cm]
            \noindent\textbf{Assinatura do Professor:} \underline{\hspace{3cm}} \hfill \textbf{Nota:} \underline{\hspace{2cm}}\\[0.5cm]
            \noindent\textbf{Assinatura do Encarregado de Educação:} \underline{\hspace{3cm}}
        }%
    }
    \vspace{1cm}
}}{%
      % style.tex not found - proceed without project macros
    }%
  }%
}

% Provide a robust fallback for macros that might be missing in style.tex
% This attempts to include the project style first (multiple relative paths),
% and only if none exist defines minimal counters and macros safely.
\IfFileExists{../../../../Teste_modelo/config/style.tex}{% Sistema de exercícios com contadores automáticos
\newcounter{exerciciocount}          % Contador principal dos exercícios
\newcounter{subexerciciocount}       % Contador dos subexercícios
\newcounter{optioncount}             % Contador das opções

% Control whether the macro prints the automatic "Exercício N." heading.
% Default: show the heading. Call \showexerciciotitlefalse to suppress.
\newif\ifshowexerciciotitle
\showexerciciotitletrue

% Macro para exercício principal
\newcommand{\exercicio}[1]{%
        \par\vspace{1.5em}% Espaçamento antes
        \refstepcounter{exerciciocount}% Incrementa contador principal
        \setcounter{subexerciciocount}{0}% Reseta contador de subexercícios
        \setcounter{optioncount}{0}% Reseta contador de opções
        % Only print the automatic heading if the flag is true
        \ifshowexerciciotitle
            \noindent\textbf{Exercício~\theexerciciocount.}\space #1\par\vspace{0.5em}%
        \else
            % When suppressed, just print the content without the heading
            #1\par\vspace{0.5em}%
        \fi
}

% Macro para subexercício
\newcommand{\subexercicio}[1]{%
    \par\vspace{0.8em}% Espaçamento menor para subexercícios
    \refstepcounter{subexerciciocount}% Incrementa contador de subexercícios
    \noindent\textbf{\theexerciciocount.\thesubexerciciocount.} #1\par\vspace{0.3em}%
}

% Macro para opção
\newcommand{\option}[1]{%
    \par
    \refstepcounter{optioncount}%
    \noindent(\alph{optioncount}) #1%
}

% Título e informações do exame
\title{1ª Questão de aula do Módulo A10: Otimização}
\author{EPRALIMA - Escola Profissional Alto Lima}

\date{}

% Cabeçalho completo do teste dentro de uma caixa simples
\newcommand{\espacoAluno}{%
    \vspace{0.5cm}
    \fbox{%
        \parbox{\textwidth}{%
            \noindent\textbf{Nome do Aluno:} \underline{\hspace{7cm}} \textbf{Turma:} \underline{\hspace{1cm}}\\[0.5cm]
            \noindent\textbf{Assinatura do Professor:} \underline{\hspace{3cm}} \hfill \textbf{Nota:} \underline{\hspace{2cm}}\\[0.5cm]
            \noindent\textbf{Assinatura do Encarregado de Educação:} \underline{\hspace{3cm}}
        }%
    }
    \vspace{1cm}
}}{%
  \IfFileExists{../../../Teste_modelo/config/style.tex}{% Sistema de exercícios com contadores automáticos
\newcounter{exerciciocount}          % Contador principal dos exercícios
\newcounter{subexerciciocount}       % Contador dos subexercícios
\newcounter{optioncount}             % Contador das opções

% Control whether the macro prints the automatic "Exercício N." heading.
% Default: show the heading. Call \showexerciciotitlefalse to suppress.
\newif\ifshowexerciciotitle
\showexerciciotitletrue

% Macro para exercício principal
\newcommand{\exercicio}[1]{%
        \par\vspace{1.5em}% Espaçamento antes
        \refstepcounter{exerciciocount}% Incrementa contador principal
        \setcounter{subexerciciocount}{0}% Reseta contador de subexercícios
        \setcounter{optioncount}{0}% Reseta contador de opções
        % Only print the automatic heading if the flag is true
        \ifshowexerciciotitle
            \noindent\textbf{Exercício~\theexerciciocount.}\space #1\par\vspace{0.5em}%
        \else
            % When suppressed, just print the content without the heading
            #1\par\vspace{0.5em}%
        \fi
}

% Macro para subexercício
\newcommand{\subexercicio}[1]{%
    \par\vspace{0.8em}% Espaçamento menor para subexercícios
    \refstepcounter{subexerciciocount}% Incrementa contador de subexercícios
    \noindent\textbf{\theexerciciocount.\thesubexerciciocount.} #1\par\vspace{0.3em}%
}

% Macro para opção
\newcommand{\option}[1]{%
    \par
    \refstepcounter{optioncount}%
    \noindent(\alph{optioncount}) #1%
}

% Título e informações do exame
\title{1ª Questão de aula do Módulo A10: Otimização}
\author{EPRALIMA - Escola Profissional Alto Lima}

\date{}

% Cabeçalho completo do teste dentro de uma caixa simples
\newcommand{\espacoAluno}{%
    \vspace{0.5cm}
    \fbox{%
        \parbox{\textwidth}{%
            \noindent\textbf{Nome do Aluno:} \underline{\hspace{7cm}} \textbf{Turma:} \underline{\hspace{1cm}}\\[0.5cm]
            \noindent\textbf{Assinatura do Professor:} \underline{\hspace{3cm}} \hfill \textbf{Nota:} \underline{\hspace{2cm}}\\[0.5cm]
            \noindent\textbf{Assinatura do Encarregado de Educação:} \underline{\hspace{3cm}}
        }%
    }
    \vspace{1cm}
}}{%
    \IfFileExists{../../Teste_modelo/config/style.tex}{% Sistema de exercícios com contadores automáticos
\newcounter{exerciciocount}          % Contador principal dos exercícios
\newcounter{subexerciciocount}       % Contador dos subexercícios
\newcounter{optioncount}             % Contador das opções

% Control whether the macro prints the automatic "Exercício N." heading.
% Default: show the heading. Call \showexerciciotitlefalse to suppress.
\newif\ifshowexerciciotitle
\showexerciciotitletrue

% Macro para exercício principal
\newcommand{\exercicio}[1]{%
        \par\vspace{1.5em}% Espaçamento antes
        \refstepcounter{exerciciocount}% Incrementa contador principal
        \setcounter{subexerciciocount}{0}% Reseta contador de subexercícios
        \setcounter{optioncount}{0}% Reseta contador de opções
        % Only print the automatic heading if the flag is true
        \ifshowexerciciotitle
            \noindent\textbf{Exercício~\theexerciciocount.}\space #1\par\vspace{0.5em}%
        \else
            % When suppressed, just print the content without the heading
            #1\par\vspace{0.5em}%
        \fi
}

% Macro para subexercício
\newcommand{\subexercicio}[1]{%
    \par\vspace{0.8em}% Espaçamento menor para subexercícios
    \refstepcounter{subexerciciocount}% Incrementa contador de subexercícios
    \noindent\textbf{\theexerciciocount.\thesubexerciciocount.} #1\par\vspace{0.3em}%
}

% Macro para opção
\newcommand{\option}[1]{%
    \par
    \refstepcounter{optioncount}%
    \noindent(\alph{optioncount}) #1%
}

% Título e informações do exame
\title{1ª Questão de aula do Módulo A10: Otimização}
\author{EPRALIMA - Escola Profissional Alto Lima}

\date{}

% Cabeçalho completo do teste dentro de uma caixa simples
\newcommand{\espacoAluno}{%
    \vspace{0.5cm}
    \fbox{%
        \parbox{\textwidth}{%
            \noindent\textbf{Nome do Aluno:} \underline{\hspace{7cm}} \textbf{Turma:} \underline{\hspace{1cm}}\\[0.5cm]
            \noindent\textbf{Assinatura do Professor:} \underline{\hspace{3cm}} \hfill \textbf{Nota:} \underline{\hspace{2cm}}\\[0.5cm]
            \noindent\textbf{Assinatura do Encarregado de Educação:} \underline{\hspace{3cm}}
        }%
    }
    \vspace{1cm}
}}{%
      % style.tex not found - define minimal counters/macros defensively
      \makeatletter
      \@ifundefined{exerciciocount}{\newcounter{exerciciocount}}{}
      \@ifundefined{subexerciciocount}{\newcounter{subexerciciocount}}{}
      \@ifundefined{optioncount}{\newcounter{optioncount}}{}

      \newcommand{\exercicio}[1]{%
        \par\vspace{1.5em}%
        \refstepcounter{exerciciocount}%
        \setcounter{subexerciciocount}{0}%
        \setcounter{optioncount}{0}%
        \noindent\textbf{Exercício~\theexerciciocount.} #1\par\vspace{0.5em}%
      }

      \newcommand{\subexercicio}[1]{%
        \par\vspace{0.8em}%
        \refstepcounter{subexerciciocount}%
        \noindent\textbf{\theexerciciocount.\thesubexerciciocount.} #1\par\vspace{0.3em}%
      }

      \newcommand{\exercicioDesenvolvimento}[1]{\par\noindent #1\par}
      \newcommand{\option}[1]{%
        \par\refstepcounter{optioncount}%
        \noindent(\alph{optioncount}) #1%
      }
      \makeatother
    }%
  }%
}

% ========== IP-BASED TEST SYSTEM MACROS (v3.5) ==========
% Support for modular exercise inclusion with numbered headings
% Provide a boolean flag to control whether the automatic heading is shown
\makeatletter
\@ifundefined{showexerciciotitletrue}{%
    \newif\ifshowexerciciotitle
    \showexerciciotitletrue
}{}
\makeatother

% Override \exercicio to respect the \ifshowexerciciotitle flag
% When false, it prints only the content without automatic heading
\renewcommand{\exercicio}[1]{%
    \ifshowexerciciotitle
        \par\vspace{1.5em}%
        \refstepcounter{exerciciocount}%
        \setcounter{subexerciciocount}{0}%
        \setcounter{optioncount}{0}%
        \noindent\textbf{Exercício~\theexerciciocount.} #1\par\vspace{0.5em}%
    \else
        #1\par
    \fi
}

\pagestyle{fancy}
\fancyhf{}
\lhead{Módulo A9 - Funções de Crescimento}
\rhead{1 - exponenciais}
\cfoot{\thepage}

\title{}
\author{}
\date{}

\begin{document}
\maketitle

\section*{1 - exponenciais}

\subsection*{Tipos de Exercícios}
\begin{itemize}
  \item \textbf{Crescimento de Bactérias} --- Modelação de crescimento bacteriano usando funções exponenciais, incluindo modelos com base e e expoentes fracionários.
  \item \textbf{Crescimento Populacional} --- Modelação de crescimento de populações através de funções exponenciais. Inclui interpretação gráfica e resolução de problemas contextualizados.
  \item \textbf{Decaimento de Medicamento} --- Modelação de decaimento de concentração de medicamentos no organismo através de funções exponenciais.
  \item \textbf{Depreciação de Valor} --- Modelação de depreciação de valor comercial de bens através de funções exponenciais decrescentes.
\end{itemize}

\vspace{1em}

% Exercício 1: MAT_A9FUNCRE_1EXP_CBA_001.tex
% meta:
% id: MAT_A9FUNCRE_1EXP_CBA_105
% module: A9_funcoes_crescimento
% concept: 1-exponenciais
% concept_name: Funções Exponenciais
% tipo: crescimento_bacterias
% tipo_nome: Crescimento de Bactérias
% difficulty: 3
% tags: crescimento_exponencial,bacterias,modelo_continuo,exponenciais
% author: sistema
\exercicio{
Uma cultura segue $n(t)=n_0 e^{kt}$. Inicialmente 400 bactérias; após 2 h triplicou.
\begin{enumerate}
  \item Determina $n_0$ e $k$.
  
\vspace{3cm}

  \item Estima $n(4)$.
  
  \vspace{3cm}
  
  \item Esboça o gráfico e interpreta três pontos.
  
    \begin{center}
  \begin{tikzpicture}[scale=0.8]
    \draw[->] (-0.5,0) -- (10,0) node[right] {$t$};
    \draw[->] (0,-0.5) -- (0,6) node[above] {$C(t)$};
    \foreach \x in {1,2,...,9}
      \draw (\x,0.1) -- (\x,-0.1);
    \foreach \y in {1,2,...,5}
      \draw (0.1,\y) -- (-0.1,\y);
  \end{tikzpicture}
  \end{center}

  \item Cria um novo problema semelhante e resolve.
  
  \vspace{3cm}
\end{enumerate}

}
\FloatBarrier

% Exercício 2: MAT_A9FUNCRE_1EXP_CBA_002.tex
% meta:
% id: MAT_A9FUNCRE_1EXP_CBA_106
% module: A9_funcoes_crescimento
% concept: 1-exponenciais
% concept_name: Funções Exponenciais
% tipo: crescimento_bacterias
% tipo_nome: Crescimento de Bactérias
% difficulty: 3
% tags: crescimento_exponencial,bacterias,exponenciais,modelacao
% author: sistema
\exercicio{
Número de bactérias: $f(t)=300\times 4^{t/3}$ para $t$ horas após 6h.
\begin{enumerate}
  \item Calcula número às 7h e às 10h.
  
  \vspace{3cm}
  \item Determina tempo até 4800 bactérias (horas e minutos).
  \vspace{3cm}
  \item Esboça o gráfico e interpreta três pontos.
    \begin{center}
  \begin{tikzpicture}[scale=0.8]
    \draw[->] (-0.5,0) -- (10,0) node[right] {$t$};
    \draw[->] (0,-0.5) -- (0,6) node[above] {$C(t)$};
    \foreach \x in {1,2,...,9}
      \draw (\x,0.1) -- (\x,-0.1);
    \foreach \y in {1,2,...,5}
      \draw (0.1,\y) -- (-0.1,\y);
  \end{tikzpicture}
  \end{center}
  \item Cria um problema adicional de crescimento e resolve.
  \vspace{3cm}
\end{enumerate}

\vspace{3cm}
}
\FloatBarrier

% Exercício 3: MAT_A9FUNCRES_1EXP_003.tex
% meta:
% id: MAT_A9FUNCRE_1EXP_CBA_001
% module: A9_funcoes_crescimento
% concept: 1-exponenciais
% concept_name: Funções Exponenciais
% tipo: crescimento_bacterias
% tipo_nome: Crescimento de Bactérias
% difficulty: 3
% tags: crescimento_exponencial,bacterias,modelo_continuo,exponenciais
% author: sistema
\exercicio{
Uma cultura segue $n(t)=n_0 e^{kt}$. Inicialmente 800 bactérias; após 4 h quadruplicou.
\begin{enumerate}
  \item Determina $n_0$ e $k$.
  \vspace{3cm}
  \item Estima $n(6)$.
  \vspace{3cm}
  \item Esboça o gráfico e interpreta três pontos.
    \begin{center}
  \begin{tikzpicture}[scale=0.8]
    \draw[->] (-0.5,0) -- (10,0) node[right] {$t$};
    \draw[->] (0,-0.5) -- (0,6) node[above] {$C(t)$};
    \foreach \x in {1,2,...,9}
      \draw (\x,0.1) -- (\x,-0.1);
    \foreach \y in {1,2,...,5}
      \draw (0.1,\y) -- (-0.1,\y);
  \end{tikzpicture}
  \end{center}
  \item Cria um novo problema semelhante e resolve.
  \vspace{3cm}
\end{enumerate}

\vspace{3cm}
}
\FloatBarrier

% Exercício 4: MAT_A9FUNCRES_1EXP_005.tex
% meta:
% id: MAT_A9FUNCRE_1EXP_CBA_002
% module: A9_funcoes_crescimento
% concept: 1-exponenciais
% concept_name: Funções Exponenciais
% tipo: crescimento_bacterias
% tipo_nome: Crescimento de Bactérias
% difficulty: 3
% tags: crescimento_exponencial,bacterias,exponenciais,modelacao
% author: sistema
\exercicio{
Número de bactérias: $f(t)=600\times 5^{t/4}$ para $t$ horas após 8h.
\begin{enumerate}
  \item Calcula número às 9h e às 12h.
  \vspace{3cm}
  \item Determina tempo até 15000 bactérias (horas e minutos).
  \vspace{3cm}
  \item Esboça o gráfico e interpreta três pontos.
  \vspace{3cm}
  \item Cria um problema adicional de crescimento e resolve.
  \vspace{3cm}
\end{enumerate}

\vspace{3cm}
}
\FloatBarrier

% Exercício 5: MAT_A9FUNCRE_1EXP_CPO_001.tex
% meta:
% id: MAT_A9FUNCRE_1EXP_CPO_107
% module: A9_funcoes_crescimento
% concept: 1-exponenciais
% concept_name: Funções Exponenciais
% tipo: crescimento_populacional
% tipo_nome: Crescimento Populacional
% difficulty: 3
% tags: crescimento_exponencial,modelacao,exponenciais,populacao
% author: sistema
\exercicio{
Numa reserva natural, a população de coelhos evolui segundo: $C(t)=520\times 1,1^t$, onde $t$ (anos) desde 1998.
\textbf{Tarefas:}
\begin{enumerate}
  \item Representa graficamente $C(t)$ (esboço qualitativo).
  
  \begin{center}
  \begin{tikzpicture}[scale=0.8]
    \draw[->] (-0.5,0) -- (10,0) node[right] {$t$};
    \draw[->] (0,-0.5) -- (0,6) node[above] {$C(t)$};
    \foreach \x in {1,2,...,9}
      \draw (\x,0.1) -- (\x,-0.1);
    \foreach \y in {1,2,...,5}
      \draw (0.1,\y) -- (-0.1,\y);
  \end{tikzpicture}
  \end{center}
  
  \item Indica se o gráfico é crescente ou decrescente e interpreta.
  \vspace{3cm}
  \item Escolhe três pontos e explica o significado.
  \vspace{3cm}
  {\renewcommand{\labelenumii}{(\alph{enumii})}
  \begin{enumerate}
    \item Determina \(C(0)\) e \(C(1)\) e interpreta os resultados.
    \vspace{2cm}
    \item Calcula a taxa de crescimento relativa anual e comenta a sua implicação para a população.
    \vspace{2cm}
  \end{enumerate}
  }
  \vspace{3cm}
  \item Cria um problema novo de crescimento exponencial e resolve-o.
  \vspace{3cm}
\end{enumerate}
}
\FloatBarrier

% Exercício 6: MAT_A9FUNCRES_1EXP_001.tex
% meta:
% id: MAT_A9FUNCRE_1EXP_CPO_001
% module: A9_funcoes_crescimento
% concept: 1-exponenciais
% concept_name: Funções Exponenciais
% tipo: crescimento_populacional
% tipo_nome: Crescimento Populacional
% difficulty: 3
% tags: crescimento_exponencial,modelacao,exponenciais,populacao
% author: sistema
\exercicio{
Numa reserva natural, a população de veados evolui segundo: $D(t)=300\times 1,2^t$, onde $t$ (anos) desde 2000.
\textbf{Tarefas:}
\begin{enumerate}
  \item Representa graficamente $D(t)$ (esboço qualitativo).
  
  \begin{center}
  \begin{tikzpicture}[scale=0.8]
    \draw[->] (-0.5,0) -- (10,0) node[right] {$t$};
    \draw[->] (0,-0.5) -- (0,6) node[above] {$D(t)$};
    \foreach \x in {1,2,...,9}
      \draw (\x,0.1) -- (\x,-0.1);
    \foreach \y in {1,2,...,5}
      \draw (0.1,\y) -- (-0.1,\y);
  \end{tikzpicture}
  \end{center}
  
  \item Indica se o gráfico é crescente ou decrescente e interpreta.
  \vspace{3cm}
  \item Escolhe três pontos e explica o significado.
  \vspace{3cm}
  \item Calcula:
  {\renewcommand{\labelenumii}{(\alph{enumenumii})}
  \begin{enumerate}
    \item Determina \(D(0)\).
    \vspace{2cm}
    \item Calcula o ano em que se atingem 800 veados (ano e mês).
    \vspace{2cm}
  \end{enumerate}
  }
  \vspace{3cm}
  \item Cria um problema novo de crescimento exponencial e resolve-o.
  \vspace{3cm}
\end{enumerate}
}
\FloatBarrier

% Exercício 7: MAT_A9FUNCOE_1EX_DMX_001.tex
% Exercise ID: MAT_A9FUNCOE_1EX_DMX_001
% Module: Módulo A9 - Funções de Crescimento | Concept: Funções Exponenciais | Type: Decaimento de Medicamento
% Difficulty: 1/5 (Muito Fácil) | Format: desenvolvimento
% Tags: decaimento, medicamento, farmacologia, crescimento, decaimento_exponencial, modelacao, exponencial, opencode_test, terminal
% Author: opencode-terminal-tester | Date: 2025-11-26
% Status: active

\exercicio{Exercício de teste para decaimento_medicamento}

% Solution:
% \begin{solucao}
% Solução exemplo
% \end{solucao}
\FloatBarrier

% Exercício 8: MAT_A9FUNCRE_1EXP_DME_001.tex
% meta:
% id: MAT_A9FUNCRE_1EXP_DME_108
% module: A9_funcoes_crescimento
% concept: 1-exponenciais
% concept_name: Funções Exponenciais
% tipo: decaimento_medicamento
% tipo_nome: Decaimento de Medicamento
% difficulty: 3
% tags: decaimento_exponencial,medicamento,exponenciais,modelacao
% author: sistema
\exercicio{
A concentração de um medicamento: $C(t)=480\times 0,987^t$ (mg/cm$^3$) após $t$ minutos.
\begin{enumerate}
  \item Esboça o gráfico e descreve a tendência.
    \begin{center}
  \begin{tikzpicture}[scale=0.8]
    \draw[->] (-0.5,0) -- (10,0) node[right] {$t$};
    \draw[->] (0,-0.5) -- (0,6) node[above] {$C(t)$};
    \foreach \x in {1,2,...,9}
      \draw (\x,0.1) -- (\x,-0.1);
    \foreach \y in {1,2,...,5}
      \draw (0.1,\y) -- (-0.1,\y);
  \end{tikzpicture}
  \end{center}
  \item Interpreta três pontos do gráfico.
  \vspace{3cm}
  \item Cria e resolve um novo problema de decaimento exponencial.
  \vspace{3cm}
\end{enumerate}
}
\FloatBarrier

% Exercício 9: MAT_A9FUNCRES_1EXP_002.tex
% meta:
% id: MAT_A9FUNCRE_1EXP_DME_001
% module: A9_funcoes_crescimento
% concept: 1-exponenciais
% concept_name: Funções Exponenciais
% tipo: decaimento_medicamento
% tipo_nome: Decaimento de Medicamento
% difficulty: 3
% tags: decaimento_exponencial,medicamento,exponenciais,modelacao
% author: sistema
\exercicio{
A concentração de um medicamento: $C(t)=600\times 0,95^t$ (mg/cm$^3$) após $t$ minutos.
\begin{enumerate}
  \item Esboça o gráfico e descreve a tendência.
    \begin{center}
  \begin{tikzpicture}[scale=0.8]
    \draw[->] (-0.5,0) -- (10,0) node[right] {$t$};
    \draw[->] (0,-0.5) -- (0,6) node[above] {$C(t)$};
    \foreach \x in {1,2,...,9}
      \draw (\x,0.1) -- (\x,-0.1);
    \foreach \y in {1,2,...,5}
      \draw (0.1,\y) -- (-0.1,\y);
  \end{tikzpicture}
  \end{center}
  \item Interpreta três pontos do gráfico.
  \vspace{3cm}
  \item Cria e resolve um novo problema de decaimento exponencial.
  \vspace{3cm}
\end{enumerate}
}
\FloatBarrier

% Exercício 10: MAT_A9FUNCOE_1EX_DVX_001.tex
% Exercise ID: MAT_A9FUNCOE_1EX_DVX_001
% Module: Módulo A9 - Funções de Crescimento | Concept: Funções Exponenciais | Type: Depreciação de Valor
% Difficulty: 4/5 (Difícil) | Format: desenvolvimento
% Tags: financeiro, depreciacao, decaimento_exponencial, decaimento, crescimento, exponencial, modelacao, valor_comercial, teste, automacao
% Author: Teste Automático | Date: 2025-11-26
% Status: active

\exercicio{Exercício de teste para depreciacao_valor}

% Solution:
% \begin{solucao}
% Solução de exemplo
% \end{solucao}
\FloatBarrier

% Exercício 11: MAT_A9FUNCRE_1EXP_DPR_001.tex
% meta:
% id: MAT_A9FUNCRE_1EXP_DPR_109
% module: A9_funcoes_crescimento
% concept: 1-exponenciais
% concept_name: Funções Exponenciais
% tipo: depreciacao_valor
% tipo_nome: Depreciação de Valor
% difficulty: 3
% tags: decaimento_exponencial,valor_comercial,financeiro,exponenciais
% author: sistema
\exercicio{
Valor de um automóvel: $f(t)=20500(0,81)^t$ (euros) após $t$ anos.
\begin{enumerate}
  \item Estima valor aos 15 meses.
  \vspace{3cm}
  \item Determina quando $f(t)<5000$.
  \vspace{3cm}
  \item Calcula percentagem de desvalorização anual.
  \vspace{3cm}
  \item Esboça o gráfico e interpreta três pontos.
    \begin{center}
  \begin{tikzpicture}[scale=0.8]
    \draw[->] (-0.5,0) -- (10,0) node[right] {$t$};
    \draw[->] (0,-0.5) -- (0,6) node[above] {$C(t)$};
    \foreach \x in {1,2,...,9}
      \draw (\x,0.1) -- (\x,-0.1);
    \foreach \y in {1,2,...,5}
      \draw (0.1,\y) -- (-0.1,\y);
  \end{tikzpicture}
  \end{center}
\end{enumerate}
}
\FloatBarrier

% Exercício 12: MAT_A9FUNCRES_1EXP_004.tex
% meta:
% id: MAT_A9FUNCRE_1EXP_DPR_001
% module: A9_funcoes_crescimento
% concept: 1-exponenciais
% concept_name: Funções Exponenciais
% tipo: depreciacao_valor
% tipo_nome: Depreciação de Valor
% difficulty: 3
% tags: decaimento_exponencial,valor_comercial,financeiro,exponenciais
% author: sistema
\exercicio{
Valor de um automóvel: $f(t)=18000(0,85)^t$ (euros) após $t$ anos.
\begin{enumerate}
  \item Estima valor aos 18 meses.
  \vspace{3cm}
  \item Determina quando $f(t)<4000$.
  \vspace{3cm}
  \item Calcula percentagem de desvalorização anual.
  \vspace{3cm}
  \item Esboça o gráfico e interpreta três pontos.
    \begin{center}
  \begin{tikzpicture}[scale=0.8]
    \draw[->] (-0.5,0) -- (10,0) node[right] {$t$};
    \draw[->] (0,-0.5) -- (0,6) node[above] {$C(t)$};
    \foreach \x in {1,2,...,9}
      \draw (\x,0.1) -- (\x,-0.1);
    \foreach \y in {1,2,...,5}
      \draw (0.1,\y) -- (-0.1,\y);
  \end{tikzpicture}
  \end{center}
\end{enumerate}
}
\FloatBarrier

\end{document}
