% Minimal sebenta template generated automatically
\documentclass[11pt,a4paper]{article}
\usepackage[utf8]{inputenc}
\usepackage[T1]{fontenc}
\usepackage{lmodern}
\usepackage{geometry}
\usepackage{fancyhdr}
\usepackage{hyperref}
\usepackage{graphicx}
\usepackage{float}
\usepackage{placeins}
\usepackage{bookmark}
\usepackage{booktabs}
\usepackage{amsmath,amssymb}
\usepackage{csquotes}
\usepackage{enumitem}
\usepackage{tikz}
\IfFileExists{pgfplots.sty}{\usepackage{pgfplots}\pgfplotsset{compat=1.17}}{}

\geometry{margin=2.5cm}

% Try to include project-specific style macros (containing \exercicio, \subexercicio, etc.)
% Try multiple relative locations to be robust across different generated output paths
\IfFileExists{../../../../Teste_modelo/config/style.tex}{% Sistema de exercícios com contadores automáticos
\newcounter{exerciciocount}          % Contador principal dos exercícios
\newcounter{subexerciciocount}       % Contador dos subexercícios
\newcounter{optioncount}             % Contador das opções

% Control whether the macro prints the automatic "Exercício N." heading.
% Default: show the heading. Call \showexerciciotitlefalse to suppress.
\newif\ifshowexerciciotitle
\showexerciciotitletrue

% Macro para exercício principal
\newcommand{\exercicio}[1]{%
        \par\vspace{1.5em}% Espaçamento antes
        \refstepcounter{exerciciocount}% Incrementa contador principal
        \setcounter{subexerciciocount}{0}% Reseta contador de subexercícios
        \setcounter{optioncount}{0}% Reseta contador de opções
        % Only print the automatic heading if the flag is true
        \ifshowexerciciotitle
            \noindent\textbf{Exercício~\theexerciciocount.}\space #1\par\vspace{0.5em}%
        \else
            % When suppressed, just print the content without the heading
            #1\par\vspace{0.5em}%
        \fi
}

% Macro para subexercício
\newcommand{\subexercicio}[1]{%
    \par\vspace{0.8em}% Espaçamento menor para subexercícios
    \refstepcounter{subexerciciocount}% Incrementa contador de subexercícios
    \noindent\textbf{\theexerciciocount.\thesubexerciciocount.} #1\par\vspace{0.3em}%
}

% Macro para opção
\newcommand{\option}[1]{%
    \par
    \refstepcounter{optioncount}%
    \noindent(\alph{optioncount}) #1%
}

% Título e informações do exame
\title{1ª Questão de aula do Módulo A10: Otimização}
\author{EPRALIMA - Escola Profissional Alto Lima}

\date{}

% Cabeçalho completo do teste dentro de uma caixa simples
\newcommand{\espacoAluno}{%
    \vspace{0.5cm}
    \fbox{%
        \parbox{\textwidth}{%
            \noindent\textbf{Nome do Aluno:} \underline{\hspace{7cm}} \textbf{Turma:} \underline{\hspace{1cm}}\\[0.5cm]
            \noindent\textbf{Assinatura do Professor:} \underline{\hspace{3cm}} \hfill \textbf{Nota:} \underline{\hspace{2cm}}\\[0.5cm]
            \noindent\textbf{Assinatura do Encarregado de Educação:} \underline{\hspace{3cm}}
        }%
    }
    \vspace{1cm}
}}{%
  \IfFileExists{../../../Teste_modelo/config/style.tex}{% Sistema de exercícios com contadores automáticos
\newcounter{exerciciocount}          % Contador principal dos exercícios
\newcounter{subexerciciocount}       % Contador dos subexercícios
\newcounter{optioncount}             % Contador das opções

% Control whether the macro prints the automatic "Exercício N." heading.
% Default: show the heading. Call \showexerciciotitlefalse to suppress.
\newif\ifshowexerciciotitle
\showexerciciotitletrue

% Macro para exercício principal
\newcommand{\exercicio}[1]{%
        \par\vspace{1.5em}% Espaçamento antes
        \refstepcounter{exerciciocount}% Incrementa contador principal
        \setcounter{subexerciciocount}{0}% Reseta contador de subexercícios
        \setcounter{optioncount}{0}% Reseta contador de opções
        % Only print the automatic heading if the flag is true
        \ifshowexerciciotitle
            \noindent\textbf{Exercício~\theexerciciocount.}\space #1\par\vspace{0.5em}%
        \else
            % When suppressed, just print the content without the heading
            #1\par\vspace{0.5em}%
        \fi
}

% Macro para subexercício
\newcommand{\subexercicio}[1]{%
    \par\vspace{0.8em}% Espaçamento menor para subexercícios
    \refstepcounter{subexerciciocount}% Incrementa contador de subexercícios
    \noindent\textbf{\theexerciciocount.\thesubexerciciocount.} #1\par\vspace{0.3em}%
}

% Macro para opção
\newcommand{\option}[1]{%
    \par
    \refstepcounter{optioncount}%
    \noindent(\alph{optioncount}) #1%
}

% Título e informações do exame
\title{1ª Questão de aula do Módulo A10: Otimização}
\author{EPRALIMA - Escola Profissional Alto Lima}

\date{}

% Cabeçalho completo do teste dentro de uma caixa simples
\newcommand{\espacoAluno}{%
    \vspace{0.5cm}
    \fbox{%
        \parbox{\textwidth}{%
            \noindent\textbf{Nome do Aluno:} \underline{\hspace{7cm}} \textbf{Turma:} \underline{\hspace{1cm}}\\[0.5cm]
            \noindent\textbf{Assinatura do Professor:} \underline{\hspace{3cm}} \hfill \textbf{Nota:} \underline{\hspace{2cm}}\\[0.5cm]
            \noindent\textbf{Assinatura do Encarregado de Educação:} \underline{\hspace{3cm}}
        }%
    }
    \vspace{1cm}
}}{%
    \IfFileExists{../../Teste_modelo/config/style.tex}{% Sistema de exercícios com contadores automáticos
\newcounter{exerciciocount}          % Contador principal dos exercícios
\newcounter{subexerciciocount}       % Contador dos subexercícios
\newcounter{optioncount}             % Contador das opções

% Control whether the macro prints the automatic "Exercício N." heading.
% Default: show the heading. Call \showexerciciotitlefalse to suppress.
\newif\ifshowexerciciotitle
\showexerciciotitletrue

% Macro para exercício principal
\newcommand{\exercicio}[1]{%
        \par\vspace{1.5em}% Espaçamento antes
        \refstepcounter{exerciciocount}% Incrementa contador principal
        \setcounter{subexerciciocount}{0}% Reseta contador de subexercícios
        \setcounter{optioncount}{0}% Reseta contador de opções
        % Only print the automatic heading if the flag is true
        \ifshowexerciciotitle
            \noindent\textbf{Exercício~\theexerciciocount.}\space #1\par\vspace{0.5em}%
        \else
            % When suppressed, just print the content without the heading
            #1\par\vspace{0.5em}%
        \fi
}

% Macro para subexercício
\newcommand{\subexercicio}[1]{%
    \par\vspace{0.8em}% Espaçamento menor para subexercícios
    \refstepcounter{subexerciciocount}% Incrementa contador de subexercícios
    \noindent\textbf{\theexerciciocount.\thesubexerciciocount.} #1\par\vspace{0.3em}%
}

% Macro para opção
\newcommand{\option}[1]{%
    \par
    \refstepcounter{optioncount}%
    \noindent(\alph{optioncount}) #1%
}

% Título e informações do exame
\title{1ª Questão de aula do Módulo A10: Otimização}
\author{EPRALIMA - Escola Profissional Alto Lima}

\date{}

% Cabeçalho completo do teste dentro de uma caixa simples
\newcommand{\espacoAluno}{%
    \vspace{0.5cm}
    \fbox{%
        \parbox{\textwidth}{%
            \noindent\textbf{Nome do Aluno:} \underline{\hspace{7cm}} \textbf{Turma:} \underline{\hspace{1cm}}\\[0.5cm]
            \noindent\textbf{Assinatura do Professor:} \underline{\hspace{3cm}} \hfill \textbf{Nota:} \underline{\hspace{2cm}}\\[0.5cm]
            \noindent\textbf{Assinatura do Encarregado de Educação:} \underline{\hspace{3cm}}
        }%
    }
    \vspace{1cm}
}}{%
      % style.tex not found - proceed without project macros
    }%
  }%
}

% Provide a robust fallback for macros that might be missing in style.tex
% This attempts to include the project style first (multiple relative paths),
% and only if none exist defines minimal counters and macros safely.
\IfFileExists{../../../../Teste_modelo/config/style.tex}{% Sistema de exercícios com contadores automáticos
\newcounter{exerciciocount}          % Contador principal dos exercícios
\newcounter{subexerciciocount}       % Contador dos subexercícios
\newcounter{optioncount}             % Contador das opções

% Control whether the macro prints the automatic "Exercício N." heading.
% Default: show the heading. Call \showexerciciotitlefalse to suppress.
\newif\ifshowexerciciotitle
\showexerciciotitletrue

% Macro para exercício principal
\newcommand{\exercicio}[1]{%
        \par\vspace{1.5em}% Espaçamento antes
        \refstepcounter{exerciciocount}% Incrementa contador principal
        \setcounter{subexerciciocount}{0}% Reseta contador de subexercícios
        \setcounter{optioncount}{0}% Reseta contador de opções
        % Only print the automatic heading if the flag is true
        \ifshowexerciciotitle
            \noindent\textbf{Exercício~\theexerciciocount.}\space #1\par\vspace{0.5em}%
        \else
            % When suppressed, just print the content without the heading
            #1\par\vspace{0.5em}%
        \fi
}

% Macro para subexercício
\newcommand{\subexercicio}[1]{%
    \par\vspace{0.8em}% Espaçamento menor para subexercícios
    \refstepcounter{subexerciciocount}% Incrementa contador de subexercícios
    \noindent\textbf{\theexerciciocount.\thesubexerciciocount.} #1\par\vspace{0.3em}%
}

% Macro para opção
\newcommand{\option}[1]{%
    \par
    \refstepcounter{optioncount}%
    \noindent(\alph{optioncount}) #1%
}

% Título e informações do exame
\title{1ª Questão de aula do Módulo A10: Otimização}
\author{EPRALIMA - Escola Profissional Alto Lima}

\date{}

% Cabeçalho completo do teste dentro de uma caixa simples
\newcommand{\espacoAluno}{%
    \vspace{0.5cm}
    \fbox{%
        \parbox{\textwidth}{%
            \noindent\textbf{Nome do Aluno:} \underline{\hspace{7cm}} \textbf{Turma:} \underline{\hspace{1cm}}\\[0.5cm]
            \noindent\textbf{Assinatura do Professor:} \underline{\hspace{3cm}} \hfill \textbf{Nota:} \underline{\hspace{2cm}}\\[0.5cm]
            \noindent\textbf{Assinatura do Encarregado de Educação:} \underline{\hspace{3cm}}
        }%
    }
    \vspace{1cm}
}}{%
  \IfFileExists{../../../Teste_modelo/config/style.tex}{% Sistema de exercícios com contadores automáticos
\newcounter{exerciciocount}          % Contador principal dos exercícios
\newcounter{subexerciciocount}       % Contador dos subexercícios
\newcounter{optioncount}             % Contador das opções

% Control whether the macro prints the automatic "Exercício N." heading.
% Default: show the heading. Call \showexerciciotitlefalse to suppress.
\newif\ifshowexerciciotitle
\showexerciciotitletrue

% Macro para exercício principal
\newcommand{\exercicio}[1]{%
        \par\vspace{1.5em}% Espaçamento antes
        \refstepcounter{exerciciocount}% Incrementa contador principal
        \setcounter{subexerciciocount}{0}% Reseta contador de subexercícios
        \setcounter{optioncount}{0}% Reseta contador de opções
        % Only print the automatic heading if the flag is true
        \ifshowexerciciotitle
            \noindent\textbf{Exercício~\theexerciciocount.}\space #1\par\vspace{0.5em}%
        \else
            % When suppressed, just print the content without the heading
            #1\par\vspace{0.5em}%
        \fi
}

% Macro para subexercício
\newcommand{\subexercicio}[1]{%
    \par\vspace{0.8em}% Espaçamento menor para subexercícios
    \refstepcounter{subexerciciocount}% Incrementa contador de subexercícios
    \noindent\textbf{\theexerciciocount.\thesubexerciciocount.} #1\par\vspace{0.3em}%
}

% Macro para opção
\newcommand{\option}[1]{%
    \par
    \refstepcounter{optioncount}%
    \noindent(\alph{optioncount}) #1%
}

% Título e informações do exame
\title{1ª Questão de aula do Módulo A10: Otimização}
\author{EPRALIMA - Escola Profissional Alto Lima}

\date{}

% Cabeçalho completo do teste dentro de uma caixa simples
\newcommand{\espacoAluno}{%
    \vspace{0.5cm}
    \fbox{%
        \parbox{\textwidth}{%
            \noindent\textbf{Nome do Aluno:} \underline{\hspace{7cm}} \textbf{Turma:} \underline{\hspace{1cm}}\\[0.5cm]
            \noindent\textbf{Assinatura do Professor:} \underline{\hspace{3cm}} \hfill \textbf{Nota:} \underline{\hspace{2cm}}\\[0.5cm]
            \noindent\textbf{Assinatura do Encarregado de Educação:} \underline{\hspace{3cm}}
        }%
    }
    \vspace{1cm}
}}{%
    \IfFileExists{../../Teste_modelo/config/style.tex}{% Sistema de exercícios com contadores automáticos
\newcounter{exerciciocount}          % Contador principal dos exercícios
\newcounter{subexerciciocount}       % Contador dos subexercícios
\newcounter{optioncount}             % Contador das opções

% Control whether the macro prints the automatic "Exercício N." heading.
% Default: show the heading. Call \showexerciciotitlefalse to suppress.
\newif\ifshowexerciciotitle
\showexerciciotitletrue

% Macro para exercício principal
\newcommand{\exercicio}[1]{%
        \par\vspace{1.5em}% Espaçamento antes
        \refstepcounter{exerciciocount}% Incrementa contador principal
        \setcounter{subexerciciocount}{0}% Reseta contador de subexercícios
        \setcounter{optioncount}{0}% Reseta contador de opções
        % Only print the automatic heading if the flag is true
        \ifshowexerciciotitle
            \noindent\textbf{Exercício~\theexerciciocount.}\space #1\par\vspace{0.5em}%
        \else
            % When suppressed, just print the content without the heading
            #1\par\vspace{0.5em}%
        \fi
}

% Macro para subexercício
\newcommand{\subexercicio}[1]{%
    \par\vspace{0.8em}% Espaçamento menor para subexercícios
    \refstepcounter{subexerciciocount}% Incrementa contador de subexercícios
    \noindent\textbf{\theexerciciocount.\thesubexerciciocount.} #1\par\vspace{0.3em}%
}

% Macro para opção
\newcommand{\option}[1]{%
    \par
    \refstepcounter{optioncount}%
    \noindent(\alph{optioncount}) #1%
}

% Título e informações do exame
\title{1ª Questão de aula do Módulo A10: Otimização}
\author{EPRALIMA - Escola Profissional Alto Lima}

\date{}

% Cabeçalho completo do teste dentro de uma caixa simples
\newcommand{\espacoAluno}{%
    \vspace{0.5cm}
    \fbox{%
        \parbox{\textwidth}{%
            \noindent\textbf{Nome do Aluno:} \underline{\hspace{7cm}} \textbf{Turma:} \underline{\hspace{1cm}}\\[0.5cm]
            \noindent\textbf{Assinatura do Professor:} \underline{\hspace{3cm}} \hfill \textbf{Nota:} \underline{\hspace{2cm}}\\[0.5cm]
            \noindent\textbf{Assinatura do Encarregado de Educação:} \underline{\hspace{3cm}}
        }%
    }
    \vspace{1cm}
}}{%
      % style.tex not found - define minimal counters/macros defensively
      \makeatletter
      \@ifundefined{exerciciocount}{\newcounter{exerciciocount}}{}
      \@ifundefined{subexerciciocount}{\newcounter{subexerciciocount}}{}
      \@ifundefined{optioncount}{\newcounter{optioncount}}{}

      \newcommand{\exercicio}[1]{%
        \par\vspace{1.5em}%
        \refstepcounter{exerciciocount}%
        \setcounter{subexerciciocount}{0}%
        \setcounter{optioncount}{0}%
        \noindent\textbf{Exercício~\theexerciciocount.} #1\par\vspace{0.5em}%
      }

      \newcommand{\subexercicio}[1]{%
        \par\vspace{0.8em}%
        \refstepcounter{subexerciciocount}%
        \noindent\textbf{\theexerciciocount.\thesubexerciciocount.} #1\par\vspace{0.3em}%
      }

      \newcommand{\exercicioDesenvolvimento}[1]{\par\noindent #1\par}
      \newcommand{\option}[1]{%
        \par\refstepcounter{optioncount}%
        \noindent(\alph{optioncount}) #1%
      }
      \makeatother
    }%
  }%
}

% ========== IP-BASED TEST SYSTEM MACROS (v3.5) ==========
% Support for modular exercise inclusion with numbered headings
% Provide a boolean flag to control whether the automatic heading is shown
\makeatletter
\@ifundefined{showexerciciotitletrue}{%
    \newif\ifshowexerciciotitle
    \showexerciciotitletrue
}{}
\makeatother

% Override \exercicio to respect the \ifshowexerciciotitle flag
% When false, it prints only the content without automatic heading
\renewcommand{\exercicio}[1]{%
    \ifshowexerciciotitle
        \par\vspace{1.5em}%
        \refstepcounter{exerciciocount}%
        \setcounter{subexerciciocount}{0}%
        \setcounter{optioncount}{0}%
        \noindent\textbf{Exercício~\theexerciciocount.} #1\par\vspace{0.5em}%
    \else
        #1\par
    \fi
}


\pagestyle{fancy}
\fancyhf{}
\lhead{Módulo P2 - Estatística}
\rhead{}
\cfoot{\thepage}

\title{}
\author{}
\date{}

\begin{document}
\maketitle

\section*{Módulo P2 - Estatística}

\textit{Introdução à Estatística Descritiva}

\vspace{1em}

Este documento contém todos os exercícios do módulo, organizados por conceito.

\tableofcontents
\newpage

\section{0 - revisoes}

% Exercise ID: MAT_P2ESTATI_REVI_D_001
% Module: Módulo P2 - Estatística | Concept: Revisões de Crescimento
% Type: desenvolvimento | Difficulty: 2/5
% Tags: revisoes
% Author: Professor | Date: 2025-11-21

\exercicio{
Uma mercearia registou o número de peças de fruta vendidas num dia. A tabela abaixo resume os dados:

\begin{center}
\begin{tabular}{|l|c|}
\hline
\textbf{Fruta} & \textbf{Quantidade} \\
\hline
Maçãs & 45 \\
Bananas & 38 \\
Laranjas & 22 \\
Peras & 15 \\
\hline
\end{tabular}
\end{center}
}

\subexercicio{Indica qual a fruta que representa a maior percentagem das vendas. Justifica a tua resposta com referência aos valores calculados.}

\subexercicio{Calcula quantas peças correspondem a $25\%$ do total de vendas. Indica se deves arredondar o resultado e justifica a tua escolha (arredondamento por defeito, por excesso ou para o inteiro mais próximo).}

\subexercicio{Calcula a percentagem conjunta de Maçãs e Bananas. Essas duas frutas representam mais de $50\%$ do total? Mostra os cálculos e conclui.}

\FloatBarrier

% Exercise ID: MAT_P2ESTATI_REVI_D_999
% Module: P2_estatistica | Concept: 0-revisoes
% Type: desenvolvimento | Difficulty: 2/5
% Tags: revisoes, teste, robustez
% Author: Teste | Date: 2025-11-21

\exercicio{
Uma fábrica produziu 100 peças num dia. A tabela mostra a distribuição por tipo:

\begin{center}
\begin{tabular}{|l|c|}
\hline
Tipo & Quantidade \\
\hline
A & 40 \\
B & 35 \\
C & 25 \\
\hline
\end{tabular}
\end{center}
}

\subexercicio{Qual a percentagem de peças do tipo A?}
\vspace{3cm}
\subexercicio{Quantas peças não são do tipo B?}
\vspace{3cm}
\subexercicio{A soma das percentagens dá 100\%? Justifique.}
\vspace{3cm}
\FloatBarrier


\newpage

\section{1 - Medicoes basicas}

% Exercise ID: MAT_P2ESTATI_1MED_ENV_001
% Module: Módulo P2 - Estatística | Concept: Medições Básicas
% Type: estatistica_na_vida | Difficulty: 2/5
% Tags: media, moda, mediana, notas, interpretacao
% Author: Professor | Date: 2025-11-25

\exercicio{
O João registou as suas notas (de 0 a 20 valores) nas primeiras 7 fichas de avaliação do ano letivo:

\begin{center}
\begin{tabular}{|c|c|c|c|c|c|c|}
\hline
\textbf{Ficha 1} & \textbf{Ficha 2} & \textbf{Ficha 3} & \textbf{Ficha 4} & \textbf{Ficha 5} & \textbf{Ficha 6} & \textbf{Ficha 7} \\
\hline
14 & 12 & 15 & 12 & 16 & 14 & 12 \\
\hline
\end{tabular}
\end{center}
}

\subexercicio{Calcula a média das notas do João. Arredonda à unidade.}

\subexercicio{Determina a moda das notas. O que significa este valor para o desempenho do João?}

\subexercicio{Calcula a mediana das notas. Mostra os valores ordenados.}

\subexercicio{Se o João precisar de ter pelo menos 13 valores de média para passar, qual é a nota mínima que precisa na próxima ficha? Justifica.}

\FloatBarrier

% Exercise ID: MAT_P2ESTATI_1MED_ENV_002
% Module: Módulo P2 - Estatística | Concept: Medições Básicas
% Type: estatistica_na_vida | Difficulty: 2/5
% Tags: media, moda, mediana, gastos, interpretacao
% Author: Professor | Date: 2025-11-25

\exercicio{
A Maria registou os seus gastos semanais em transporte (em euros) durante um mês:

\begin{center}
\begin{tabular}{|c|c|c|c|}
\hline
\textbf{Semana 1} & \textbf{Semana 2} & \textbf{Semana 3} & \textbf{Semana 4} \\
\hline
15 & 18 & 15 & 22 \\
\hline
\end{tabular}
\end{center}
}

\subexercicio{Calcula a média dos gastos semanais da Maria.}

\subexercicio{Determina a moda dos gastos. Qual o significado prático deste valor?}

\subexercicio{Calcula a mediana. Compara-a com a média.}

\subexercicio{A Maria tem um orçamento mensal de 70€ para transporte. Considerando os dados, achas que ela consegue cumprir o orçamento? Justifica com base nos valores estatísticos calculados.}

\FloatBarrier

% Exercise ID: MAT_P2ESTATI_1MX_ENV_001
% Module: Módulo P2 - Estatística | Concept: Medições Básicas | Type: Estatística na Vida Quotidiana
% Difficulty: 2/5 (Fácil) | Format: desenvolvimento
% Tags: tabelas, moda, interpretacao, media, calculos_com_dados, mediana, vida_quotidiana, teste, automacao
% Author: Teste Automático | Date: 2025-11-26
% Status: active

\exercicio{Exercício de teste para estatistica_na_vida}

% Solution:
% \begin{solucao}
% Solução de exemplo
% \end{solucao}

\FloatBarrier

% Exercise ID: MAT_P2ESTATI_1MX_ENV_002
% Module: Módulo P2 - Estatística | Concept: Medições Básicas | Type: Estatística na Vida Quotidiana
% Difficulty: 2/5 (Fácil) | Format: desenvolvimento
% Tags: interpretacao, mediana, tabelas, moda, vida_quotidiana, media, calculos_com_dados, teste, automacao
% Author: Teste Automático | Date: 2025-11-26
% Status: active

\exercicio{Exercício de teste para estatistica_na_vida}

\FloatBarrier

% Exercise ID: MAT_P2ESTATI_1MED_EPO_001
% Module: Módulo P2 - Estatística | Concept: Medições Básicas
% Type: estatistica_poupanca | Difficulty: 2/5
% Tags: media, moda, mediana, poupanca, interpretacao
% Author: Professor | Date: 2025-11-25

\exercicio{
A família Silva registou as suas poupanças mensais (em euros) durante 6 meses:

\begin{center}
\begin{tabular}{|c|c|c|c|c|c|}
\hline
\textbf{Jan} & \textbf{Fev} & \textbf{Mar} & \textbf{Abr} & \textbf{Mai} & \textbf{Jun} \\
\hline
150 & 200 & 150 & 180 & 200 & 220 \\
\hline
\end{tabular}
\end{center}
}

\subexercicio{Calcula a média das poupanças mensais da família Silva.}

\subexercicio{Determina a moda dos valores registados. Explica o que significa este valor no contexto da poupança.}

\subexercicio{Calcula a mediana das poupanças. Mostra os passos do cálculo.}

\subexercicio{Compara a média e a mediana obtidas. O que podes concluir sobre a distribuição das poupanças da família?}

\FloatBarrier

% Exercise ID: MAT_P2ESTATI_1MED_EPO_002
% Module: Módulo P2 - Estatística | Concept: Medições Básicas
% Type: estatistica_poupanca | Difficulty: 2/5
% Tags: media, moda, mediana, poupanca, interpretacao
% Author: Professor | Date: 2025-11-25

\exercicio{
Uma associação de jovens decidiu poupar para uma viagem de finalistas. A tabela seguinte mostra o valor poupado (em euros) por cada membro durante um mês:

\begin{center}
\begin{tabular}{|l|c|}
\hline
\textbf{Membro} & \textbf{Poupança (€)} \\
\hline
Ana & 25 \\
Bruno & 30 \\
Carla & 25 \\
Daniel & 40 \\
Eva & 30 \\
Filipe & 25 \\
Gonçalo & 35 \\
Helena & 30 \\
\hline
\end{tabular}
\end{center}
}

\subexercicio{Calcula a média das poupanças do grupo.}

\subexercicio{Determina a moda. Quantos membros pouparam esse valor?}

\subexercicio{Calcula a mediana das poupanças. Ordena primeiro os valores.}

\subexercicio{Se o objetivo é que cada membro contribua com pelo menos 30€, quantos membros ficaram abaixo desse valor? Usa a mediana para justificar se o grupo está no bom caminho.}

\FloatBarrier

% Exercise ID: MAT_P2ESTATI_1MX_EPX_001
% Module: Módulo P2 - Estatística | Concept: Medições Básicas | Type: Estatística Aplicada à Poupança
% Difficulty: 3/5 (Médio) | Format: desenvolvimento
% Tags: poupanca, mediana, moda, media, interpretacao, calculos_com_dados, tabelas, teste, automacao
% Author: Teste Automático | Date: 2025-11-26
% Status: active

\exercicio{Exercício de teste para estatistica_poupanca}

% Solution:
% \begin{solucao}
% Solução de exemplo
% \end{solucao}

\FloatBarrier

% Exercise ID: MAT_P2ESTATI_1MED_EPU_001
% Module: Módulo P2 - Estatística | Concept: Medições Básicas
% Type: estatistica_pura | Difficulty: 1/5
% Tags: media, moda, mediana, calculo_direto
% Author: Professor | Date: 2025-11-25

\exercicio{
Considera o seguinte conjunto de dados:

\begin{center}
$\{5, 8, 12, 8, 6, 10, 8, 9\}$
\end{center}
}

\subexercicio{Calcula a média aritmética dos valores.}

\subexercicio{Determina a moda do conjunto de dados.}

\subexercicio{Ordena os valores e calcula a mediana.}

\subexercicio{Qual das três medidas (média, moda ou mediana) melhor representa o ``centro'' destes dados? Justifica a tua resposta.}

\FloatBarrier

% Exercise ID: MAT_P2ESTATI_1MED_EPU_002
% Module: Módulo P2 - Estatística | Concept: Medições Básicas
% Type: estatistica_pura | Difficulty: 2/5
% Tags: media, moda, mediana, calculo_direto
% Author: Professor | Date: 2025-11-25

\exercicio{
Considera o seguinte conjunto de dados:

\begin{center}
$\{3, 7, 7, 10, 15, 7, 12, 9, 11\}$
\end{center}
}

\subexercicio{Calcula a média aritmética. Apresenta o resultado com uma casa decimal.}

\subexercicio{Identifica a moda. Quantas vezes aparece esse valor?}

\subexercicio{Calcula a mediana. Nota: este conjunto tem um número ímpar de elementos.}

\subexercicio{Se adicionarmos o valor 100 ao conjunto, como achas que isso afetaria a média e a mediana? Calcula os novos valores e comenta a diferença.}

\FloatBarrier

% Exercise ID: MAT_P2ESTATI_1MX_EPX_001
% Module: Módulo P2 - Estatística | Concept: Medições Básicas | Type: Estatística Pura
% Difficulty: 4/5 (Difícil) | Format: desenvolvimento
% Tags: calculo_direto, tabelas, mediana, media, moda, calculos_com_dados, teste, automacao
% Author: Teste Automático | Date: 2025-11-26
% Status: active

\exercicio{Exercício de teste para estatistica_pura}

\FloatBarrier


\newpage

\section{2 - Variabilidade}

% Exercise ID: MAT_P2ESTATI_2VX_EMV_001
% Module: Módulo P2 - Estatística | Concept: Medidas de Variabilidade | Type: Escolha maior variabilidade
% Difficulty: 4/5 (Difícil) | Format: desenvolvimento
% Tags: variabilidade, variancia, desvio_medio, desvio_padrao, teste, automacao
% Author: Teste Automático | Date: 2025-11-26
% Status: active

\exercicio{Exercício de teste para escolha_m_variabilidade}

% Solution:
% \begin{solucao}
% Solução de exemplo
% \end{solucao}

\FloatBarrier

% Exercise ID: MAT_P2ESTATI_VARI_EM_001
% Module: P2_estatistica | Concept: 2-Variabilidade
% Type: escolha_multipla | Difficulty: 1/5
% Tags: Variabilidade, escolha multipla
% Author: Professor | Date: 2025-11-21

\exercicio{
Considere as seguintes situações e escolha a que apresenta maior variabilidade:
}

% \subexercicio{...}  % REMOVED placeholder - preencher a alínea real
\subexercicio{Temperatura ao longo de um ano (diária) vs temperatura ao longo de um dia (horária).}
\subexercicio{Número de carros que passam numa rua por hora vs número de carros que passam numa semana na mesma rua.}
\subexercicio{Peso de recém-nascidos numa maternidade vs peso de adultos numa população.}
\subexercicio{Nota de um aluno nas várias avaliações ao longo do ano vs nota média da turma em cada avaliação.}
% \subexercicio{...}  % REMOVED placeholder - preencher a alínea real
% \subexercicio{...}  % REMOVED placeholder - preencher a alínea real

\FloatBarrier

% Exercise ID: MAT_P2ESTATI_2VAR_MDV_001
% Module: Módulo P2 - Estatística | Concept: Variabilidade
% Type: medidas_de_variabilidade | Difficulty: 2/5
% Tags: amplitude, desvio_absoluto_medio, variabilidade, poupanca
% Author: Professor | Date: 2025-11-25

\exercicio{
Uma família registou as suas poupanças mensais (em euros) durante 5 meses:

\begin{center}
$\{120, 180, 150, 200, 100\}$
\end{center}
}

\subexercicio{Calcula a amplitude das poupanças. Identifica o valor máximo e o valor mínimo.}

\subexercicio{Calcula a média das poupanças.}

\subexercicio{Calcula o Desvio Absoluto Médio (DAM) usando a fórmula:
\[
\text{DAM} = \frac{\sum_{i=1}^{n} |x_i - \bar{x}|}{n}
\]
Para isso:
\begin{enumerate}
\item Calcula a diferença de cada valor para a média (em módulo).
\item Soma todas as diferenças.
\item Divide pelo número de valores.
\end{enumerate}}

\subexercicio{O que indica um DAM alto ou baixo sobre a consistência das poupanças da família? Interpreta o valor que obtiveste.}

\FloatBarrier

% Exercise ID: MAT_P2ESTATI_2VAR_MDV_002
% Module: Módulo P2 - Estatística | Concept: Variabilidade
% Type: medidas_de_variabilidade | Difficulty: 2/5
% Tags: amplitude, desvio_absoluto_medio, variabilidade, notas
% Author: Professor | Date: 2025-11-25

\exercicio{
Dois alunos, Pedro e Joana, tiveram as seguintes notas (de 0 a 20) em 6 testes:

\begin{center}
\begin{tabular}{|l|c|c|c|c|c|c|}
\hline
& \textbf{Teste 1} & \textbf{Teste 2} & \textbf{Teste 3} & \textbf{Teste 4} & \textbf{Teste 5} & \textbf{Teste 6} \\
\hline
\textbf{Pedro} & 10 & 18 & 8 & 16 & 12 & 14 \\
\hline
\textbf{Joana} & 12 & 14 & 13 & 13 & 12 & 14 \\
\hline
\end{tabular}
\end{center}
}

\subexercicio{Calcula a amplitude das notas de cada aluno.}

\subexercicio{Calcula a média das notas de cada aluno.}

\subexercicio{Calcula o Desvio Absoluto Médio (DAM) das notas de cada aluno.}

\subexercicio{Qual dos alunos teve um desempenho mais consistente? Justifica usando os valores da amplitude e do DAM.}

\FloatBarrier


\newpage


\end{document}
