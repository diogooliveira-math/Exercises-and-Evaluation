% Minimal sebenta template generated automatically
\documentclass[11pt,a4paper]{article}
\usepackage[utf8]{inputenc}
\usepackage[T1]{fontenc}
\usepackage{lmodern}
\usepackage{geometry}
\usepackage{fancyhdr}
\usepackage{hyperref}
\usepackage{graphicx}
\usepackage{float}
\usepackage{placeins}
\usepackage{bookmark}
\usepackage{booktabs}
\usepackage{amsmath,amssymb}
\usepackage{csquotes}
\usepackage{enumitem}
\usepackage{tikz}
\IfFileExists{pgfplots.sty}{\usepackage{pgfplots}\pgfplotsset{compat=1.17}}{}

\geometry{margin=2.5cm}

% Try to include project-specific style macros (containing \exercicio, \subexercicio, etc.)
% Try multiple relative locations to be robust across different generated output paths
\IfFileExists{../../../../Teste_modelo/config/style.tex}{% Sistema de exercícios com contadores automáticos
\newcounter{exerciciocount}          % Contador principal dos exercícios
\newcounter{subexerciciocount}       % Contador dos subexercícios
\newcounter{optioncount}             % Contador das opções

% Control whether the macro prints the automatic "Exercício N." heading.
% Default: show the heading. Call \showexerciciotitlefalse to suppress.
\newif\ifshowexerciciotitle
\showexerciciotitletrue

% Macro para exercício principal
\newcommand{\exercicio}[1]{%
        \par\vspace{1.5em}% Espaçamento antes
        \refstepcounter{exerciciocount}% Incrementa contador principal
        \setcounter{subexerciciocount}{0}% Reseta contador de subexercícios
        \setcounter{optioncount}{0}% Reseta contador de opções
        % Only print the automatic heading if the flag is true
        \ifshowexerciciotitle
            \noindent\textbf{Exercício~\theexerciciocount.}\space #1\par\vspace{0.5em}%
        \else
            % When suppressed, just print the content without the heading
            #1\par\vspace{0.5em}%
        \fi
}

% Macro para subexercício
\newcommand{\subexercicio}[1]{%
    \par\vspace{0.8em}% Espaçamento menor para subexercícios
    \refstepcounter{subexerciciocount}% Incrementa contador de subexercícios
    \noindent\textbf{\theexerciciocount.\thesubexerciciocount.} #1\par\vspace{0.3em}%
}

% Macro para opção
\newcommand{\option}[1]{%
    \par
    \refstepcounter{optioncount}%
    \noindent(\alph{optioncount}) #1%
}

% Título e informações do exame
\title{1ª Questão de aula do Módulo A10: Otimização}
\author{EPRALIMA - Escola Profissional Alto Lima}

\date{}

% Cabeçalho completo do teste dentro de uma caixa simples
\newcommand{\espacoAluno}{%
    \vspace{0.5cm}
    \fbox{%
        \parbox{\textwidth}{%
            \noindent\textbf{Nome do Aluno:} \underline{\hspace{7cm}} \textbf{Turma:} \underline{\hspace{1cm}}\\[0.5cm]
            \noindent\textbf{Assinatura do Professor:} \underline{\hspace{3cm}} \hfill \textbf{Nota:} \underline{\hspace{2cm}}\\[0.5cm]
            \noindent\textbf{Assinatura do Encarregado de Educação:} \underline{\hspace{3cm}}
        }%
    }
    \vspace{1cm}
}}{%
  \IfFileExists{../../../Teste_modelo/config/style.tex}{% Sistema de exercícios com contadores automáticos
\newcounter{exerciciocount}          % Contador principal dos exercícios
\newcounter{subexerciciocount}       % Contador dos subexercícios
\newcounter{optioncount}             % Contador das opções

% Control whether the macro prints the automatic "Exercício N." heading.
% Default: show the heading. Call \showexerciciotitlefalse to suppress.
\newif\ifshowexerciciotitle
\showexerciciotitletrue

% Macro para exercício principal
\newcommand{\exercicio}[1]{%
        \par\vspace{1.5em}% Espaçamento antes
        \refstepcounter{exerciciocount}% Incrementa contador principal
        \setcounter{subexerciciocount}{0}% Reseta contador de subexercícios
        \setcounter{optioncount}{0}% Reseta contador de opções
        % Only print the automatic heading if the flag is true
        \ifshowexerciciotitle
            \noindent\textbf{Exercício~\theexerciciocount.}\space #1\par\vspace{0.5em}%
        \else
            % When suppressed, just print the content without the heading
            #1\par\vspace{0.5em}%
        \fi
}

% Macro para subexercício
\newcommand{\subexercicio}[1]{%
    \par\vspace{0.8em}% Espaçamento menor para subexercícios
    \refstepcounter{subexerciciocount}% Incrementa contador de subexercícios
    \noindent\textbf{\theexerciciocount.\thesubexerciciocount.} #1\par\vspace{0.3em}%
}

% Macro para opção
\newcommand{\option}[1]{%
    \par
    \refstepcounter{optioncount}%
    \noindent(\alph{optioncount}) #1%
}

% Título e informações do exame
\title{1ª Questão de aula do Módulo A10: Otimização}
\author{EPRALIMA - Escola Profissional Alto Lima}

\date{}

% Cabeçalho completo do teste dentro de uma caixa simples
\newcommand{\espacoAluno}{%
    \vspace{0.5cm}
    \fbox{%
        \parbox{\textwidth}{%
            \noindent\textbf{Nome do Aluno:} \underline{\hspace{7cm}} \textbf{Turma:} \underline{\hspace{1cm}}\\[0.5cm]
            \noindent\textbf{Assinatura do Professor:} \underline{\hspace{3cm}} \hfill \textbf{Nota:} \underline{\hspace{2cm}}\\[0.5cm]
            \noindent\textbf{Assinatura do Encarregado de Educação:} \underline{\hspace{3cm}}
        }%
    }
    \vspace{1cm}
}}{%
    \IfFileExists{../../Teste_modelo/config/style.tex}{% Sistema de exercícios com contadores automáticos
\newcounter{exerciciocount}          % Contador principal dos exercícios
\newcounter{subexerciciocount}       % Contador dos subexercícios
\newcounter{optioncount}             % Contador das opções

% Control whether the macro prints the automatic "Exercício N." heading.
% Default: show the heading. Call \showexerciciotitlefalse to suppress.
\newif\ifshowexerciciotitle
\showexerciciotitletrue

% Macro para exercício principal
\newcommand{\exercicio}[1]{%
        \par\vspace{1.5em}% Espaçamento antes
        \refstepcounter{exerciciocount}% Incrementa contador principal
        \setcounter{subexerciciocount}{0}% Reseta contador de subexercícios
        \setcounter{optioncount}{0}% Reseta contador de opções
        % Only print the automatic heading if the flag is true
        \ifshowexerciciotitle
            \noindent\textbf{Exercício~\theexerciciocount.}\space #1\par\vspace{0.5em}%
        \else
            % When suppressed, just print the content without the heading
            #1\par\vspace{0.5em}%
        \fi
}

% Macro para subexercício
\newcommand{\subexercicio}[1]{%
    \par\vspace{0.8em}% Espaçamento menor para subexercícios
    \refstepcounter{subexerciciocount}% Incrementa contador de subexercícios
    \noindent\textbf{\theexerciciocount.\thesubexerciciocount.} #1\par\vspace{0.3em}%
}

% Macro para opção
\newcommand{\option}[1]{%
    \par
    \refstepcounter{optioncount}%
    \noindent(\alph{optioncount}) #1%
}

% Título e informações do exame
\title{1ª Questão de aula do Módulo A10: Otimização}
\author{EPRALIMA - Escola Profissional Alto Lima}

\date{}

% Cabeçalho completo do teste dentro de uma caixa simples
\newcommand{\espacoAluno}{%
    \vspace{0.5cm}
    \fbox{%
        \parbox{\textwidth}{%
            \noindent\textbf{Nome do Aluno:} \underline{\hspace{7cm}} \textbf{Turma:} \underline{\hspace{1cm}}\\[0.5cm]
            \noindent\textbf{Assinatura do Professor:} \underline{\hspace{3cm}} \hfill \textbf{Nota:} \underline{\hspace{2cm}}\\[0.5cm]
            \noindent\textbf{Assinatura do Encarregado de Educação:} \underline{\hspace{3cm}}
        }%
    }
    \vspace{1cm}
}}{%
      % style.tex not found - proceed without project macros
    }%
  }%
}

% Provide a robust fallback for macros that might be missing in style.tex
% This attempts to include the project style first (multiple relative paths),
% and only if none exist defines minimal counters and macros safely.
\IfFileExists{../../../../Teste_modelo/config/style.tex}{% Sistema de exercícios com contadores automáticos
\newcounter{exerciciocount}          % Contador principal dos exercícios
\newcounter{subexerciciocount}       % Contador dos subexercícios
\newcounter{optioncount}             % Contador das opções

% Control whether the macro prints the automatic "Exercício N." heading.
% Default: show the heading. Call \showexerciciotitlefalse to suppress.
\newif\ifshowexerciciotitle
\showexerciciotitletrue

% Macro para exercício principal
\newcommand{\exercicio}[1]{%
        \par\vspace{1.5em}% Espaçamento antes
        \refstepcounter{exerciciocount}% Incrementa contador principal
        \setcounter{subexerciciocount}{0}% Reseta contador de subexercícios
        \setcounter{optioncount}{0}% Reseta contador de opções
        % Only print the automatic heading if the flag is true
        \ifshowexerciciotitle
            \noindent\textbf{Exercício~\theexerciciocount.}\space #1\par\vspace{0.5em}%
        \else
            % When suppressed, just print the content without the heading
            #1\par\vspace{0.5em}%
        \fi
}

% Macro para subexercício
\newcommand{\subexercicio}[1]{%
    \par\vspace{0.8em}% Espaçamento menor para subexercícios
    \refstepcounter{subexerciciocount}% Incrementa contador de subexercícios
    \noindent\textbf{\theexerciciocount.\thesubexerciciocount.} #1\par\vspace{0.3em}%
}

% Macro para opção
\newcommand{\option}[1]{%
    \par
    \refstepcounter{optioncount}%
    \noindent(\alph{optioncount}) #1%
}

% Título e informações do exame
\title{1ª Questão de aula do Módulo A10: Otimização}
\author{EPRALIMA - Escola Profissional Alto Lima}

\date{}

% Cabeçalho completo do teste dentro de uma caixa simples
\newcommand{\espacoAluno}{%
    \vspace{0.5cm}
    \fbox{%
        \parbox{\textwidth}{%
            \noindent\textbf{Nome do Aluno:} \underline{\hspace{7cm}} \textbf{Turma:} \underline{\hspace{1cm}}\\[0.5cm]
            \noindent\textbf{Assinatura do Professor:} \underline{\hspace{3cm}} \hfill \textbf{Nota:} \underline{\hspace{2cm}}\\[0.5cm]
            \noindent\textbf{Assinatura do Encarregado de Educação:} \underline{\hspace{3cm}}
        }%
    }
    \vspace{1cm}
}}{%
  \IfFileExists{../../../Teste_modelo/config/style.tex}{% Sistema de exercícios com contadores automáticos
\newcounter{exerciciocount}          % Contador principal dos exercícios
\newcounter{subexerciciocount}       % Contador dos subexercícios
\newcounter{optioncount}             % Contador das opções

% Control whether the macro prints the automatic "Exercício N." heading.
% Default: show the heading. Call \showexerciciotitlefalse to suppress.
\newif\ifshowexerciciotitle
\showexerciciotitletrue

% Macro para exercício principal
\newcommand{\exercicio}[1]{%
        \par\vspace{1.5em}% Espaçamento antes
        \refstepcounter{exerciciocount}% Incrementa contador principal
        \setcounter{subexerciciocount}{0}% Reseta contador de subexercícios
        \setcounter{optioncount}{0}% Reseta contador de opções
        % Only print the automatic heading if the flag is true
        \ifshowexerciciotitle
            \noindent\textbf{Exercício~\theexerciciocount.}\space #1\par\vspace{0.5em}%
        \else
            % When suppressed, just print the content without the heading
            #1\par\vspace{0.5em}%
        \fi
}

% Macro para subexercício
\newcommand{\subexercicio}[1]{%
    \par\vspace{0.8em}% Espaçamento menor para subexercícios
    \refstepcounter{subexerciciocount}% Incrementa contador de subexercícios
    \noindent\textbf{\theexerciciocount.\thesubexerciciocount.} #1\par\vspace{0.3em}%
}

% Macro para opção
\newcommand{\option}[1]{%
    \par
    \refstepcounter{optioncount}%
    \noindent(\alph{optioncount}) #1%
}

% Título e informações do exame
\title{1ª Questão de aula do Módulo A10: Otimização}
\author{EPRALIMA - Escola Profissional Alto Lima}

\date{}

% Cabeçalho completo do teste dentro de uma caixa simples
\newcommand{\espacoAluno}{%
    \vspace{0.5cm}
    \fbox{%
        \parbox{\textwidth}{%
            \noindent\textbf{Nome do Aluno:} \underline{\hspace{7cm}} \textbf{Turma:} \underline{\hspace{1cm}}\\[0.5cm]
            \noindent\textbf{Assinatura do Professor:} \underline{\hspace{3cm}} \hfill \textbf{Nota:} \underline{\hspace{2cm}}\\[0.5cm]
            \noindent\textbf{Assinatura do Encarregado de Educação:} \underline{\hspace{3cm}}
        }%
    }
    \vspace{1cm}
}}{%
    \IfFileExists{../../Teste_modelo/config/style.tex}{% Sistema de exercícios com contadores automáticos
\newcounter{exerciciocount}          % Contador principal dos exercícios
\newcounter{subexerciciocount}       % Contador dos subexercícios
\newcounter{optioncount}             % Contador das opções

% Control whether the macro prints the automatic "Exercício N." heading.
% Default: show the heading. Call \showexerciciotitlefalse to suppress.
\newif\ifshowexerciciotitle
\showexerciciotitletrue

% Macro para exercício principal
\newcommand{\exercicio}[1]{%
        \par\vspace{1.5em}% Espaçamento antes
        \refstepcounter{exerciciocount}% Incrementa contador principal
        \setcounter{subexerciciocount}{0}% Reseta contador de subexercícios
        \setcounter{optioncount}{0}% Reseta contador de opções
        % Only print the automatic heading if the flag is true
        \ifshowexerciciotitle
            \noindent\textbf{Exercício~\theexerciciocount.}\space #1\par\vspace{0.5em}%
        \else
            % When suppressed, just print the content without the heading
            #1\par\vspace{0.5em}%
        \fi
}

% Macro para subexercício
\newcommand{\subexercicio}[1]{%
    \par\vspace{0.8em}% Espaçamento menor para subexercícios
    \refstepcounter{subexerciciocount}% Incrementa contador de subexercícios
    \noindent\textbf{\theexerciciocount.\thesubexerciciocount.} #1\par\vspace{0.3em}%
}

% Macro para opção
\newcommand{\option}[1]{%
    \par
    \refstepcounter{optioncount}%
    \noindent(\alph{optioncount}) #1%
}

% Título e informações do exame
\title{1ª Questão de aula do Módulo A10: Otimização}
\author{EPRALIMA - Escola Profissional Alto Lima}

\date{}

% Cabeçalho completo do teste dentro de uma caixa simples
\newcommand{\espacoAluno}{%
    \vspace{0.5cm}
    \fbox{%
        \parbox{\textwidth}{%
            \noindent\textbf{Nome do Aluno:} \underline{\hspace{7cm}} \textbf{Turma:} \underline{\hspace{1cm}}\\[0.5cm]
            \noindent\textbf{Assinatura do Professor:} \underline{\hspace{3cm}} \hfill \textbf{Nota:} \underline{\hspace{2cm}}\\[0.5cm]
            \noindent\textbf{Assinatura do Encarregado de Educação:} \underline{\hspace{3cm}}
        }%
    }
    \vspace{1cm}
}}{%
      % style.tex not found - define minimal counters/macros defensively
      \makeatletter
      \@ifundefined{exerciciocount}{\newcounter{exerciciocount}}{}
      \@ifundefined{subexerciciocount}{\newcounter{subexerciciocount}}{}
      \@ifundefined{optioncount}{\newcounter{optioncount}}{}

      \newcommand{\exercicio}[1]{%
        \par\vspace{1.5em}%
        \refstepcounter{exerciciocount}%
        \setcounter{subexerciciocount}{0}%
        \setcounter{optioncount}{0}%
        \noindent\textbf{Exercício~\theexerciciocount.} #1\par\vspace{0.5em}%
      }

      \newcommand{\subexercicio}[1]{%
        \par\vspace{0.8em}%
        \refstepcounter{subexerciciocount}%
        \noindent\textbf{\theexerciciocount.\thesubexerciciocount.} #1\par\vspace{0.3em}%
      }

      \newcommand{\exercicioDesenvolvimento}[1]{\par\noindent #1\par}
      \newcommand{\option}[1]{%
        \par\refstepcounter{optioncount}%
        \noindent(\alph{optioncount}) #1%
      }
      \makeatother
    }%
  }%
}

% ========== IP-BASED TEST SYSTEM MACROS (v3.5) ==========
% Support for modular exercise inclusion with numbered headings
% Provide a boolean flag to control whether the automatic heading is shown
\makeatletter
\@ifundefined{showexerciciotitletrue}{%
    \newif\ifshowexerciciotitle
    \showexerciciotitletrue
}{}
\makeatother

% Override \exercicio to respect the \ifshowexerciciotitle flag
% When false, it prints only the content without automatic heading
\renewcommand{\exercicio}[1]{%
    \ifshowexerciciotitle
        \par\vspace{1.5em}%
        \refstepcounter{exerciciocount}%
        \setcounter{subexerciciocount}{0}%
        \setcounter{optioncount}{0}%
        \noindent\textbf{Exercício~\theexerciciocount.} #1\par\vspace{0.5em}%
    \else
        #1\par
    \fi
}

\pagestyle{fancy}
\fancyhf{}
\lhead{Módulo P2 - Estatística}
\rhead{1 - Medicoes basicas}
\cfoot{\thepage}

\title{}
\author{}
\date{}

\begin{document}
\maketitle

\section*{1 - Medicoes basicas}

\subsection*{Tipos de Exercícios}
\begin{itemize}
  \item \textbf{Estatística na Vida Quotidiana} --- Exercícios aplicando medidas estatísticas a valores do dia a dia de alunos e pessoas (notas, gastos, tempos, etc.).
  \item \textbf{Estatística Aplicada à Poupança} --- Exercícios aplicando medidas estatísticas (média, moda, mediana) a tabelas que cronometram poupanças de família, pessoa ou instituição.
  \item \textbf{Estatística Pura} --- Exercícios diretos com listas de valores numéricos para calcular medidas estatísticas (média, moda, mediana).
\end{itemize}

\vspace{1em}

% Exercício 1: MAT_P2ESTATI_1MED_ENV_001.tex
% Exercise ID: MAT_P2ESTATI_1MED_ENV_001
% Module: Módulo P2 - Estatística | Concept: Medições Básicas
% Type: estatistica_na_vida | Difficulty: 2/5
% Tags: media, moda, mediana, notas, interpretacao
% Author: Professor | Date: 2025-11-25

\exercicio{
O João registou as suas notas (de 0 a 20 valores) nas primeiras 7 fichas de avaliação do ano letivo:

\begin{center}
\begin{tabular}{|c|c|c|c|c|c|c|}
\hline
\textbf{Ficha 1} & \textbf{Ficha 2} & \textbf{Ficha 3} & \textbf{Ficha 4} & \textbf{Ficha 5} & \textbf{Ficha 6} & \textbf{Ficha 7} \\
\hline
14 & 12 & 15 & 12 & 16 & 14 & 12 \\
\hline
\end{tabular}
\end{center}
}

\subexercicio{Calcula a média das notas do João. Arredonda à unidade.}

\subexercicio{Determina a moda das notas. O que significa este valor para o desempenho do João?}

\subexercicio{Calcula a mediana das notas. Mostra os valores ordenados.}

\subexercicio{Se o João precisar de ter pelo menos 13 valores de média para passar, qual é a nota mínima que precisa na próxima ficha? Justifica.}
\FloatBarrier

% Exercício 2: MAT_P2ESTATI_1MED_ENV_002.tex
% Exercise ID: MAT_P2ESTATI_1MED_ENV_002
% Module: Módulo P2 - Estatística | Concept: Medições Básicas
% Type: estatistica_na_vida | Difficulty: 2/5
% Tags: media, moda, mediana, gastos, interpretacao
% Author: Professor | Date: 2025-11-25

\exercicio{
A Maria registou os seus gastos semanais em transporte (em euros) durante um mês:

\begin{center}
\begin{tabular}{|c|c|c|c|}
\hline
\textbf{Semana 1} & \textbf{Semana 2} & \textbf{Semana 3} & \textbf{Semana 4} \\
\hline
15 & 18 & 15 & 22 \\
\hline
\end{tabular}
\end{center}
}

\subexercicio{Calcula a média dos gastos semanais da Maria.}

\subexercicio{Determina a moda dos gastos. Qual o significado prático deste valor?}

\subexercicio{Calcula a mediana. Compara-a com a média.}

\subexercicio{A Maria tem um orçamento mensal de 70€ para transporte. Considerando os dados, achas que ela consegue cumprir o orçamento? Justifica com base nos valores estatísticos calculados.}
\FloatBarrier

% Exercício 3: MAT_P2ESTATI_1MX_ENV_001.tex
% Exercise ID: MAT_P2ESTATI_1MX_ENV_001
% Module: Módulo P2 - Estatística | Concept: Medições Básicas | Type: Estatística na Vida Quotidiana
% Difficulty: 2/5 (Fácil) | Format: desenvolvimento
% Tags: tabelas, moda, interpretacao, media, calculos_com_dados, mediana, vida_quotidiana, teste, automacao
% Author: Teste Automático | Date: 2025-11-26
% Status: active

\exercicio{Exercício de teste para estatistica_na_vida}

% Solution:
% \begin{solucao}
% Solução de exemplo
% \end{solucao}
\FloatBarrier

% Exercício 4: MAT_P2ESTATI_1MX_ENV_002.tex
% Exercise ID: MAT_P2ESTATI_1MX_ENV_002
% Module: Módulo P2 - Estatística | Concept: Medições Básicas | Type: Estatística na Vida Quotidiana
% Difficulty: 2/5 (Fácil) | Format: desenvolvimento
% Tags: interpretacao, mediana, tabelas, moda, vida_quotidiana, media, calculos_com_dados, teste, automacao
% Author: Teste Automático | Date: 2025-11-26
% Status: active

\exercicio{Exercício de teste para estatistica_na_vida}
\FloatBarrier

% Exercício 5: MAT_P2ESTATI_1MED_EPO_001.tex
% Exercise ID: MAT_P2ESTATI_1MED_EPO_001
% Module: Módulo P2 - Estatística | Concept: Medições Básicas
% Type: estatistica_poupanca | Difficulty: 2/5
% Tags: media, moda, mediana, poupanca, interpretacao
% Author: Professor | Date: 2025-11-25

\exercicio{
A família Silva registou as suas poupanças mensais (em euros) durante 6 meses:

\begin{center}
\begin{tabular}{|c|c|c|c|c|c|}
\hline
\textbf{Jan} & \textbf{Fev} & \textbf{Mar} & \textbf{Abr} & \textbf{Mai} & \textbf{Jun} \\
\hline
150 & 200 & 150 & 180 & 200 & 220 \\
\hline
\end{tabular}
\end{center}
}

\subexercicio{Calcula a média das poupanças mensais da família Silva.}

\subexercicio{Determina a moda dos valores registados. Explica o que significa este valor no contexto da poupança.}

\subexercicio{Calcula a mediana das poupanças. Mostra os passos do cálculo.}

\subexercicio{Compara a média e a mediana obtidas. O que podes concluir sobre a distribuição das poupanças da família?}
\FloatBarrier

% Exercício 6: MAT_P2ESTATI_1MED_EPO_002.tex
% Exercise ID: MAT_P2ESTATI_1MED_EPO_002
% Module: Módulo P2 - Estatística | Concept: Medições Básicas
% Type: estatistica_poupanca | Difficulty: 2/5
% Tags: media, moda, mediana, poupanca, interpretacao
% Author: Professor | Date: 2025-11-25

\exercicio{
Uma associação de jovens decidiu poupar para uma viagem de finalistas. A tabela seguinte mostra o valor poupado (em euros) por cada membro durante um mês:

\begin{center}
\begin{tabular}{|l|c|}
\hline
\textbf{Membro} & \textbf{Poupança (€)} \\
\hline
Ana & 25 \\
Bruno & 30 \\
Carla & 25 \\
Daniel & 40 \\
Eva & 30 \\
Filipe & 25 \\
Gonçalo & 35 \\
Helena & 30 \\
\hline
\end{tabular}
\end{center}
}

\subexercicio{Calcula a média das poupanças do grupo.}

\subexercicio{Determina a moda. Quantos membros pouparam esse valor?}

\subexercicio{Calcula a mediana das poupanças. Ordena primeiro os valores.}

\subexercicio{Se o objetivo é que cada membro contribua com pelo menos 30€, quantos membros ficaram abaixo desse valor? Usa a mediana para justificar se o grupo está no bom caminho.}
\FloatBarrier

% Exercício 7: MAT_P2ESTATI_1MX_EPX_001.tex
% Exercise ID: MAT_P2ESTATI_1MX_EPX_001
% Module: Módulo P2 - Estatística | Concept: Medições Básicas | Type: Estatística Aplicada à Poupança
% Difficulty: 3/5 (Médio) | Format: desenvolvimento
% Tags: poupanca, mediana, moda, media, interpretacao, calculos_com_dados, tabelas, teste, automacao
% Author: Teste Automático | Date: 2025-11-26
% Status: active

\exercicio{Exercício de teste para estatistica_poupanca}

% Solution:
% \begin{solucao}
% Solução de exemplo
% \end{solucao}
\FloatBarrier

% Exercício 8: MAT_P2ESTATI_1MED_EPU_001.tex
% Exercise ID: MAT_P2ESTATI_1MED_EPU_001
% Module: Módulo P2 - Estatística | Concept: Medições Básicas
% Type: estatistica_pura | Difficulty: 1/5
% Tags: media, moda, mediana, calculo_direto
% Author: Professor | Date: 2025-11-25

\exercicio{
Considera o seguinte conjunto de dados:

\begin{center}
$\{5, 8, 12, 8, 6, 10, 8, 9\}$
\end{center}
}

\subexercicio{Calcula a média aritmética dos valores.}

\subexercicio{Determina a moda do conjunto de dados.}

\subexercicio{Ordena os valores e calcula a mediana.}

\subexercicio{Qual das três medidas (média, moda ou mediana) melhor representa o ``centro'' destes dados? Justifica a tua resposta.}
\FloatBarrier

% Exercício 9: MAT_P2ESTATI_1MED_EPU_002.tex
% Exercise ID: MAT_P2ESTATI_1MED_EPU_002
% Module: Módulo P2 - Estatística | Concept: Medições Básicas
% Type: estatistica_pura | Difficulty: 2/5
% Tags: media, moda, mediana, calculo_direto
% Author: Professor | Date: 2025-11-25

\exercicio{
Considera o seguinte conjunto de dados:

\begin{center}
$\{3, 7, 7, 10, 15, 7, 12, 9, 11\}$
\end{center}
}

\subexercicio{Calcula a média aritmética. Apresenta o resultado com uma casa decimal.}

\subexercicio{Identifica a moda. Quantas vezes aparece esse valor?}

\subexercicio{Calcula a mediana. Nota: este conjunto tem um número ímpar de elementos.}

\subexercicio{Se adicionarmos o valor 100 ao conjunto, como achas que isso afetaria a média e a mediana? Calcula os novos valores e comenta a diferença.}
\FloatBarrier

% Exercício 10: MAT_P2ESTATI_1MX_EPX_001.tex
% Exercise ID: MAT_P2ESTATI_1MX_EPX_001
% Module: Módulo P2 - Estatística | Concept: Medições Básicas | Type: Estatística Pura
% Difficulty: 4/5 (Difícil) | Format: desenvolvimento
% Tags: calculo_direto, tabelas, mediana, media, moda, calculos_com_dados, teste, automacao
% Author: Teste Automático | Date: 2025-11-26
% Status: active

\exercicio{Exercício de teste para estatistica_pura}
\FloatBarrier

\end{document}
