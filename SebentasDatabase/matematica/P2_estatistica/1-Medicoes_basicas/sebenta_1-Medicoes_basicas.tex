% Template para geração automática de sebentas
% Gerado automaticamente - NÃO EDITAR MANUALMENTE
\documentclass[12pt,a4paper]{article}

% Encoding e idioma
\usepackage[utf8]{inputenc}
\usepackage[T1]{fontenc}
\usepackage[portuguese]{babel}

% Matemática
\usepackage{amsmath}
\usepackage{amssymb}
\usepackage{amsthm}
\usepackage{mathtools}

% Gráficos e figuras
\usepackage{graphicx}
\usepackage{tikz}
\usetikzlibrary{calc,patterns,angles,quotes}
\usepackage{pgfplots}
\pgfplotsset{compat=1.18}
% Force float barriers when needed to keep figures right after exercises
\usepackage{placeins}
\usepackage{float}  % Para figure[H] - força figuras exatamente onde definidas

% Layout e formatação
\usepackage{geometry}
\geometry{a4paper,margin=2.5cm,top=3cm,bottom=3cm}
\usepackage{fancyhdr}
\usepackage{enumitem}
\usepackage{multicol}
\usepackage{booktabs}

% Hyperlinks e referências
\usepackage{hyperref}
\hypersetup{
    colorlinks=true,
    linkcolor=blue,
    urlcolor=blue,
    citecolor=blue
}

% Sistema de exercícios - macros personalizadas
\newcounter{exerciciocount}
\newcounter{subexerciciocount}
\newcounter{optioncount}

\newcommand{\exercicio}[1][]{% 
    \par\vspace{1.5em}%
    \refstepcounter{exerciciocount}%
    \setcounter{subexerciciocount}{0}%
    \setcounter{optioncount}{0}%
    \noindent\textbf{Exercício~\theexerciciocount.}%
    \ifx&#1&%
        % Argumento vazio - o conteúdo vem depois
        \par\vspace{0.3em}%
    \else%
        % Argumento fornecido - incluir inline
        \ #1\par\vspace{0.5em}%
    \fi%
}

\newcommand{\subexercicio}[1]{%
    \par\vspace{0.8em}%
    \refstepcounter{subexerciciocount}%
    \noindent\textbf{\theexerciciocount.\thesubexerciciocount.} #1\par\vspace{0.3em}%
}

\newcommand{\option}[1]{%
    \par
    \refstepcounter{optioncount}%
    \noindent(\alph{optioncount}) #1%
}

% Campos para respostas
\newcommand{\campo}[1][2.0cm]{\makebox[#1]{\hrulefill}}
\newcommand{\campoLetra}[1][1.2cm]{\makebox[#1]{\hrulefill}}
\newcommand{\campoCents}[1][3.0cm]{\makebox[#1]{\hrulefill}}% Cabeçalho e rodapé
\pagestyle{fancy}
\fancyhf{}
\fancyhead[L]{Módulo P2 - Estatística}
\fancyhead[R]{1 - Medicoes basicas}
\fancyfoot[C]{\thepage}

% Metadados do documento
\title{}
\author{}
\date{}

\begin{document}

% Remover título, autor e data - apenas conteúdo
\thispagestyle{fancy}

\section*{1 - Medicoes basicas}

\subsection*{Tipos de Exercícios}
\begin{itemize}
  \item \textbf{Estatística na Vida Quotidiana} --- Exercícios aplicando medidas estatísticas a valores do dia a dia de alunos e pessoas (notas, gastos, tempos, etc.).
  \item \textbf{Estatística Aplicada à Poupança} --- Exercícios aplicando medidas estatísticas (média, moda, mediana) a tabelas que cronometram poupanças de família, pessoa ou instituição.
  \item \textbf{Estatística Pura} --- Exercícios diretos com listas de valores numéricos para calcular medidas estatísticas (média, moda, mediana).
\end{itemize}

\vspace{1em}

% Exercício 1: MAT_P2ESTATI_1MED_ENV_001.tex
% Exercise ID: MAT_P2ESTATI_1MED_ENV_001
% Module: Módulo P2 - Estatística | Concept: Medições Básicas
% Type: estatistica_na_vida | Difficulty: 2/5
% Tags: media, moda, mediana, notas, interpretacao
% Author: Professor | Date: 2025-11-25

\exercicio{
O João registou as suas notas (de 0 a 20 valores) nas primeiras 7 fichas de avaliação do ano letivo:

\begin{center}
\begin{tabular}{|c|c|c|c|c|c|c|}
\hline
\textbf{Ficha 1} & \textbf{Ficha 2} & \textbf{Ficha 3} & \textbf{Ficha 4} & \textbf{Ficha 5} & \textbf{Ficha 6} & \textbf{Ficha 7} \\
\hline
14 & 12 & 15 & 12 & 16 & 14 & 12 \\
\hline
\end{tabular}
\end{center}
}

\subexercicio{Calcula a média das notas do João. Arredonda à unidade.}

\subexercicio{Determina a moda das notas. O que significa este valor para o desempenho do João?}

\subexercicio{Calcula a mediana das notas. Mostra os valores ordenados.}

\subexercicio{Se o João precisar de ter pelo menos 13 valores de média para passar, qual é a nota mínima que precisa na próxima ficha? Justifica.}
\FloatBarrier

% Exercício 2: MAT_P2ESTATI_1MED_ENV_002.tex
% Exercise ID: MAT_P2ESTATI_1MED_ENV_002
% Module: Módulo P2 - Estatística | Concept: Medições Básicas
% Type: estatistica_na_vida | Difficulty: 2/5
% Tags: media, moda, mediana, gastos, interpretacao
% Author: Professor | Date: 2025-11-25

\exercicio{
A Maria registou os seus gastos semanais em transporte (em euros) durante um mês:

\begin{center}
\begin{tabular}{|c|c|c|c|}
\hline
\textbf{Semana 1} & \textbf{Semana 2} & \textbf{Semana 3} & \textbf{Semana 4} \\
\hline
15 & 18 & 15 & 22 \\
\hline
\end{tabular}
\end{center}
}

\subexercicio{Calcula a média dos gastos semanais da Maria.}

\subexercicio{Determina a moda dos gastos. Qual o significado prático deste valor?}

\subexercicio{Calcula a mediana. Compara-a com a média.}

\subexercicio{A Maria tem um orçamento mensal de 70€ para transporte. Considerando os dados, achas que ela consegue cumprir o orçamento? Justifica com base nos valores estatísticos calculados.}
\FloatBarrier

% Exercício 3: MAT_P2ESTATI_1MED_EPO_001.tex
% Exercise ID: MAT_P2ESTATI_1MED_EPO_001
% Module: Módulo P2 - Estatística | Concept: Medições Básicas
% Type: estatistica_poupanca | Difficulty: 2/5
% Tags: media, moda, mediana, poupanca, interpretacao
% Author: Professor | Date: 2025-11-25

\exercicio{
A família Silva registou as suas poupanças mensais (em euros) durante 6 meses:

\begin{center}
\begin{tabular}{|c|c|c|c|c|c|}
\hline
\textbf{Jan} & \textbf{Fev} & \textbf{Mar} & \textbf{Abr} & \textbf{Mai} & \textbf{Jun} \\
\hline
150 & 200 & 150 & 180 & 200 & 220 \\
\hline
\end{tabular}
\end{center}
}

\subexercicio{Calcula a média das poupanças mensais da família Silva.}

\subexercicio{Determina a moda dos valores registados. Explica o que significa este valor no contexto da poupança.}

\subexercicio{Calcula a mediana das poupanças. Mostra os passos do cálculo.}

\subexercicio{Compara a média e a mediana obtidas. O que podes concluir sobre a distribuição das poupanças da família?}
\FloatBarrier

% Exercício 4: MAT_P2ESTATI_1MED_EPO_002.tex
% Exercise ID: MAT_P2ESTATI_1MED_EPO_002
% Module: Módulo P2 - Estatística | Concept: Medições Básicas
% Type: estatistica_poupanca | Difficulty: 2/5
% Tags: media, moda, mediana, poupanca, interpretacao
% Author: Professor | Date: 2025-11-25

\exercicio{
Uma associação de jovens decidiu poupar para uma viagem de finalistas. A tabela seguinte mostra o valor poupado (em euros) por cada membro durante um mês:

\begin{center}
\begin{tabular}{|l|c|}
\hline
\textbf{Membro} & \textbf{Poupança (€)} \\
\hline
Ana & 25 \\
Bruno & 30 \\
Carla & 25 \\
Daniel & 40 \\
Eva & 30 \\
Filipe & 25 \\
Gonçalo & 35 \\
Helena & 30 \\
\hline
\end{tabular}
\end{center}
}

\subexercicio{Calcula a média das poupanças do grupo.}

\subexercicio{Determina a moda. Quantos membros pouparam esse valor?}

\subexercicio{Calcula a mediana das poupanças. Ordena primeiro os valores.}

\subexercicio{Se o objetivo é que cada membro contribua com pelo menos 30€, quantos membros ficaram abaixo desse valor? Usa a mediana para justificar se o grupo está no bom caminho.}
\FloatBarrier

% Exercício 5: MAT_P2ESTATI_1MED_EPU_001.tex
% Exercise ID: MAT_P2ESTATI_1MED_EPU_001
% Module: Módulo P2 - Estatística | Concept: Medições Básicas
% Type: estatistica_pura | Difficulty: 1/5
% Tags: media, moda, mediana, calculo_direto
% Author: Professor | Date: 2025-11-25

\exercicio{
Considera o seguinte conjunto de dados:

\begin{center}
$\{5, 8, 12, 8, 6, 10, 8, 9\}$
\end{center}
}

\subexercicio{Calcula a média aritmética dos valores.}

\subexercicio{Determina a moda do conjunto de dados.}

\subexercicio{Ordena os valores e calcula a mediana.}

\subexercicio{Qual das três medidas (média, moda ou mediana) melhor representa o ``centro'' destes dados? Justifica a tua resposta.}
\FloatBarrier

% Exercício 6: MAT_P2ESTATI_1MED_EPU_002.tex
% Exercise ID: MAT_P2ESTATI_1MED_EPU_002
% Module: Módulo P2 - Estatística | Concept: Medições Básicas
% Type: estatistica_pura | Difficulty: 2/5
% Tags: media, moda, mediana, calculo_direto
% Author: Professor | Date: 2025-11-25

\exercicio{
Considera o seguinte conjunto de dados:

\begin{center}
$\{3, 7, 7, 10, 15, 7, 12, 9, 11\}$
\end{center}
}

\subexercicio{Calcula a média aritmética. Apresenta o resultado com uma casa decimal.}

\subexercicio{Identifica a moda. Quantas vezes aparece esse valor?}

\subexercicio{Calcula a mediana. Nota: este conjunto tem um número ímpar de elementos.}

\subexercicio{Se adicionarmos o valor 100 ao conjunto, como achas que isso afetaria a média e a mediana? Calcula os novos valores e comenta a diferença.}
\FloatBarrier


\end{document}
