% Título e informações do exame
\title{Questão de aula do MÓDULO P2 - Estatística}
\author{EPRALIMA - Escola Profissional Alto Lima}

\date{}

% Cabeçalho completo do teste dentro de uma caixa arredondada
\newcommand{\espacoAluno}{%
    \vspace{0.5cm}
    \begin{tcolorbox}[colback=white,colframe=black,arc=6pt,boxrule=0.8pt]
        \noindent\textbf{Nome do Aluno:} \rule{9.5cm}{0.4pt} \textbf{Turma:} \rule{1cm}{0.4pt}\\[0.5cm]
        \noindent\textbf{Assinatura do Professor:} \rule{3cm}{0.4pt} \hfill \textbf{Nota:} \rule{2cm}{0.4pt}\\[0.5cm]
        \noindent\textbf{Assinatura do Encarregado de Educação:} \rule{3cm}{0.4pt}
    \end{tcolorbox}
    \vspace{1cm}
}

% Minimal macros expected by Exercise fragments
% Create a boolean flag to control whether the automatic heading is shown.
% If the flag macros are already defined elsewhere, do not redefine them.
\makeatletter
\@ifundefined{showexerciciotitletrue}{%
        \newif\ifshowexerciciotitle
        \showexerciciotitletrue
}{}
\makeatother

% Provide a safe, robust definition for \exercicio if it is not defined.
% Add a counter `exercicio` so each call increments and can be referenced.
% Use \DeclareRobustCommand so it behaves well in moving arguments.
\makeatletter
% Define the counter unless already present
\@ifundefined{c@exercicio}{%
    \newcounter{exercicio}
}{}

% Define the macro only if it's not defined already
\@ifundefined{exercicio}{%
    \DeclareRobustCommand{\exercicio}[1]{%
        \refstepcounter{exercicio}% increment the counter and make it ref'able
        \ifshowexerciciotitle
            \par\noindent\textbf{Exercício~\theexercicio: }#1\par
        \else
            #1\par
        \fi
    }%
}{}
\makeatother