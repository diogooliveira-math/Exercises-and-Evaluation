% Título e informações do exame
\title{Questão de aula do MÓDULO P4 - Funções}
\author{EPRALIMA - Escola Profissional Alto Lima}

\date{}

% Cabeçalho completo do teste dentro de uma caixa arredondada
\newcommand{\espacoAluno}{%
    \vspace{0.5cm}
    \begin{tcolorbox}[colback=white,colframe=black,arc=6pt,boxrule=0.8pt]
        \noindent\textbf{Nome do Aluno:} \rule{9.5cm}{0.4pt} \textbf{Turma:} \rule{1cm}{0.4pt}\\[0.5cm]
        \noindent\textbf{Assinatura do Professor:} \rule{3cm}{0.4pt} \hfill \textbf{Nota:} \rule{2cm}{0.4pt}\\[0.5cm]
        \noindent\textbf{Assinatura do Encarregado de Educação:} \rule{3cm}{0.4pt}
    \end{tcolorbox}
    \vspace{1cm}
}

% Minimal macros expected by Exercise fragments
% Provide a safe definition for \exercicio if not already defined.
% Provide a boolean flag to control whether the automatic heading is shown.
\makeatletter
\@ifundefined{showexerciciotitletrue}{%
    \newif\ifshowexerciciotitle
    \showexerciciotitletrue
}{}

% Exercicio counter (numbering)
\providecommand{\resetExercicioCounter}{\setcounter{exercicio}{0}} % optional reset command
\@ifundefined{exercicio}{\newcounter{exercicio}}{} % define counter if not present
\makeatother

% Provide a safe definition for \exercicio if not already defined.
% The macro respects the \ifshowexerciciotitle flag: when false it prints only the content.
\providecommand{\exercicio}[1]{%
    \ifshowexerciciotitle
        \refstepcounter{exercicio}%
        \par\noindent\textbf{Exercício \theexercicio: }#1\par
    \else
        #1\par
    \fi
}