\exercicio{Considere o triângulo $ABC$ retângulo em $C$, onde os pontos $A$ e $B$ pertencem ao gráfico da função linear $f(x) = 0.5x + 2$.}

\begin{center}
\begin{tikzpicture}[scale=1.2]
    % Eixos principais
    \draw[thick,->] (-0.5,0) -- (9.5,0) node[right] {$x$};
    \draw[thick,->] (0,-0.5) -- (0,7.5) node[above] {$y$};
    
    % Marcações nos eixos
    \foreach \x in {1,2,3,4,5,6,7,8,9}
        \draw (\x,0.1) -- (\x,-0.1) node[below] {$\x$};
    
    % Origem
    \node[below left] at (0,0) {$0$};
    
    % Função linear f(x) = 0.5x + 2
    \draw[thick, blue, domain=-1:10] plot (\x, {0.5*\x + 2}) node[right] {$f(x) = 0.5x + 2$};
    
    % Triângulo ABC (sem coordenadas visíveis)
    \coordinate (A) at (2,3);
    \coordinate (B) at (5,4.5);
    \coordinate (C) at (5,3);
    
    % Desenhar o triângulo
    \draw[thick, red] (A) -- (B) -- (C) -- cycle;
    
    % Pontos marcados (sem coordenadas)
    \fill[red] (A) circle (3pt) node[below left] {$A$};
    \fill[red] (B) circle (3pt) node[above right] {$B$};
    \fill[red] (C) circle (3pt) node[right] {$C$};
\end{tikzpicture}
\end{center}

\subexercicio{Observe o gráfico seguinte e determine as coordenadas dos vértices do triângulo.}

\vspace{4cm}

\subexercicio{Calcule o ângulo BÂC.}