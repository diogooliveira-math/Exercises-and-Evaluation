\documentclass[12pt]{article}
\usepackage{amsmath}
\usepackage{tikz}
\usepackage{multicol}
\usepackage[portuguese]{babel}
\usepackage[margin=2cm]{geometry}
\usepackage{enumitem} % carregado depois de babel

\usepackage{array}
\usepackage{booktabs}

\usepackage{fancyhdr}   % <-- Adicionado para suportar \pagestyle{fancy}

% Comando para campo de resposta
\newcommand{\campoLetra}[1][1.2cm]{\makebox[#1]{\hrulefill}}

% Comando para caixa (marcar com X)
\newcommand{\caixa}[1][0.8cm]{\fbox{\rule{0pt}{0.6em}\hspace{#1}}}

\newcommand{\campo}[1][2.0cm]{\makebox[#1]{\hrulefill}}
\newcommand{\campoCents}[1][3.0cm]{\makebox[#1]{\hrulefill}}

\pagestyle{fancy}
\fancyhf{}
\rhead{Matemática, Módulo P1: Modelos matemáticos para a cidadania}
\lhead{Ficha de exercícios nº1}
\cfoot{\thepage}
\thispagestyle{fancy}


\usepackage{fourier}    % Utopia + math support
\everymath{\displaystyle}

\newcounter{exercicio} % creates a new counter
\setcounter{exercicio}{0} % optional: start from 0

\newcommand{\exercicio}{%
    \refstepcounter{exercicio}% increment counter
    \section*{Exercício \theexercicio}%
}

% Define keys and defaults
\pgfkeys{
  /referencial/.is family, /referencial,
  xmin/.initial=-1,
  xmax/.initial=5,
  ymin/.initial=-1,
  ymax/.initial=5,
  xstep/.initial=1,
  ystep/.initial=1,
}

\tikzset{every picture/.style={scale=1.2, line width=1.7pt}} %Sets globally the personal configurations for drawns that is perfect for printing


% Command with key–value options
% Example: \drawReferencial[xmin=-2, xmax=6, ymin=-1, ymax=4, xstep=2, ystep=1]
\newcommand{\drawReferencial}[1][]{%
  \pgfkeys{/referencial, #1} % load defaults + user options

  % Draw axes
  \draw[->] (\pgfkeysvalueof{/referencial/xmin},0) --
            (\pgfkeysvalueof{/referencial/xmax},0) node[right] {$x$};
  \draw[->] (0,\pgfkeysvalueof{/referencial/ymin}) --
            (0,\pgfkeysvalueof{/referencial/ymax}) node[above] {$y$};

  % Ticks on x-axis
  \foreach \x in {\pgfkeysvalueof{/referencial/xmin},
                  \pgfkeysvalueof{/referencial/xstep},...,
                  \pgfkeysvalueof{/referencial/xmax}}
    \draw (\x,0.1) -- (\x,-0.1) node[below] {\x};

  % Ticks on y-axis
  \foreach \y in {\pgfkeysvalueof{/referencial/ymin},
                  \pgfkeysvalueof{/referencial/ystep},...,
                  \pgfkeysvalueof{/referencial/ymax}}
    \draw (0.1,\y) -- (-0.1,\y) node[left] {\y};
}

% Comando para um campo de resposta sublinhado
\newcommand{\respostaCampo}[1][2.0cm]{\underline{\hspace{#1}}}

% Comando para caixa de escolha (marcar com X)
\newcommand{\caixaV}[1][0.6cm]{\fbox{\rule{0pt}{0.6em}\hspace{#1}}}
\newcommand{\caixaF}[1][0.6cm]{\fbox{\rule{0pt}{0.6em}\hspace{#1}}}


\date{}
\begin{document}

\exercicio
Instrucções: Para cada afirmação, escreve ``V'' se considerares a afirmação verdadeira ou ``F'' se considerares falsa.

\begin{enumerate}[label=\textbf{(\arabic*)}, leftmargin=*, itemsep=1.0\baselineskip]
  \item A maioria simples exige que um candidato obtenha mais votos do que qualquer outro, independentemente do número total de votos válidos.\\
\textbf{V} \ \ \caixaV \quad \textbf{F} \ \ \caixaF

  \item A maioria absoluta significa obter mais de metade dos votos válidos.\\
\textbf{V} \ \ \caixaV \quad \textbf{F} \ \ \caixaF

  \item Votos nulos não entram na contagem dos votos válidos.\\
\textbf{V} \ \ \caixaV \quad \textbf{F} \ \ \caixaF

  \item A abstenção corresponde a eleitores que se registaram mas não compareceram à votação.\\
\textbf{V} \ \ \caixaV \quad \textbf{F} \ \ \caixaF

  \item Para uma eleição que exige maioria absoluta, se nenhum candidato obtiver mais de metade dos votos válidos, realiza-se um segundo turno entre os dois mais votados.\\
\textbf{V} \ \ \caixaV \quad \textbf{F} \ \ \caixaF

  \item Se um eleitor deposita uma folha de voto em que não marca nenhum candidato, esse voto é considerado voto em branco.\\
\textbf{V} \ \ \caixaV \quad \textbf{F} \ \ \caixaF
\end{enumerate}

\exercicio
\begin{minipage}[t]{0.58\textwidth}
\begin{enumerate}[label=\textbf{(\arabic*)}, leftmargin=*, itemsep=10pt]
  \item Maioria simples
  \item Maioria absoluta
  \item Voto nulo
  \item Voto em branco
  \item Abstenção
\end{enumerate}
\end{minipage}
\hfill
\begin{minipage}[t]{0.38\textwidth}
\begin{enumerate}[label=\textbf{\Alph*}, leftmargin=*, itemsep=8pt]
  \item Votos expressos sem contar abstenções; maioria superior a metade dos votos válidos.
  \item Voto depositado que não assinala qualquer opção; em muitos sistemas é contabilizado separadamente dos nulos.
  \item Exige mais votos que qualquer outro concorrente; não precisa de exceder metade dos votos válidos.
  \item Voto inválido por irregularidade na folha (marcações contraditórias, sinais identificadores, etc.).
  \item Eleitor inscrito que não comparece à mesa de voto no dia da eleição.
\end{enumerate}
\end{minipage}

\vspace{8pt}
\noindent Respostas: 1. \campoLetra \quad 2. \campoLetra \quad 3. \campoLetra \quad 4. \campoLetra \quad 5. \campoLetra


\exercicio

Instruções: resolve cada exercício e escreve a resposta nos campos indicados. Usa cêntimos quando for pedido.

\begin{enumerate}[label=\textbf{\arabic*.}, leftmargin=*, itemsep=12pt]

\item O salário base é \,870\,€. Vai aumentar 6{,}1\%. Quanto ficará o salário?\\
Resposta: \campoCents

Regra de três (interpretação: \(6{,}1\%\) de \(870\) é quanto):


\[
\begin{array}{rcl}
100\% &\longleftrightarrow& 870\\

6{,}1\% &\longleftrightarrow& x
\end{array}
\]


Resolução:


\[
x=\dfrac{6{,}1\times 870}{100}=\dfrac{6{,}1\times870}{100}=53{,}07\ \text{€}.
\]


Valor final \(=870+53{,}07=\mathbf{923{,}07\ \text{€}}\).

\item O salário base é \,1\,245\,€. Aumenta 4{,}75\%. Quanto ficará o salário (arredonda a cêntimos)?\\

\vspace{3cm}

Resposta: \campoCents

\item Um produto custa \,320\,€. Aplica-se um desconto de 12{,}5\%. Qual é o preço depois do desconto?\\

\vspace{3cm}

Resposta: \campoCents

\item O valor da Netflix passou de \,5{,}00\,€ para \,7{,}50\,€. Qual foi a percentagem do aumento?\\
Resposta: \campo[2.5cm]

Regra de três (interpretação: \(2,5€\) de \(5\) é quanto):


\[
\begin{array}{rcl}
100\% &\longleftrightarrow& 5€\\

x &\longleftrightarrow& 2,5€
\end{array}
\]


Resolução:


\[
x=\dfrac{6{,}1\cdot 870}{100}=\dfrac{6{,}1\times870}{100}=53{,}07\ \text{€}.
\]


Valor final \(=870+53{,}07=\mathbf{923{,}07\ \text{€}}\).

\item Um passe mensal custa \,60\,€ e passou a custar \,51\,€. Qual foi a percentagem de redução?\\

\vspace{3cm}

Resposta: \campo[2.5cm]

\item Um salário de \,900\,€ aumenta 3\% no mês 1 e mais 2{,}5\% no mês 2. Qual é o salário final (arredonda a cêntimos)?\\

\vspace{3cm}

Resposta: \campoCents

\item Depois de um aumento de 20\% um produto custa \,180\,€. Quanto custava antes do aumento?\\

\vspace{3cm}

Resposta: \campoCents


\end{enumerate}

\bigskip

\section*{Eleições}

Instruções: para cada tabela responde às perguntas escritas. Escreve as contas breves e a resposta nos campos indicados.

\exercicio
\begin{center}
\begin{tabular}{lrr}
\toprule
\textbf{Candidato} & \textbf{Votos} & \textbf{\% (para preencher)} \\
\midrule
A & 420 & \\
B & 310 & \\
C & 170 & \\
\midrule
\textbf{Total votos válidos} & 900 & \\
Votos em branco & 25 & \\
Votos nulos & 10 & \\
\bottomrule
\end{tabular}
\end{center}

1. Alguém obteve maioria simples? Resposta e cálculo: \campo[6.0cm]  

2. Alguém obteve maioria absoluta? Resposta e cálculo: \campo[6.0cm]

\exercicio
\begin{center}
\begin{tabular}{lrr}
\toprule
\textbf{Candidato} & \textbf{Votos} & \textbf{\% (para preencher)} \\
\midrule
D & 1\,150 & \\
E & 980 & \\
F & 720 & \\
G & 150 & \\
\midrule
\textbf{Total votos válidos} & 3\,000 &  \\
Votos em branco & 60 & \\
Votos nulos & 40 & \\
\bottomrule
\end{tabular}
\end{center}

1. Alguém obteve maioria simples? Resposta e cálculo: \campo[6.0cm]  

2. Alguém obteve maioria absoluta? Resposta e cálculo: \campo[6.0cm]

3. Qual foi a percentagem de votos brancos? E nulos? \campo[6.0cm]

\exercicio
\begin{center}
\begin{tabular}{lrr}
\toprule
\textbf{Candidato} & \textbf{Votos} & \textbf{\% (para preencher)} \\
\midrule
H & 2\,300 & \\
I & 1\,900 & \\
\midrule
\textbf{Total votos válidos} & 4\,200 & \\
Votos em branco & 150 & \\
Votos nulos & 50 & \\
Eleitores inscritos & 6\,000 & \\
\bottomrule
\end{tabular}
\end{center}

1. Alguém obteve maioria simples? Resposta e cálculo: \campo[6.0cm]  

2. Alguém obteve maioria absoluta? Resposta e cálculo: \campo[6.0cm]  

3. Qual é a taxa de abstenção (em percentagem)? Resposta e cálculo: \campo[6.0cm]

4. Qual foi a percentagem de votos brancos? E nulos? \campo[6.0cm]

\exercicio
\begin{center}
\begin{tabular}{lrr}
\toprule
\textbf{Candidato} & \textbf{Votos} & \textbf{\% (para preencher)}\\
\midrule
J & 340 & \\
K & 330 & \\
L & 310 & \\
M & 20 & \\
\midrule
\textbf{Total votos válidos} & 1\,000 & \\
Votos em branco & 5 & \\
Votos nulos & 0 & \\
\bottomrule
\end{tabular}
\end{center}

1. Alguém obteve maioria simples? Resposta e cálculo: \campo[6.0cm]  

2. Alguém obteve maioria absoluta? Resposta e cálculo: \campo[6.0cm]

3. Qual foi a percentagem de votos brancos? E nulos? \campo[6.0cm]
\exercicio
\begin{center}
\begin{tabular}{lrr}
\toprule
\textbf{Candidato} & \textbf{\%} & \textbf{Votos (para preencher)}\\
\midrule
A & 38\% & \\
B & 34\% & \\
C & 25\% & \\
D & 3\% & \\
\midrule
\textbf{Total votos válidos} & & 1\,200\\
Votos em branco & & 10\\
Votos nulos & & 5\\
\bottomrule
\end{tabular}
\end{center}

1. Alguém obteve maioria simples? Resposta e cálculo: \campo[6.0cm]

2. Alguém obteve maioria absoluta? Resposta e cálculo: \campo[6.0cm]

3. Qual foi a percentagem de votos brancos? E nulos? \campo[6.0cm]

\vspace{1cm}

\exercicio
\begin{center}
\begin{tabular}{lrr}
\toprule
\textbf{Candidato} & \textbf{\%} & \textbf{Votos (para preencher)}\\
\midrule
E & 40\% & \\
F & 30\% & \\
G & 20\% & \\
H & 10\% & \\
\midrule
\textbf{Total votos válidos} & & 2\,000\\
Votos em branco & & 15\\
Votos nulos & & 5\\
\bottomrule
\end{tabular}
\end{center}

1. Alguém obteve maioria simples? Resposta e cálculo: \campo[6.0cm]

2. Alguém obteve maioria absoluta? Resposta e cálculo: \campo[6.0cm]

3. Qual foi a percentagem de votos brancos? E nulos? \campo[6.0cm]
\vspace{1cm}

\exercicio
\begin{center}
\begin{tabular}{lrr}
\toprule
\textbf{Candidato} & \textbf{\%} & \textbf{Votos (para preencher)}\\
\midrule
I & 45\% & \\
J & 35\% & \\
K & 15\% & \\
L & 5\% & \\
\midrule
\textbf{Total votos válidos} & & 800\\
Votos em branco & & 20\\
Votos nulos & & 0\\
\bottomrule
\end{tabular}
\end{center}

1. Alguém obteve maioria simples? Resposta e cálculo: \campo[6.0cm]

2. Alguém obteve maioria absoluta? Resposta e cálculo: \campo[6.0cm]

3. Qual foi a percentagem de votos brancos? E nulos? \campo[6.0cm]

\end{document}