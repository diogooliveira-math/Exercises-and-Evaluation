\documentclass[a4paper,12pt]{article}
\usepackage[utf8]{inputenc}
\usepackage[portuguese]{babel}
\usepackage{geometry}
\usepackage{enumitem}
\usepackage{amsmath}
\usepackage{pgfplots}
\usepackage{hyperref}
\geometry{margin=2.5cm}


\pgfplotsset{compat=1.17}

\title{Desafio de Investigação Matemática}
\date{}

\begin{document}

\maketitle

\section*{Tema}
Crescimento exponencial em contextos reais!

\section*{Objetivos}

Cada grupo deverá apresentar um relatório que demonstre:

\begin{itemize}
  \item Compreensão do comportamento das funções exponenciais.
  \item Capacidade de aplicar modelos matemáticos a situações reais.
  \item Interpretação gráfica de fenómenos de crescimento.
\end{itemize}

\textbf{Nota:} O relatório deve incluir os \textbf{nomes dos participantes do grupo}.

\section*{Contexto do Problema}

Numa reserva natural, a população de coelhos evolui segundo a fórmula:
\[
C(t) = 500 \times 1{,}1^t
\]
onde \( t \) representa o número de anos desde o início de 1998.

\section*{Tarefa 1: Desenhar o Gráfico}

Representa graficamente a função \( C(t) \).

\begin{center}
\begin{tikzpicture}
\begin{axis}[
  axis lines = middle,
  xlabel = {},
  ylabel = {},
  xtick = \empty,
  ytick = \empty,
  xmin = -10, xmax = 10,
  ymin = -10, ymax = 10,
  grid = none,
  enlargelimits = true,
  axis line style = {->, thick},
]
% Aqui podes adicionar uma função, se quiseres:
% \addplot[domain=-10:10, thick, blue]{2^x};
\end{axis}
\end{tikzpicture}
\end{center}


\section*{Tarefa 2: Interpretar o Gráfico}

\begin{itemize}
  \item O gráfico é crescente ou decrescente?
  \item O que significa esse comportamento no contexto da população de coelhos?
\end{itemize}

\section*{Tarefa 3: Interpretar Pontos}

Escolhe três pontos do gráfico e explica o seu significado.

\section*{Tarefa 4: Resolver Problemas}

\begin{enumerate}
  \item Quantos coelhos existiam no início de 2000?
  \item Em que ano e mês terão sido atingidos 1300 coelhos?
\end{enumerate}

Mostra todos os cálculos e raciocínios!

\section*{Tarefa 5: Criar o Teu Próprio Problema}

Inventa um problema relacionado com crescimento exponencial e apresenta a sua resolução. Podes usar outro contexto (população, dinheiro, tecnologia, etc.) — sê criativo!

\end{document}
