\documentclass[12pt]{article}
\usepackage[a4paper,margin=2.5cm]{geometry}
\usepackage{amsmath,amssymb}
\usepackage{tikz}
\usepackage{graphicx}
\usepackage{enumitem}
\usepackage{fancyhdr}
\usepackage{titlesec}
\usepackage{multicol}
\usepackage{fourier}    % Utopia + math support
\everymath{\displaystyle}
\usepackage[portuguese]{babel}


\usepackage{textcomp} % Para o símbolo do euro






\usepackage{array}
\usepackage{booktabs}

\usepackage{fancyhdr}   % <-- Adicionado para suportar \pagestyle{fancy}



\usepackage{fourier}    % Utopia + math support


% Comando para campo de resposta
\newcommand{\campoLetra}[1][1.2cm]{\makebox[#1]{\hrulefill}}

% Comando para caixa (marcar com X)
\newcommand{\caixa}[1][0.8cm]{\fbox{\rule{0pt}{0.6em}\hspace{#1}}}

\newcommand{\campo}[1][2.0cm]{\makebox[#1]{\hrulefill}}
\newcommand{\campoCents}[1][3.0cm]{\makebox[#1]{\hrulefill}}


% Comando para caixa de escolha (marcar com X)
\newcommand{\caixaV}[1][0.6cm]{\fbox{\rule{0pt}{0.6em}\hspace{#1}}}
\newcommand{\caixaF}[1][0.6cm]{\fbox{\rule{0pt}{0.6em}\hspace{#1}}}


\DeclareUnicodeCharacter{202F}{~}


\tikzset{every picture/.style={scale=1.2, line width=1.7pt}} %Sets globally the personal configurations for drawns that is perfect for printing

\newcounter{exercicio}


\newcommand{\exercicio}{%
    \refstepcounter{exercicio}%
    \section*{Exercício \theexercicio}%
}

\pagestyle{fancy}
\fancyhf{}
\rhead{Matemática, Módulo P1: Modelos matemáticos para a cidadania }
\lhead{Questão de Aula nº1}
\cfoot{\thepage}

\titleformat{\section}{\normalfont\Large\bfseries}{}{0em}{}

\begin{document}

\begin{center}
    \vspace{0.5cm}
    Nome: \rule{10cm}{0.4pt} \hfill Turma: \rule{3cm}{0.4pt}
\end{center}

\vspace{0.8cm}


\exercicio

Considera a seguinte correspondência:



\[
\begin{array}{c|c}
\text{Pessoa} & \text{Número de sapatos que calça} \\
\hline
\text{Ana} & 37 \\
\text{Bruno} & 42 \\
\text{Carla} & 39 \\
\text{David} & 42 \\
\end{array}
\]



\textbf{Pergunta:} Esta correspondência é uma função? Justifica a tua resposta escolhendo a opção correta e explicando por que as outras estão erradas.

\begin{enumerate}
    \item[(A)] Não é uma função, porque o número 42 aparece duas vezes.
    \item[(B)] É uma função, porque cada pessoa está associada a um único número.
    \item[(C)] Não é uma função, porque Bruno e David calçam o mesmo número.
    \item[(D)] Não é uma função, porque há números repetidos na segunda coluna.
\end{enumerate}

\vspace{5 cm}

\exercicio

Considera a função $f(x) = 2x - 3$.

\begin{itemize}
    \item[a)] Calcula $f(0)$, $f(2)$ e $f(5)$.

\vspace{3 cm}

    \item[b)] Qual é o valor de $x$ tal que $f(x) = 1$?

\vspace{3 cm}
\end{itemize}

\exercicio

Uma receita de bolo de iogurte para \textbf{2 pessoas} leva os seguintes ingredientes:

\begin{itemize}
    \item 1 iogurte natural
    \item 2 copos de açúcar
    \item 3 copos de farinha
    \item 1/2 copo de óleo
    \item 3 ovos
\end{itemize}

Pretende-se fazer esta receita para \textbf{10 pessoas}. 

\vspace{5 cm}

\exercicio

O Clube A comprou um jogador por 12 milhões de euros, o que representa 20\% do seu orçamento anual.

\bigskip


\end{document}

